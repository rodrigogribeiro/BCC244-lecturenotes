\documentclass[a4paper]{article}

\usepackage[portuguese]{babel}
\usepackage[utf8]{inputenc}
\usepackage{graphicx,hyperref}
\usepackage{float}
\usepackage{listings}
\usepackage{proof,tikz}
\usepackage{amssymb,amsthm,stmaryrd}


\usepackage[edges]{forest}
\usetikzlibrary{automata, positioning, arrows}


\newtheorem{Lemma}{Lema}
\newtheorem{Theorem}{Teorema}
\theoremstyle{definition}
\newtheorem{Example}{Exemplo}
\newtheorem{Definition}{Definição}


\usepackage{fancyhdr}
  \pagestyle{fancy}
  \fancyhf{}
  \lhead{Teoria da Computação}
  \rhead{Aula 7}
  \lfoot{Prof. Rodrigo Ribeiro}
  \rfoot{\thepage}
  \renewcommand{\footrulewidth}{0.4pt}
  \pagestyle{fancy}

\tikzset{
        ->,  % makes the edges directed
        >=stealth', % makes the arrow heads bold
        node distance=2cm,
        every state/.style={thick, fill=gray!10},
        initial text=$\,$
        }
  

\begin{document}

  \title{Aula 7 - Propriedades de fechamento \\para linguagens regulares}
  \author{Rodrigo Ribeiro}

  \maketitle


  \pagestyle{fancy}


  \section*{Objetivos}

  \begin{itemize}
     \item Apresentar o conceito matemático de operação fechada.
     \item Demonstrar que a classe das linguagens regulares é
           fechada sobre união, interseção, complementação,
           concatenação e fecho de Kleene.
     \item Apresentar aplicações de propriedades de fechamento.
  \end{itemize}

  \section{Propriedades de fechamento de LRs}

  \begin{Definition}[Operação fechada]\label{fechada}
    Dizemos que uma função $f : A \times A \times ... \times A \to A$ é fechada
    sobre um conjunto $A$ se $f(a_1,a_2,...,a_n) \in A$, em que $a_i \in A$.
  \end{Definition}

  \begin{Example}
    Alguns exemplos de operações fechadas: A adição e multiplicação são fechadas
    sobre o conjunto dos números naturais. Pois, para qualquer $n,m
    \in\mathbb{N}$ temos que $n + m \in \mathbb{N}$ e $nm \in \mathbb{N}$.
    Porém, a operação de subtração não é fechada sobre $\mathbb{N}$. Pois,
    sempre que $m > n$ temos que $n - m < 0$ e, portanto, não pertence a $\mathbb{N}$.
  \end{Example}

  \begin{Theorem}
    A classe das linguagens regulares é fechada sobre união, interseção e
    complementação.
  \end{Theorem}
  \begin{proof}
    Consequência das construções de produto e complementação apresentadas na
    Aula 4.
  \end{proof}

  \begin{Theorem}
    A classe das linguagens regulares é fechada sobre a concatenação.
  \end{Theorem}
  \begin{proof}
    Sejam $M_1=(E_1,\Sigma,\delta_1,i_1,F_1)$ e
    $M_2=(E_2,\Sigma,\delta_2,i_2,F_2)$
    dois AFDs que aceitam $L(M_1)$ e $L(M_2)$.
    O AFN$\lambda$ $M_3 =(E_1 \cup E_2, \Sigma, \delta_3, \{i_1\}, F_2)$ aceita
    $L(M_1)L(M_2)$, em que $\delta_3$ é definida como:
    \[
      \delta_3 = \delta_1 \cup \delta_2 \cup \{(e,\lambda, i_2)\,|\,e \in F_1\} 
    \]
  \end{proof}

  \begin{Theorem}
    A classe das linguagens regulares é fechada sobre o fecho de Kleene.
  \end{Theorem}
  \begin{proof}
    Seja $M = (E,\Sigma,\delta, i, F)$ um AFD que aceita $L(M)$. O AFN$\lambda$
    $M' = (E', \Sigma, \delta', I', F')$ aceita $L(M)^*$, em que:
    \begin{itemize}
      \item $E' = E \cup\{i'\},\, i' \not\in E$.
      \item $I' = \{i'\}$
      \item $F' = F \cup \{i'\}$
      \item $\delta'$ é definida como:
        \begin{itemize}
        \item $\delta'(i',\lambda) = \{i\}$.
        \item $\delta'(e,a) = \{\delta(e,a)\}$, para todo $e\in E$,
          $a\in\Sigma$.
        \item $\delta'(e,\lambda) = \{i'\}$, para todo $e \in F$.
        \end{itemize}
    \end{itemize}
  \end{proof}

  \subsection{Aplicando as propriedades de fechamento}

  Propriedades de fechamento podem ser utilizadas para: 1) provar que uma
  linguagem é regular; 2) provar que uma linguagem não é regular e 3)
  permitir a construção incremental de AF para linguagens. Apresentaremos
  exemplos de cada uma dessas aplicações.

  \begin{Example}[Aplicação 1]\label{exemplo1}
    Considere a seguinte linguagem
    \[
      L=\{w \in \{0,1\}^*\,|\,w \textit{ possui um
        n}^o\textit{ par de 0's e um único 1}\}
    \]
    Podemos mostrar que essa linguagem é regular usando propriedades de
    fechamento. Para isso, note que $L = L_1 \cap L_2$, em que:
    \[
      \begin{array}{l}
        L_1 = \{w \in \{0,1\}^*\,|\,w\textit{ possui um n}^o\textit{ par de
        0's}\}\\
        L_2 = \{0\}^*\{1\}\{0\}^*
      \end{array}
    \]
    Para mostrar que tanto $L_1$ quanto $L_2$ são regulares, basta construir
    um AF que as aceite.
  \end{Example}
  
  \begin{Example}[Aplicação 2]
    Seja $L=\{0^n1^m2^p\,|\,n = m + p\}$. Provaremos que $L$ não é regular
    usando propriedades de fechamento. A idéia para mostrar que $L$ não é
    regular é aplicarmos propriedades de fechamento de forma a obter uma
    linguagem que já sabemos a priori não ser regular. Ora, se $L$ fosse
    regular, a aplicação de propriedades de fechamento deveria produzir
    somente linguagens regulares. Com isso, obtemos a contradição necessária.
    Segue a demonstração abaixo.

    Suponha que $L$ seja uma linguagem regular. Sabe-se que $\{a\}^*\{b\}^*$ é
    uma linguagem regular. Sendo assim, temos que $L\cap \{a\}^*\{b\}^*$ também
    deve ser uma linguagem regular, já que as LRs são fechadas sobre a
    interseção. Porém, $L\cap \{a\}^*\{b\}^* = \{a^nb^n\,|\,n \geq 0\}$ que
    não é uma LR. Logo, $L$ não pode ser uma linguagem regular.
  \end{Example}
 
  \section{Exercícios}

  \begin{enumerate}
     \item Considerando as linguagens $L,\,L_1$ e $L_2$ apresentadas no
       Exemplo \ref{exemplo1}. Faça o que se pede.
       \begin{enumerate}
         \item Apresente AFDs para $L_1$ e $L_2$.
         \item Usando as propriedades de fechamento, construa um AF para $L$.
         \end{enumerate}
       \item Sejam $L_1$ e $L_2$ duas linguagens. Mostre que sim ou que não:
         \begin{enumerate}
          \item se $L_1$ e $L_2$ não são regulares, $L_1\cup L_2$ não
                é regular.
          \item se $L_1$ e $L_2$ não são regulares, $L_1\cap L_2$ não
                é regular.
          \item se $L_1$ não é regular, $\overline{L_1}$ não é regular.
          \item se $L_1$ é regular e $L_2$ não é regular,
                $L_1\cup L_2$ não é regular.
          \item se $L_1$ é regular, $L_2$ não é regular e
                $L_1\cap L_2$ é regular, $L_1\cup L_2$ não é regular.
          \item se $L_1$ é regular, $L_2$ não é regular e
                $L_1\cap L_2$ não é regular, $L_1\cup L_2$ não é regular.
       \end{enumerate}
     \item Prove que os seguintes conjuntos não são regulares usando
       propriedades de fechamento.
       \begin{enumerate}
          \item $\{0,1\}^* - \{0^n1^n\,|\,n \geq 0\}$.
          \item $\{0^n1^m \,|\, n < m\} \cup \{0^n1^m\,|\, m < n\}$
       \end{enumerate}
  \end{enumerate}
\end{document}
