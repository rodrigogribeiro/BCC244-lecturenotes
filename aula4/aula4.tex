\documentclass[a4paper]{article}

\usepackage[portuguese]{babel}
\usepackage[utf8]{inputenc}
\usepackage{graphicx,hyperref}
\usepackage{float}
\usepackage{listings}
\usepackage{proof,tikz}
\usepackage{amssymb,amsthm,stmaryrd}


\usepackage[edges]{forest}
\usetikzlibrary{automata, positioning, arrows}


\newtheorem{Lemma}{Lema}
\newtheorem{Theorem}{Teorema}
\theoremstyle{definition}
\newtheorem{Example}{Exemplo}
\newtheorem{Definition}{Definição}


\usepackage{fancyhdr}
  \pagestyle{fancy}
  \fancyhf{}
  \lhead{Teoria da Computação}
  \rhead{Aula 4}
  \lfoot{Prof. Rodrigo Ribeiro}
  \rfoot{\thepage}
  \renewcommand{\footrulewidth}{0.4pt}
  \pagestyle{fancy}

\tikzset{
        ->,  % makes the edges directed
        >=stealth', % makes the arrow heads bold
        node distance=3cm,
        every state/.style={thick, fill=gray!10},
        initial text=$\,$
        }
  

\begin{document}

  \title{Aula 4 - Produto, complementação de AFs e\\ introdução aos autômatos
    finitos não determínisticos}
  \author{Rodrigo Ribeiro}

  \maketitle


  \pagestyle{fancy}


  \section*{Objetivos}

  \begin{itemize}
     \item Apresentar as construções de produto (união e interseção) e
       complementação de AFDs. 
     \item Apresentar a correção da construção de produto.
     \item Apresentar o conceito de autômato finito não-determínistico (AFN).
  \end{itemize}

  \section{A construção do produto}

  \begin{Definition}[Produto de AFDs]
    Sejam $M_1 = (E_1,\Sigma,\delta_1,i_1,F_1)$ e $M_2 =
    (E_2,\Sigma,\delta_2,i_2,F_2)$ dois AFDs quaisquer. O AFD
    $M_3=(E_3,\Sigma,\delta_3,i_3,F_3)$ é o produto de $M_1$ e
    $M_2$, em que:
    \begin{itemize}
       \item $E_3 = E_1 \times E_2$
       \item $\delta_3((e_1,e_2),a) = (\delta_1(e_1,a),\delta_2(e_2,a))$.
       \item Para $F_3$, temos duas possibilidades:
         \begin{itemize}
           \item Para interseção: $F_3 = F_1 \times F_2$.
           \item Para união: $F_3 = (F_1 \times E_2) \cup (E_1 \times F_2)$.
         \end{itemize}
    \end{itemize}
  \end{Definition}

  \begin{Example}
    Considere os seguintes AFDs, um que reconhece $L_1 = \{0\}\{0,1\}^*$ e o
    segundo $L_2 = \{0,1\}^*\{1\}$:
    \begin{figure}[H]
      \begin{minipage}{0.45\textwidth}
        \begin{tikzpicture}[node distance=2cm]
          \node[state, initial](s0){$A$} ;
          \node[state, accepting, right of=s0](s1){$B$} ;
          \node[state, below of=s0] (err) {$e$} ;
          \draw (s0) edge[above] node{$0$} (s1)
                (s0) edge[left] node{$1$} (err)
                (err) edge[loop right] node{$0,1$} (err)
                (s1) edge[loop above] node{$0,1$} (err) ;
        \end{tikzpicture}
      \end{minipage}\hfill
      \begin{minipage}{0.45\textwidth}
        \begin{tikzpicture}[node distance=2cm]
          \node[state,initial](s0){$C$} ;
          \node[state,accepting, right of=s0](s1){$D$} ;
          \draw (s0) edge[loop above] node{$0$} (s0)
                (s0) edge[above, bend left] node{$1$} (s1)
                (s1) edge[loop above] node{$1$} (s1)
                (s1) edge[below, bend left] node{$0$} (s0) ;
        \end{tikzpicture}
      \end{minipage}
      \centering
    \end{figure}
    Aplicando a construção de produto, podemos criar AFDs que $L_1 \cap L_2$
    e $L_1\cup L_2$ que são apresentados a seguir. Vamos usar as seguintes
    abreviações para os conjuntos de estados: $1$ para o conjunto $\{A,C\}$,
    $2$ para $\{B,C\}$, $3$ para $\{e,D\}$, $4$ para $\{e,C\}$ e $5$ para
    $\{B,D\}$. Primeiro, apresentamos o AFD para $L_1 \cap L_2$.
    \begin{figure}[H]
      \begin{tikzpicture}[node distance=2cm]
        \node[state,initial](s0){$1$} ;
        \node[state, above right of=s0](s1){$2$} ;
        \node[state, below right of=s0](s2){$3$} ;
        \node[state, right of=s2](s3){$4$} ;
        \node[state, accepting, right of=s1](s4){$5$} ;
        \draw (s0) edge[above left] node{$0$} (s1)
              (s0) edge[below left] node{$1$} (s2)
              (s1) edge[loop above] node{$0$} (s1)
              (s1) edge[above, bend left] node{$1$} (s4)
              (s2) edge[above, bend left] node{$0$} (s3)
              (s2) edge[loop above] node{$1$} (s2)
              (s3) edge[loop above] node{$0$} (s3)
              (s3) edge[below, bend left] node{$1$} (s2)
              (s4) edge[below, bend left] node{$0$} (s1)
              (s4) edge[loop above] node{$1$} (s4) ;
      \end{tikzpicture}
      \centering
    \end{figure}
    A seguir apresentamos o AFD para $L_1 \cup L_2$.
    \begin{figure}[H]
      \begin{tikzpicture}[node distance=2cm]
        \node[state,initial](s0){$1$} ;
        \node[state, accepting, above right of=s0](s1){$2$} ;
        \node[state, accepting, below right of=s0](s2){$3$} ;
        \node[state, right of=s2](s3){$4$} ;
        \node[state, accepting, right of=s1](s4){$5$} ;
        \draw (s0) edge[above left] node{$0$} (s1)
              (s0) edge[below left] node{$1$} (s2)
              (s1) edge[loop above] node{$0$} (s1)
              (s1) edge[above, bend left] node{$1$} (s4)
              (s2) edge[above, bend left] node{$0$} (s3)
              (s2) edge[loop above] node{$1$} (s2)
              (s3) edge[loop above] node{$0$} (s3)
              (s3) edge[below, bend left] node{$1$} (s2)
              (s4) edge[below, bend left] node{$0$} (s1)
              (s4) edge[loop above] node{$1$} (s4) ;
      \end{tikzpicture}
      \centering
    \end{figure}
  \end{Example}

  \subsection{Correção da construção de produto}

  Apresentaremos a correção para a construção da interseção. O mesmo raciocínio
  aplica-se para a união.

  \begin{Lemma}\label{lemma1}
    Para todo $e_1 \in E_1, e_2\in E_2$ e $w \in \Sigma^*$, temos que
    $\widehat{\delta}_3((e_1,e_2), w) = (\widehat{\delta}_1(e_1,w),\widehat{\delta}_2(e_2,w))$.
  \end{Lemma}
  \begin{proof} Indução sobre $w$.
    \begin{itemize}
       \item Caso base ($w=\lambda$). Temos:
         \[
           \widehat{\delta}_3((e_1,e_2), \lambda) = (e_1,e_2) = (\widehat{\delta}_1(e_1,\lambda),\widehat{\delta}(e_2,\lambda))
         \]
       \item Passo indutivo. Suponha $a\in\Sigma$ arbitrario e que para todo $e_1,e_2$;
         $\widehat{\delta}_3((e_1,e_2), w) =
         (\widehat{\delta}_1(e_1,w),\widehat{\delta}_2(e_2,w))$.
         Temos:
         \[
           \begin{array}{lcl}
             \widehat{\delta}_3((e_1,e_2), aw) & = & \{\textit{def. de
                                                     }\widehat{\delta}.\}\\
             \widehat{\delta}_3(\delta_3((e_1,e_2),a),w) & = &
                                                               \{\textit{def. de
                                                               }\delta_3\}\\
             \widehat{\delta}_3((\delta_1(e_1,a),\delta_2(e_2,a)),w) & = &
                                                                           \{\textit{I.H.}\}\\
             (\widehat{\delta}_1(\delta_1(e_1,a),w),\widehat{\delta}_2(\delta_2(e_2,a),w))
                                               & = & \{\textit{def. de
                                                     }\widehat{\delta}\} \\
             (\widehat{\delta}_1(e_1,aw),\widehat{\delta}(e_2,aw))
           \end{array}
         \]
    \end{itemize}
  \end{proof}

  \begin{Theorem}{Correção da construção de produto para interseção}
    Sejam $M_1$ e $M_2$ dois AFDs quaisquer e $M_3$ o AFD construído para
    a interseção usando $M_1$ e $M_2$. Temos que $L(M_3) = L(M_1) \cap L(M_2)$.
  \end{Theorem}
  \begin{proof}
    Suponha $w \in \Sigma^*$. Temos:
    \[
      \begin{array}{lcl}
        w \in L(M_3) & \leftrightarrow & \{\textit{def. de }L(M_3)\}\\
        \widehat{\delta}_3(i_3,w) \in F_3 & \leftrightarrow & \{\textit{def. de }i_3,F_3\}\\
        \widehat{\delta}_3((i_1,i_2),w) \in F_1 \times F_2 & \leftrightarrow &
                                                                               \{\textit{Lema
                                                                               }
                                                                               \ref{lemma1}\}
        \\
        (\widehat{\delta}_1(e_1,w),\widehat{\delta}_2(e_2,w)) \in F_1\times F_2
                     & \leftrightarrow & \{\textit{Prod. Cart.}\} \\
        \widehat{\delta}_1(e_1,w) \in F_1 \land \widehat{\delta}_2(e_2,w)\in F_2
                     & \leftrightarrow & \{\textit{def. de }L(M)\} \\
        w \in L(M_1) \land w \in L(M_2) & \leftrightarrow & \{\textit{def.
                                                            de }\cap\}\\
        w \in L(M_1) \cap L(M_2)
      \end{array}
    \]
    Como $w$ é arbitrário, temos que $L(M_3) = L(M_1)\cap L(M_2)$.
  \end{proof}

  \section{A construção da complementação}

  \begin{Definition}
    Seja $M=(E,\Sigma,\delta,i,F)$ um AFD que reconhece uma linguagem $L$.
    O AFD $\overline{M} = (E,\Sigma,\delta,i,E - F)$ aceita $\overline{L}$.
  \end{Definition}

  \begin{Example}\label{afd00}
    Considere a seguinte linguagem sobre $\Sigma = \{0,1\}$:
    \[
       L = \{0,1\}^*\{00\}\{0,1\}^*
     \]
     O seguinte AFD aceita as palavras pertencentes a esta
     linguagem.

     \begin{figure}[ht]
       \begin{tikzpicture}
         \node[state, initial]               (s0){$A$} ;
         \node[state,            right of=s0](s1){$B$} ;
         \node[state, accepting, right of=s1](s2){$C$} ;

         \draw (s0) edge[loop above]       node{$1$}   (s0)
               (s0) edge[bend left, above] node{$0$}   (s1)
               (s1) edge[above]            node{$0$}   (s2)
               (s1) edge[bend left, below] node{$1$}   (s0)
               (s2) edge[loop above]       node{$0,1$} (s2) ;
       \end{tikzpicture}
       \centering      
     \end{figure}
     Podemos usar a construção de complementação para construir um
     AFD para $\overline{L}$. Para isso, basta marcar como finais
     todos os estados não finais e como não finais os antigos estados finais.
     Abaixo apresentamos o resultado para o AFD acima apresentado.
     \begin{figure}[ht]
       \begin{tikzpicture}
         \node[state, initial, accepting]               (s0){$A$} ;
         \node[state, accepting, right of=s0](s1){$B$} ;
         \node[state, right of=s1](s2){$C$} ;

         \draw (s0) edge[loop above]       node{$1$}   (s0)
               (s0) edge[bend left, above] node{$0$}   (s1)
               (s1) edge[above]            node{$0$}   (s2)
               (s1) edge[bend left, below] node{$1$}   (s0)
               (s2) edge[loop above]       node{$0,1$} (s2) ;
       \end{tikzpicture}
       \centering      
     \end{figure}   
  \end{Example}  


  \section{Introdução aos autômatos finitos não determínisticos}

  \begin{Example}[Um exemplo informal]
    Considere a tarefa de construir um AFD para a seguinte linguagem
    $\{0,1\}^*\{00\}$.
    \begin{figure}[H]
      \begin{tikzpicture}
        \node[state,initial](s0){$A$} ;
        \node[state, right of=s0](s1){$B$};
        \node[state, accepting, right of=s1](s2){$C$};
        \draw (s0) edge[loop above] node{$1$}(s0)
              (s0) edge[above, bend left] node{$0$}(s1)
              (s1) edge[above, bend left] node{$0$}(s2)
              (s1) edge[above, bend left] node{$1$}(s0)
              (s2) edge[below, bend left] node{$1$}(s0)
              (s2) edge[loop above] node{$0$}(s2);
      \end{tikzpicture}
      \centering
    \end{figure}
    Apesar de correto, o AFD acima é ``poluído'' por transições necessárias
    para garantir que a palavra aceita realmente termina em 00. Uma alternativa,
    mais compacta, e de mais fácil entendimento seria o seguinte autômato:
    \begin{figure}[H]
      \begin{tikzpicture}[node distance=2cm]
        \node[state,initial](s0){$A$} ;
        \node[state, right of=s0](s1){$B$};
        \node[state, accepting, right of=s1](s2){$C$};
        \draw (s0) edge[loop above] node{$0,1$}(s0)
              (s0) edge[above] node{$0$}(s1)
              (s1) edge[above] node{$0$}(s2) ;
      \end{tikzpicture}
      \centering
    \end{figure}
    Após uma rápida análise, podemos perceber que o autômato anterior realmente
    aceita apenas palavras da linguagem $\{0,1\}^*\{00\}$. Porém, o autômato
    anterior não é um AFD, pois a partir do estado $A$ processando $0$ na
    palavra de entrada, temos transições tanto para o próprio estado $A$ quanto
    para o estado $B$. O autômato acima apresentado é um exemplo de um autômato
    finito não determínistico.
  \end{Example}

  \begin{Definition}[Autômato finito não determinístico - AFN]
    Um AFN é uma quíntupla $M=(E,\Sigma,\delta,I,F)$, em que:
    \begin{itemize}
       \item $E$: conjunto finito de estados.
       \item $\Sigma$: alfabeto.
       \item $\delta : E \times \Sigma \to \mathcal{P}(E)$: função total de transição.
       \item $I \subseteq E$ : conjunto de estados iniciais.
       \item $F \subseteq E$ : conjunto de estados finais.
    \end{itemize}
    Observe que a função de transição produz, como resultado, um conjunto de
    estados e, por isso, seu contra-domínio é $\mathcal{P}(E)$, o conjunto de
    todos subconjuntos de estados.
  \end{Definition}

  \begin{Example}
    Vamos considerar o AFN apresentado anteriormente sob um ponto de vista de
    sua definição matemática. Primeiramente, temos que o seu conjunto de estados
    e seu alfabeto são, respectivamente, $E=\{A,B,C\}$ e $\Sigma-\{0,1\}$. O
    conjunto de estados iniciais é $I=\{A\}$ e o de estados finais $F=\{C\}$.
    Finalmente, a função de transição para esse AFN é dada pela seguinte tabela.
    \begin{figure}[H]
      \begin{tabular}{|c|c|c|}
        \hline
        $\delta$ & $0$      & $1$ \\ \hline
        $A$      &$\{A,B\}$ & $\{A\}$ \\
        $B$      & $\{C\}$  & $\emptyset$ \\
        $C$      & $\emptyset$ & $\emptyset$\\ \hline
      \end{tabular}
      \centering
    \end{figure}
  \end{Example}

  \subsection{Linguagem aceita por um AFN}

  Intuitivamente, dizemos que uma palavra $w\in\Sigma^*$ é aceita por um AFN $M$
  se \textbf{existe} uma computação que termina em algum estado final de $M$
  quando essa é iniciada em um estado inicial para a palavra $w$. O objetivo
  dessa seção é apresentar formalmente essa noção.

  \begin{Definition}[Função de transição estendida]
    Seja $M=(E,\Sigma,\delta,I,F)$ um AFN qualquer. A função de transição
    estendida $\widehat{\delta}^* : \mathcal{P}(E) \times \Sigma^*\to\mathcal{P}(E)$ é
    definida como:
    \begin{itemize}
       \item $\widehat{\delta}^*(\emptyset,w) = \emptyset$, para todo
         $w\in\Sigma^*$.
       \item $\widehat{\delta}^*(A,\lambda) = A$, para todo $A\subseteq E$.
       \item $\widehat{\delta}^*(A,ay) = \widehat{\delta}^*(\bigcup_{e \in
           A}\delta(e,a), y)$, para todo $a\in \Sigma, y \in \Sigma^*$ e $A
         \subseteq E$.
    \end{itemize}
  \end{Definition}

  \begin{Definition}[Linguagem aceita por um AFN]
    Seja $M=(E,\Sigma,\delta,I,F)$ um AFN qualquer. A linguagem aceita por $M$,
    $L(M)$, é definida como:
    \[
      L(M) =\{w\in\Sigma^*\,|\,\widehat{\delta}^*(I,w)\cap F \neq \emptyset\}
    \]
    Isto é, o conjunto de palavras $w$ cujo processamento termina em um conjunto
    de estados com pelo menos um estado final.
  \end{Definition}
  
  \section{Exercícios}

  \begin{enumerate}
    \item Apresente AFDs para as seguintes linguagens sobre $\Sigma = \{0,1\}$.
      Se possível, tente expressá-las como linguagens ``menores'' e use as
      construções apresentadas para obter o AFD final.
      \begin{enumerate}
        \item Palavras cujo tamanho é menor que 3.
        \item Palavras cujo tamanho é maior que 3.
        \item Palavras com no máximo 3 ocorrências de 1's.
        \item Palavras que contém um ou dois 1's e cujo tamanhé múltiplo de 3.
      \end{enumerate}
    \item Sejam $M_1$ e $M_2$ dois AFDs quaisquer. Mostre como construir um AFD
      $M_3$ tal que $L(M_3) = L(M_1) - L(M_2)$.
    \item Construa AFNs para as seguintes linguagens. Considere o alfabeto $\Sigma=\{a,b,c\}$.
      \begin{enumerate}
        \item Palavras em que o último símbolo seja igual ao primeiro.
        \item Palavras em que o útlimo símbolo seja diferente do primeiro.
        \item Palavras terminadas em abab.
        \item Palavras em que o último símbolo não tenha ocorrido antes.
        \end{enumerate}
    \item Mostre como a partir de um AFN qualquer é possível construir um AFN
      equivalente contendo apenas um estado inicial.
  \end{enumerate}
\end{document}
