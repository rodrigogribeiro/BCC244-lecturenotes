\documentclass[a4paper]{article}

\usepackage[portuguese]{babel}
\usepackage[utf8]{inputenc}
\usepackage{graphicx,hyperref}
\usepackage{float}
\usepackage{listings}
\usepackage{proof,tikz}
\usepackage{amssymb,amsthm,stmaryrd, amsmath}


\usepackage[edges]{forest}
\usetikzlibrary{automata, positioning, arrows}


\newtheorem{Lemma}{Lema}
\newtheorem{Theorem}{Teorema}
\theoremstyle{definition}
\newtheorem{Example}{Exemplo}
\newtheorem{Definition}{Definição}


\usepackage{fancyhdr}
  \pagestyle{fancy}
  \fancyhf{}
  \lhead{Teoria da Computação}
  \rhead{Aula 15}
  \lfoot{Prof. Rodrigo Ribeiro}
  \rfoot{\thepage}
  \renewcommand{\footrulewidth}{0.4pt}
  \pagestyle{fancy}

\tikzset{
        -,  % makes the edges directed
        >=stealth', % makes the arrow heads bold
        node distance=3cm,
        every state/.style={thick, fill=gray!10},
        initial text=$\,$
        }
  

\begin{document}

  \title{Aula 15 - Equivalência entre autômatos de pilha e gramáticas livres de contexto}
  \author{Rodrigo Ribeiro}

  \maketitle


  \pagestyle{fancy}


  \section*{Objetivos}

  \begin{itemize}
     \item Apresentar a construção de um APN a partir de uma gramática.
     \item Apresentar a construção de uma gramática a partir de um APN.     
  \end{itemize}

  \section{Equivalência entre GLCs e APs}

  \subsection{Construindo um AP a partir de uma GLC}

  \begin{Theorem}
    Para qualquer GLC $G$, existe um AP $M$ que aceita $L(G)$.
  \end{Theorem}
  \begin{proof}
    Seja $G' = (V,\Sigma,R,P)$ a gramática equivalente a $G$ na FNG.
    Um APN que aceita $L(G')$ é: $M=(\{i,f\},\Sigma, V, \delta,\{i\},\{f\})$,
    em que $\delta$ é tal que:
    \begin{itemize}
      \item $\delta(i,\lambda,\lambda) = \{[f, P]\}$
      \item Se $P \to \lambda \in R,\,\delta(f,\lambda,P) = \{[f,\lambda]\}$
      \item $\delta(f, a, X) = \{[f,y]\,|\,y \in V^* \land X \to ay \in R\}$.
    \end{itemize}
  \end{proof}

  \begin{Example}
    Para ilustrar a construção de um AP a partir de uma gramática, vamos
    considerar a seguinte gramática na FNG:
    \[
      \begin{array}{l}
        P \to 0PUP\,|\,1PZP\,|\,\lambda\\
        U \to 1 \\
        Z \to 0 \\
      \end{array}
    \]
    O AP que aceita $L(G)$ é apresentado abaixo:
    \begin{figure}[H]
      \begin{tikzpicture}
        \node[state, initial](i){$i$};
        \node[state, accepting, right of=i](f){$f$};
        \draw (i) edge[above] node{$\lambda,\lambda/P$}(f)
        (f) edge[loop right]
        node{$
          \begin{array}{l}
            \lambda, P / \lambda \\
            0, P / PUP \\
            1, P / PZP \\
            1, U / \lambda \\
            0, Z / \lambda \\
          \end{array}
          $}(f) ;
      \end{tikzpicture}
      \centering
    \end{figure}
  \end{Example}

  \subsection{Construindo uma GLC a partir de um AP}

  Seja $M = (E,\Sigma,\Gamma,I,F)$ um APN qualquer, $e,e' \in E$ e
  $X \in \Gamma \cup \{\lambda\}$. Denote por $C(e,X,e')$ o conjunto
  de palavras $w \in \Sigma^*$ em que $M$ começando em $e$, com pilha
  contendo $X$ termina em $e'$ com pilha vazia, isto é:
  \[
    C(e,X,e') =\{w \in \Sigma^*\,|\,[e,w,X]\vdash^*[e',\lambda,\lambda]\}
  \]
  Note que $L(M) = \bigcup_{(i,f) \in I \times F} C(i,\lambda,f)$.

  Suponha que seja possível construir uma GLC que gere todas as palavras em
  $C(e,X,e')$ usando, como símbolo de partida, $[e,X,e']$. Dessa forma,
  o problema de obter uma GLC equivalente a um AP consiste em gerar regras
  para a variável $[i,\lambda,f]$, para cada $i \in I, f\in F$. O próximo
  teorema dá os detalhes dessa construção.

  
  \begin{Theorem}\label{thmglcap}
    Para cada AP $M$, existe uma GLC que gera $L(M)$.
  \end{Theorem}
  \begin{proof}
    Seja $M = (E,\Sigma,\Gamma,I,F)$. A gramática $G = (V,\Sigma,R,P)$ tal que
    $L(G) = L(M)$ é:
    \begin{itemize}
      \item $V = \{P\} \cup E \times (\Gamma \cup \{\lambda\}) \times E$.
      \item O conjunto de regras $R$ é formado por:
        \begin{itemize}
           \item Para cada $i \in I$ e $f \in F$, $P \to [i,\lambda,f]$;
           \item Para cada $e\in E$, $[e,\lambda,e] \to \lambda$;
           \item Para cada transição $[e',z] \in \delta(e, a, A)$, em que $a\in
             \Sigma\cup\{\lambda\}$, $z\in \Gamma^*$ e $A \in \Gamma \cup
             \{\lambda\}$, temos as seguintes regras:
             \begin{itemize}
               \item Se $z = \lambda$, $[e,A,d] \to a[e',\lambda,d]$ para cada
                 $d \in E$.
               \item Se $z\neq \lambda$, $[e,A,d] \to a[e', B_1,
                 d_1]...[d_{n-1},B_n,d_n]$, em que $z = B_1...B_n$ e
                 $(d_1,...,d_n) \in E^n$.
               \item Se $A = \lambda$, adicionamente temos as seguintes regras:
                 \begin{itemize}
                   \item Se $z = \lambda$, $[e,C,d] \to a[e',C,d]$, para cada $C \in \Gamma$ e
                     cada $d \in E$.
                   \item Se $z \neq \lambda$, $[e,C,d] \to a[e',B_1,d_1]...[d_{n
                       - 1},B_n,d_n][d_n,C,d_{n + 1}]$ para cada $C \in \Gamma$ e
                    cada $(d_1,...,d_{n+1}) \in E^{n+1}$.
                 \end{itemize}
             \end{itemize}
        \end{itemize}
    \end{itemize}
  \end{proof}

  A construção de um GLC usando o teorema anterior irá produzir muitas variáveis
  inúteis. Podemos minimizar esse efeito seguindo as seguintes recomendações:
  \begin{enumerate}
    \item Começando com as regras de $P$, gerar apenas variáveis alcançáveis por
      $P$.
    \item Se não há caminho de $e$ para $e'$ no AP $M$, variáveis $[e,X,e']$ são
      inúteis.
    \item Se toda computação de $e$ para $e'$ só aumenta a pilha, então
      variáveis $[e,X,e']$ são inúteis.
  \end{enumerate}
  
  \begin{Example}
    Para ilustrar a construção de uma gramática equivalente a um APN, considere o seguinte AP:
    \begin{figure}[H]
      \begin{tikzpicture}
        \node[state,initial,accepting](i){$i$};
        \node[state,accepting, right of=i](f){$f$};
        \draw (i)edge[loop above]node{$a,\lambda/B$}(i)
        (i)edge[above]node{$b,B/C$}(f)
        (f)edge[loop above]node{$\begin{array}{l}b, B/C\\ c, C/\lambda \end{array}$}(f);
      \end{tikzpicture}
      \centering
    \end{figure}
    Seguinte a construção especificada no teorema~\ref{thmglcap}, obtemos as
    seguintes regras:
    \[
      \begin{array}{l}
        P \to \lbrack 0,\lambda,0 \rbrack \,|\,\lbrack 0,\lambda,1 \rbrack\\
        \lbrack 0,\lambda,0 \rbrack \to \lambda \\
        \lbrack 1,\lambda,1 \rbrack \to \lambda \\
      \end{array}
    \]
    A partir da transição, $\delta(0,a,\lambda) = [0,B]$, temos:
    \[
      \begin{array}{l}
        \lbrack 0,\lambda,0 \rbrack \to a[0,B,0]\\
        \lbrack 0,\lambda,1 \rbrack \to a[0,B,1]\\
        \lbrack 0,B,1 \rbrack \to a\lbrack 0,B,0\rbrack \lbrack
        0,B,1\rbrack\,|\,a\lbrack 0,B,1\rbrack \lbrack 1,B,1\rbrack\\
      \end{array}
    \]
    A partir da transição $\delta(0,b,B) = [1,C]$, obtemos:
    \[
      \begin{array}{l}
        \lbrack 0,B,1\rbrack \to b\lbrack 1,\lambda,1\rbrack\\
      \end{array}
    \]
    Finalmente, a transição $\delta(1,b,B) = [1,C]$ gera as seguintes regras:
    \[
      \begin{array}{l}
        \lbrack 1,B,1\rbrack \to b\lbrack 1,\lambda,1\rbrack \\
        \lbrack 1,C,1\rbrack  \to c\lbrack 1,\lambda,1\rbrack\\
      \end{array}
    \]
  \end{Example}
  
  \section{Exercícios} 

  \begin{enumerate}
     \item Construa um APN equivalente a gramática a seguir.
       \[
         \begin{array}{l}
           P \to BPA \,|\,A \\
           A \to aA \,|\, \lambda\\
           B \to Bba \,|\,\lambda\\
         \end{array}
       \]
     \item Construa a gramática equivalente ao seguinte AP:
    \begin{figure}[H]
      \begin{tikzpicture}
        \node[state,initial,accepting](i){$i$};
        \node[state,accepting, right of=i](f){$f$};
        \draw (i)edge[loop above]node{$0,\lambda/X$}(i)
        (i)edge[above]node{$1,X/\lambda$}(f)
        (f)edge[loop above]node{$1,X/\lambda$}(f);
      \end{tikzpicture}
      \centering
    \end{figure}
  \end{enumerate}
\end{document}
