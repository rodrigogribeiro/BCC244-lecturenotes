\documentclass[a4paper]{article}

\usepackage[portuguese]{babel}
\usepackage[utf8]{inputenc}
\usepackage{graphicx,hyperref}
\usepackage{float}
\usepackage{listings}
\usepackage{proof,tikz}
\usepackage{amssymb,amsthm,stmaryrd}


\usepackage[edges]{forest}
\usetikzlibrary{automata, positioning, arrows}


\newtheorem{Lemma}{Lema}
\newtheorem{Theorem}{Teorema}
\theoremstyle{definition}
\newtheorem{Example}{Exemplo}
\newtheorem{Definition}{Definição}


\usepackage{fancyhdr}
  \pagestyle{fancy}
  \fancyhf{}
  \lhead{Teoria da Computação}
  \rhead{Aula 9}
  \lfoot{Prof. Rodrigo Ribeiro}
  \rfoot{\thepage}
  \renewcommand{\footrulewidth}{0.4pt}
  \pagestyle{fancy}

\tikzset{
        ->,  % makes the edges directed
        >=stealth', % makes the arrow heads bold
        node distance=2cm,
        every state/.style={thick, fill=gray!10},
        initial text=$\,$
        }
  

\begin{document}

  \title{Aula 9 - Gramáticas regulares e problemas de decisão para LRs.}
  \author{Rodrigo Ribeiro}

  \maketitle


  \pagestyle{fancy}


  \section*{Objetivos}

  \begin{itemize}
     \item Apresentar os conceito de gramática, derivação e linguagem gerada por
       uma gramática.
     \item Apresentar a definição de gramática regular e sua equivalência com
       autômatos finitos.
     \item Apresentar alguns problemas de decisão para LRs.
  \end{itemize}

  \section{Gramáticas}

  \begin{Definition}[Gramática]
    Uma gramática $G=(V,\Sigma,R,P)$ é uma quádrupla em que:
    \begin{itemize}
       \item $V$: conjunto finito de variáveis.
       \item $\Sigma$: alfabeto
       \item $R\subseteq (V \cup \Sigma)^+ \times (V\cup \Sigma^*)$: conjunto de
         regras.
       \item $P\in V$: variável de partida.
    \end{itemize}
  \end{Definition}

  \begin{Example}
    Assim como autômatos são representados, informalmente, usando grafos
    dirigidos, gramáticas são representadas informalmente por uma sequência
    de suas regras. A definição matemática de uma gramática mostra que uma
    regra é apenas um par: $(x,y)$. Porém, para melhor visualização, regras
    são representadas usando a seguinte notação: $x \to y$.

    Abaixo apresentamos um exemplo de gramática.
    \[
      \begin{array}{lcl}
        A & \to & 0A1 \,|\, \lambda
      \end{array}
    \]

    A gramática acima é formada por duas regras: 1) $A \to 0A1$ e 2 $A \to
    \lambda$. Normalmente, omitidos o lado esquerdo de regras consecutivas
    que compartilham o mesmo lado esquerdo. Para isso, simplesmente usamos
    o símbolo $|$.

    A gramática acima é formalmente representada por $(\{A\},\{0,1\},R,A)$, em
    que $R =\{(A,0A1),(A,\lambda)\}$.

    A seguir, apresentamos o mecanismo de produção de palavras
    em uma gramática.
  \end{Example}

  \begin{Definition}[Derivação]
    Seja $G = (V,\Sigma,R,P)$ uma gramática qualquer.
    Dizemos que $x \Rightarrow
    y$ em $G$, em que $x,y \in (V\cup \Sigma)^*$, se  há uma regra $u \to v \in
    R$ tal que $u$ ocorre em $x$ e $y$ é o resultado de substituir uma
    ocorrência de $u$ em $x$ por $v$. A relação $\Rightarrow^n$ é definida 
    recursivamente como:
    \begin{itemize}
      \item $x \Rightarrow^0 x$ para todo $x \in (V \cup \Sigma)^*$;
      \item $w \Rightarrow^n x u y$ e $u \to v \in R$ então $w \Rightarrow^{n +
          1}x v y$; para todo $w,x,y \in (V\cup \Sigma)^*$, $n\geq 0$.
    \end{itemize}
    Finalmente, dizemos que $x$ deriva $y$, $x\Rightarrow^*y$, se existe 
    $n\geq 0$ tal que $x\Rightarrow^n y$.
  \end{Definition}

  \begin{Example}
    Vamos considerar a seguinte gramática:
    \[
      \begin{array}{lcl}
        A & \to & 0A1 \,|\, \lambda
      \end{array}
    \]
    Abaixo, mostramos a derivação de $0011$.
    \[
      \begin{array}{lcl}
        A   & \Rightarrow & \{A \to 0A1\}\\
        0A1 & \Rightarrow & \{A \to 0A1\}\\
        00A11 & \Rightarrow & \{A \to \lambda\} \\
        0011
      \end{array}
    \]
  \end{Example}

  \begin{Definition}[Linguagem gerada por uma gramática]
    Seja $G = (V,\Sigma,R,P)$ uma gramática qualquer. A linguagem gerada por
    $G$, $L(G)$, é definida como:
    \[
      L(G) = \{w \in \Sigma^* \,|\, P \Rightarrow^* w\}
    \]
  \end{Definition}

  \section{Gramáticas regulares}

  \begin{Definition}[Gramática regular]\label{gramreg}
    Seja $G = (V,\Sigma,R,P)$ uma gramática. Dizemos que $G$ é uma gramática
    regular (GR), se todas as suas regras possuem uma das formas:
    \begin{itemize}
      \item $X \to a$
      \item $X \to aZ$
      \item $X \to \lambda$
    \end{itemize}
    Para $X,Z \in V$, $a \in \Sigma$.  
  \end{Definition}

  \begin{Example}
    A seguinte gramática
    \[
      \begin{array}{lcl}
        A & \to & 0A1 \,|\, \lambda
      \end{array}
    \]
    não é uma GR. Pois a regra $A \to 0A1$ não está de acordo com a Definição
    \ref{gramreg}. Por sua vez, a seguinte gramática é uma GR:
    \[
      \begin{array}{lcl}
        A & \to & 0A \,|\,0\,|\,1B \\
        B & \to & 1B \,|\,\lambda
      \end{array}
    \]
    A próxima subseção descreverá a equivalência entre GR e AFs.
  \end{Example}

  \subsection{Construindo um AF a partir de uma GR}

  \begin{Theorem}\label{teorema1}
    Toda gramática regular gera uma linguagem regular.
  \end{Theorem}
  \begin{proof}
    Seja $G = (V,\Sigma,R,P)$ uma GR. O AFN $M = (E, \Sigma, \delta, \{P\}, F)$
    é tal que $L(G) = L(M)$, em que:
    \begin{itemize}
      \item $E = \left\{ \begin{array}{ll}
                         V \cup \{Z\} & \textit{se R possui alguma regra }X \to
                                        a\\
                         V            & \textit{caso contrário}
                       \end{array}\right.$
      \item $\delta(X,a) = \left\{
          \begin{array}{ll}
            \{Y\} & \textit{se R possuir a regra }X \to aY\\
            \{Z\} & \textit{se R possuir a regra }X \to a\\
          \end{array}  
        \right.$
      \item $F = \left\{
          \begin{array}{ll}
            \{X\,|\,X \to \lambda\} \cup \{Z\} & \textit{se existe regra }X \to a\\
            \{X\,|\,X \to \lambda\} & \textit{caso contrário.}
          \end{array}
                \right.$
    \end{itemize}
    Pode-se mostrar que $P \Rightarrow^* w$ se e somente se
    $\widehat{\delta}(P,w)\cap F \neq \emptyset$ por indução sobre $w$.          
  \end{proof}

  \begin{Example}
    Considere a seguinte gramática regular:
    \[
      \begin{array}{lcl}
        A & \to & 0A \,|\, 1B \,|\, \lambda \\
        B & \to & 1B \,|\, 1
      \end{array}
    \]
    A partir dessa GR, podemos construir o seguinte AF:
    \begin{figure}[H]
      \begin{tikzpicture}
        \node[state,initial,accepting](s0){$A$} ;
        \node[state, right of=s0](s1){$B$} ;
        \node[state,accepting, right of=s1](z){$Z$};
        \draw (s0) edge[loop above]node{$0$}(s0)
              (s0) edge[above]node{$1$}(s1)
              (s1) edge[loop above]node{$1$}(s1)
              (s1) edge[above]node{$1$}(z);
      \end{tikzpicture}
      \centering
    \end{figure}
    
  \end{Example}
  
  \subsection{Construindo uma GR a partir de um AF}

  \begin{Theorem}\label{teorema2}
    Toda linguagem regular é gerada por uma gramática regular.
  \end{Theorem}
  \begin{proof}
    Seja $M = (E,\Sigma,\delta,i,F)$ um AFD. A GR $G = (E,\Sigma, R, i)$ é tal
    que $L(G) = L(M)$, em que:
    \[
      R = \{e \to ae' \,|\,e'\in\delta(e,a)\} \cup \{e \to \lambda\,|\,e \in F\}
    \]
  \end{proof}

  \begin{Example}
    Como exemplo da construção do Teorema \ref{teorema2}, considere o seguinte
    AF:
    \begin{figure}[H]
      \begin{tikzpicture}
        \node[state,initial](s0){$A$};
        \node[state, accepting, right of=s0](s1){$B$};
        \node[state, below of=s0](s2){$C$} ;
        \node[state, right of=s2](s3){$D$} ;
        \draw (s0)edge[bend left, right]node{$a$}(s2)
              (s1)edge[bend left, right]node{$a$}(s3)
              (s3)edge[bend left, left]node{$a$}(s1)        
              (s0)edge[above]node{$b$}(s1)
              (s2)edge[above]node{$b$}(s3)
              (s2)edge[bend left, left]node{$a$}(s0) ;
       
      \end{tikzpicture}
      \centering
    \end{figure}
    A partir deste, podemos obter a seguinte gramática regular:
    \[
      \begin{array}{lcl}
        A & \to & aC \,|\, bB \\
        B & \to & aD \,|\, \lambda\\
        C & \to & aA \,|\, bD \\
        D & \to & aB
      \end{array}
    \]
  \end{Example}
  

  \section{Problemas de decisão para LRs}

  Os seguintes problemas de decisão sobre LRs são decidíveis (isto é, existem algoritmos
  que terminam e dão a resposta correta para todas as entradas):
  \begin{enumerate}
    \item Determinar se uma LR $L$ é infinita. Se $L$ é uma LR, existe um AFD
      $M$ para $L$. Sabe-se que a linguagem de um AFD é infinita se este possui
      um ciclo. Logo, o algoritmo para determinar se uma LR é infinita consiste
      em testar se o grafo correspondente ao seu AFD possui ciclos.
    \item Determinar se $L(M) = \emptyset$ para um AF $M$. Para saber se a
      linguagem aceita por um AF é vazia basta determinar se existe um caminho
      entre o estado inicial e algum estado final, o que pode ser determinado
      por um algoritmo de caminhamento em grafos.
    \item Dados dois AFs $M_1$ e $M_2$ determinar se $L(M_1) = L(M_2)$. Sejam
      $M'_1$ o AFD mínimo equivalente a $M_1$ e $M'_2$ o AFD mínimo equivalente
      a $M_2$. Temos que $L(M_1) = L(M_2)$ se, e somente se, os grafos
      correspondentes a $M'_1$ e $M'_2$ forem isomórficos.
  \end{enumerate}
  
  \section{Exercícios}

  \begin{enumerate}
     \item Para cada linguagem abaixo, apresente um AFD para a mesma. A partir
       do AFD apresentado, construa a GR usando a técnica do
       Teorema~\ref{teorema2}.
       \begin{enumerate}
         \item Palavras que começam e terminam com 1.
         \item Palavras que começam e terminam com 1 e tem pelo menos um 0.
         \end{enumerate}
      \item Utilizando a construção do Teorema \ref{teorema1}, apresente AFs
        para a seguinte GR.
    \[
      \begin{array}{lcl}
        A & \to & 0A \,|\,0\,|\,1B \\
        B & \to & 1B \,|\,\lambda
      \end{array}
    \]
      \item Apresente um algoritmo para decidir o seguinte problema: Dadas duas
        expressões regulares $e_1$  e $e_2$, decidir se $L(e_1) \subseteq
        L(e_2)$.
      \item Apresente um algoritmo para decidir o seguinte problema: Dados dois
        AF's $M_1$ e $M_2$, determinar se $L(M_1)$ e $L(M_2)$ são disjuntas.
      \end{enumerate}
\end{document}
