\documentclass[a4paper]{article}

\usepackage[portuguese]{babel}
\usepackage[utf8]{inputenc}
\usepackage{graphicx,hyperref}
\usepackage{float}
\usepackage{listings}
\usepackage{proof,tikz}
\usepackage{amssymb,amsthm,stmaryrd}


\usepackage[edges]{forest}
\usetikzlibrary{automata, positioning, arrows}


\newtheorem{Lemma}{Lema}
\newtheorem{Theorem}{Teorema}
\theoremstyle{definition}
\newtheorem{Example}{Exemplo}
\newtheorem{Definition}{Definição}


\usepackage{fancyhdr}
  \pagestyle{fancy}
  \fancyhf{}
  \lhead{Teoria da Computação}
  \rhead{Aula 11}
  \lfoot{Prof. Rodrigo Ribeiro}
  \rfoot{\thepage}
  \renewcommand{\footrulewidth}{0.4pt}
  \pagestyle{fancy}

\tikzset{
        ->,  % makes the edges directed
        >=stealth', % makes the arrow heads bold
        node distance=3cm,
        every state/.style={thick, fill=gray!10},
        initial text=$\,$
        }
  

\begin{document}

  \title{Aula extra 1 - Derivadas para expressões regulares}
  \author{Rodrigo Ribeiro}

  \maketitle


  \pagestyle{fancy}


  \section*{Objetivos}

  \begin{itemize}
     \item Apresentar o conceito de derivadas de expressão regular
       e como esse pode ser usado para implementar algoritmos
       de casamento de padrão.
     \item Apresentar a construção de um AFD a partir de uma
          expressão regular, usando derivadas.
  \end{itemize}


  \section{Derivadas de expressões regulares}

  \begin{Definition}[Derivada de linguagens]
    Seja $L \subseteq \Sigma^*$ e $a \in \Sigma$. A derivada
    de $L$ com respeito ao símbolo $a$ é definida como
    \[
      L_a =\{w \in \Sigma^*\,|\, aw \in L\}
    \]
  \end{Definition}

  \begin{Example}
    Considere a seguinte linguagem $L = \{01, 10, 11, 000, 101\}$.
    Temos que $L_0 =\{1, 00\}$. Isso é, $L_0$ é o conjunto formado
    pelo sufixo de palavras pertencentes a $L$ que começam com $0$,
    sem esse símbolo. Evidentemente, se uma palavra não inicia com $0$,
    essa não faz parte do conjunto $L_0$.
  \end{Example}

  A definição acima apresenta o conceito de derivada em termos de conjuntos.
  Uma definição alternativa define esse conceito em termos de expressões
  regulares. Para isso, é necessário definir uma função que determina se
  uma certa expressão regular $e$ aceita ou não a palavra vazia, $\lambda$.

  \begin{Definition}
    A função $\nu$ determina se $\lambda$ pertence a linguagem de uma expressão
    regular $e$.
    \[
      \begin{array}{lcl}
        \nu(\emptyset) & = & \bot\\
        \nu(\lambda)   & = & \top \\
        \nu(a)         & = & \bot \\
        \nu(e_1\,e_2)  & = & \nu(e_1) \land \nu(e_2)\\
        \nu(e_1 + e_2) & = & \nu(e_1) \lor \nu(e_2)\\
        \nu(e^*)       & = & \top  
      \end{array}
    \]
  \end{Definition}  

  \begin{Example}
    A expressão regular $a^*(b + \lambda)$ aceita a palavra vazia, conforme
    mostrado pela execução da função $\nu$ a seguir.
    \[
      \begin{array}{lc}
        \nu(a^*(b + \lambda)) & = \\
        \nu(a^*) \land \nu(b + \lambda) & = \\
        \top \land (\nu(b) \lor \nu(\lambda)) & = \\
        \top \land (\bot \lor \top) & = \\
        \top
      \end{array}
    \]
    Por sua vez, a expressão regular $(a + b)c^*$ não aceita a string vazia,
    uma vez que $\nu((a + b)c^*) = \bot$, conforme apresentado a seguir.
    \[
      \begin{array}{lc}
        \nu((a + b)c^*) & = \\
        \nu(a + b) \land \nu(c^*) & = \\
        (\nu(a) \lor \nu(b)) \land \top & = \\
        (\bot \lor \bot) \land \top & = \\
        \bot
      \end{array}
    \]
  \end{Example}
  
  Utilizando a função $\nu$, podemos definir a função $\delta_a(e)$ que calcula
  a derivada de uma expressão regular $e$ com respeito a um símbolo $a$.

  \begin{Definition}[Derivada de uma expressão regular]
    A função $\delta_a(e)$ calcula a derivada de uma expressão regular $e$
    com respeito a um símbolo $a$ e é definida como:
    \[
      \begin{array}{lcl}
        \delta_a(\emptyset) & = & \emptyset \\
        \delta_a(\lambda)   & = & \lambda \\
        \delta_a(b)         & = & \left\{  \begin{array}{ll}
                                             \lambda   & \textit{se } a = b \\
                                             \emptyset & \textit{caso contrário}\\ 
                                           \end{array}\right. \\
        \delta_a(e_1\,e_2)  & = & \left\{  \begin{array}{ll}
                                             \delta_a(e_1)e_2 + \delta_a(e_2) &
                                                                                \textit{se
                                                                                }
                                                                                \nu(e_1)
                                                                                =
                                                                                \top
                                             \\
                                             \delta_a(e_1)e_2 & \textit{caso contrário}\\
                                           \end{array}\right. \\
        \delta_a(e_1 + e_2) & = & \delta_a(e_1) + \delta_a(e_2) \\
        \delta_a(e^*)  & = & \delta_a(e)e^*\\
      \end{array}
    \]
  \end{Definition}

  \begin{Example}
    Considere a seguinte expressão regular $(ab)^*$. A seguir apresentamos o
    cálculo de $\delta_a((ab)^*)$.
    \[
      \begin{array}{lc}
      \delta_a((ab)^*) & = \\
      \delta_a(ab)(ab)^* & = \\
      \delta_a(a)b(ab)^* & = \\
      \lambda b(ab)^* & = \\
        b(ab)^* \\
      \end{array}
    \]
    Note que $(ab)^*$ é o conjunto de palavras formadas por repetições de $ab$.
    Logo, a derivada com repeito a $a$ de $(ab)^*$ é o conjunto de palavras
    $b(ab)^*$, isto é palavras que começam com $b$, visto que o primeiro $a$ é
    removido pela operação de derivada.

    Por sua vez, $\delta_b((ab)^*) = \emptyset$, como pode ser observado pelo
    cálculo a seguir.
    \[
      \begin{array}{lc}
      \delta_b((ab)^*) & = \\
      \delta_b(ab)(ab)^* & = \\
      \delta_b(a)b(ab)^* & = \\
      \emptyset b (ab)^* & = \\
        \emptyset\\
        \end{array}
    \]  
  \end{Example}

  \subsection{Construindo um AFD utilizando derivadas}

  A construção de um AFD a partir de uma expressão regular utilizando derivadas
  consiste é explicado utilizando o exemplo a seguir.

  \begin{Example}
    Considere a seguinte expressão regular: $(ab)^*$. O AFD equivalente a essa
    expressão pode ser construído usando derivadas da seguinte maneira.

    Primeiramente, temos que o estado inicial do AFD corresponde a expressão
    regular original.

    \begin{figure}[H]
      \begin{tikzpicture}
        \node[state, initial](s0){$(ab)^*$} ;
      \end{tikzpicture}
      \centering
    \end{figure}

    Agora, calculamos a derivada com respeito a cada símbolo do alfatebo para
    a expressão regular do estado inicial. Como já mostramos nos exemplos
    anteriores, essas são $\delta_a((ab)^*) = b(ab)^*$ e
    $\delta_b((ab)^*) = \emptyset$. Deste resultado, obtemos duas novas
    expressões regulares que serão incluídas como novos estados neste AFD.
    Além disso, incluiremos transições do estado inicial para estes novos
    estados usando o símbolo do alfabeto que foi utilizado no cálculo
    da derivada.
    \begin{figure}[H]
      \begin{tikzpicture}
        \node[state, initial](s0){$(ab)^*$} ;
        \node[state, right of=s0](s1){$b(ab)^*$} ;
        \node[state, below of=s0](s2){$\emptyset$} ;
        \draw (s0) edge[above] node{$a$}(s1)
              (s0) edge[left] node{$b$}(s2) ; 
      \end{tikzpicture}
      \centering
    \end{figure}
    Tendo realizado todas as transições sobre o estado correspondente a
    $(ab)^*$, repetimos o processo para os outros estados. Evidentemente, o
    resultado de $\delta_a(\emptyset) = \delta_b(\emptyset) = \emptyset$.
    \begin{figure}[H]
      \begin{tikzpicture}
        \node[state, initial](s0){$(ab)^*$} ;
        \node[state, right of=s0](s1){$b(ab)^*$} ;
        \node[state, below of=s0](s2){$\emptyset$} ;
        \draw (s0) edge[above] node{$a$}(s1)
              (s0) edge[left] node{$b$}(s2) 
              (s2) edge[loop left] node{$a,b$}(s2);
      \end{tikzpicture}
      \centering
    \end{figure}
    Para o estado $b(ab)^*$, temos que $\delta_a(b(ab)^*) = \emptyset$ e
    $\delta_b(b(ab)^*) = (ab)^*$, o que nos faz incluir novas transições no AFD
    anterior.
    \begin{figure}[H]
      \begin{tikzpicture}
        \node[state, initial](s0){$(ab)^*$} ;
        \node[state, right of=s0](s1){$b(ab)^*$} ;
        \node[state, below of=s0](s2){$\emptyset$} ;
        \draw (s0) edge[above, bend left] node{$a$}(s1)
              (s0) edge[left] node{$b$}(s2) 
              (s2) edge[loop left] node{$a,b$}(s2)
              (s1) edge[below, bend left] node{$b$}(s0)
              (s1) edge[above] node{$a$}(s2) ;
      \end{tikzpicture}
      \centering
    \end{figure}
    Como todos os estados possuem transiçõs para todos os símbolos do alfabeto,
    temos que a construção do AFD está quase concluída. Falta apenas marcar os
    estados finais que são aqueles cujas expressões regulares aceitam a palavra
    vazia. Abaixo apresentamos a versão final do AFD para a expressão regular
    $(ab)^*$.
    \begin{figure}[H]
      \begin{tikzpicture}
        \node[state, initial, accepting](s0){$(ab)^*$} ;
        \node[state, right of=s0](s1){$b(ab)^*$} ;
        \node[state, below of=s0](s2){$\emptyset$} ;
        \draw (s0) edge[above, bend left] node{$a$}(s1)
              (s0) edge[left] node{$b$}(s2) 
              (s2) edge[loop left] node{$a,b$}(s2)
              (s1) edge[below, bend left] node{$b$}(s0)
              (s1) edge[above] node{$a$}(s2) ;
      \end{tikzpicture}
      \centering
    \end{figure}
  \end{Example}

  \subsection{Casamento de padrão utilizando derivadas}

  Podemos estender a noção de derivada para palavras ao invés de símbolos de um
  alfabeto. A idéia é calcular a derivada para cada símbolo da palavra até que
  todos sejam processados. Se ao final desse processo obtermos uma expressão
  regular que aceita a palavra vazia, temos que a string processada pertence
  a linguagem da expressão regular em questão.

  A seguir apresentamos essa função.

  \[
    \begin{array}{lcl}
      \delta^*(e,\lambda) & = & \nu(e) \\
      \delta^*(e,ay)      & = & \delta^*(\delta_a(e), y) \\
    \end{array}
  \]

  \begin{Example}
    A palavra $abab$ pertence à linguagem da expressão regular $(ab)^*$,
    conforme ilustrado pelo cálculo a seguir.
    \[
      \begin{array}{lc}
        \delta^*((ab)^*,abab) & = \\
        \delta^*(\delta_a((ab)^*),bab) & = \\
        \delta^*(b(ab)^*,bab) & = \\
        \delta^*(\delta_b(b(ab)^*),ab) & = \\
        \delta^*((ab)^*,ab) & = \\
        \delta^*(\delta_a((ab)^*),b) & = \\
        \delta^*(b(ab)^*,b) & = \\
        \delta^*(\delta_b(b(ab)^*),\lambda) & = \\
        \delta^*((ab)^*,\lambda) & = \\
        \nu((ab)^*) & = \\
        \top
      \end{array}
    \]
  \end{Example}
  
  \section{Exercícios} 

  \begin{enumerate}
     \item Usando o método baseado em derivadas, construa um AFD para $(00 + 1)^*01$.
  \end{enumerate}
\end{document}
