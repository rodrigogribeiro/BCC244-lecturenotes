\documentclass[a4paper]{article}

\usepackage[portuguese]{babel}
\usepackage[utf8]{inputenc}
\usepackage{graphicx,hyperref}
\usepackage{float}
\usepackage{listings}
\usepackage{proof,tikz}
\usepackage{algorithm2e}
\usepackage{amssymb,amsthm,stmaryrd}


\usepackage[edges]{forest}
\usetikzlibrary{automata, positioning, arrows}


\newtheorem{Lemma}{Lema}
\newtheorem{Theorem}{Teorema}
\theoremstyle{definition}
\newtheorem{Example}{Exemplo}
\newtheorem{Definition}{Definição}
\newtheorem{Fact}{Fato}


\usepackage{fancyhdr}
  \pagestyle{fancy}
  \fancyhf{}
  \lhead{Teoria da Computação}
  \rhead{Aula 27}
  \lfoot{Prof. Rodrigo Ribeiro}
  \rfoot{\thepage}
  \renewcommand{\footrulewidth}{0.4pt}
  \pagestyle{fancy}

\tikzset{
        ->,  % makes the edges directed
        >=stealth', % makes the arrow heads bold
        node distance=3cm,
        every state/.style={thick, fill=gray!10},
        initial text=$\,$
        }
  

\begin{document}

\title{Aula 27 - Problemas Indecidíveis sobre GLCs}
  \author{Rodrigo Ribeiro}

  \maketitle

  \pagestyle{fancy}


  \section*{Objetivos}

  \begin{itemize}
    \item Apresentar como reduzir o PCP a diversos problemas envolvendo GLCs.
  \end{itemize}


  \section{Introdução}

  O objetivo desta aula é apresentar a prova de indecibilidade de alguns
  problemas envolvendo GLCs. Mostraremos que o problema de determinar se
  uma gramática arbitrária é ambígua é indecidível. A prova de que esse
  problema é indecidível consiste em uma redução ao problema de correspondência
  de Post.


  \section{Reduzindo o SCP a GLCs}

  Dado um SCP $S = (\Sigma, P)$, em que $P = [(x_1,y_1),...,(x_n,y_n)]$
  podemos construir duas GLCs, que chamaremos de $G_x$ e $G_y$, que
  representarão os componentes de um SCP. A construção seguinte ilustra como
  definir essas gramáticas.
  
  
  \begin{Definition}
    Seja um SCP $S = (\Sigma, P)$, em que $P = [(x_1,y_1),...,(x_n,y_n)]$ e
    $\{s_1,...,s_n\}$ símbolos distintos tais que $\forall i. s_i\not\in\Sigma$.
    O objetivo dos símbolos $s_i$ é que esses são utilizados para indexar pares
    da solução do SCP. Usando o SCP, podemos construir duas GLCs sobre o
    alfabeto $\Sigma' = \Sigma\cup\{s_1,...,s_n\}$:
    \begin{itemize}
       \item $G_x=(\{P_x\},\Sigma', R_x, P_x)$, em que as regras são da forma:
         \begin{itemize}
           \item $P_x \to x_iP_x s_i$, para $1 \leq i \leq n$;
           \item $P_x \to x_is_i$, para $1 \leq i \leq n$.
           \end{itemize}
       \item $G_y=(\{P_y\},\Sigma', R_y, P_y)$, em que as regras são da forma:
         \begin{itemize}
           \item $P_y \to y_iP_y s_i$, para $1 \leq i \leq n$;
           \item $P_y \to y_is_i$, para $1 \leq i \leq n$.
         \end{itemize}
    \end{itemize}
  \end{Definition}

  A seguir ilustramos essa construção para um SCP de exemplo.

  \begin{Example}
    Seja o SCP $(\{0,1\}, [(10,0), (0,010), (01,11)])$. Como esse SCP possui 3
    pares, usaremos os símbolos $\{s_1,s_2,s_3\}$ para indexar estes pares. As
    gramáticas produzidas para o SCP acima são:
    \begin{itemize}
      \item Gramática $P_x$
        \[
          \begin{array}{lcl}
            P_x & \to &  10 P_x s_1 \,|\, 0P_x s_2 \,|\, 01P_x s_3\\
                & \mid & 10s_1 \,|\, 0s_2\,|\, 01s_3\\
          \end{array}
        \]
      \item Gramática $P_y$
        \[
          \begin{array}{lcl}
            P_y & \to & 0P_y s_1 \,|\, 010P_ys_2 \,|\, 11 P_y s_3 \\
                & \mid & 0s_1 \,|\, 010s_2 \,|\, 11s_3\\
          \end{array}
        \]
      \end{itemize}

      Conforme visto na aula 25, temos que o SCP acima possui solução 2131.
      Logo, podemos mostrar uma derivações nessas gramáticas que refletem
      a solução do SCP. Primeiramente, a derivação em $P_x$:
      \[
        \begin{array}{lcll}
          P_x & \Rightarrow & 0P_xs_2       & \{P_x \to 0 P_x s_2\} \\
              & \Rightarrow & 010P_x s_1s_2 & \{P_x \to 10 P_x s_1\} \\
              & \Rightarrow & 01001P_x s_3s_1s_2 & \{P_x \to 01 P_x s_3\} \\
              & \Rightarrow & 0100110s_1s_3s_1s_2 & \{P_x \to 10s_1\}\\
        \end{array}
      \]
      A mesma derivação na gramática $P_y$ gera:
      \[
        \begin{array}{lcll}
          P_y & \Rightarrow & 010P_ys_2       & \{P_y \to 010 P_y s_2\} \\
              & \Rightarrow & 0100P_ys_1s_2   & \{P_y \to 0P_y s_1\}\\
              & \Rightarrow & 010011P_ys_3s_1s_2 & \{P_y \to 11 P_y s_3\} \\
              & \Rightarrow & 0100110s_1s_3s_1s_2 & \{P_y \to 0s_1\}\\
        \end{array}
      \]
      Observe que $P_x \Rightarrow^* 100110 s_1 s_3 s_1 s_2$ e
      $P_y \Rightarrow^* 100110 s_1 s_3 s_1 s_2$.
    \end{Example}

    Do exemplo anterior, podemos perceber que se $i_1 i_2 ... i_k$ é uma
    solução para um SCP, existem derivações nas gramáticas $G_x$ e $G_y$ tais
    que $P_x \Rightarrow x_{i_1}...x_{i_k}s_{i1}...s_{ik}$, 
    $P_y \Rightarrow y_{i_1}...y_{i_k}s_{i1}...s_{ik}$. e
    $x_{i_1}...x_{i_k} = y_{i_1}...y_{i_k}$.

    Desse fato, podemos concluir que o SCP possui solução se, e somente se,
    $L(G_x) \cap L(G_y) \neq \emptyset$ (pois ambas derivam a mesma palavra que
    corresponde a solução do SCP).

    \begin{Theorem}
      O problema de determinar se duas GLCs são disjuntas é indecidível.
    \end{Theorem}
    \begin{proof}
      Consequência da construção apresentada para as gramáticas $G_x$ e $G_y$.
    \end{proof}

    Conforme apresentado em aulas anteriores, as LLCs não são fechadas sobre a
    complementação. Porém, as gramáticas $G_x$ e $G_y$ possuem GLCs que geram o
    seu complemento.

    \begin{Definition}[Gramática para o complemento de $G_x$ e $G_y$]
      Seja um SCP $S = (\Sigma, P)$, em que $P = [(x_1,y_1),...,(x_n,y_n)]$ e
      $\{s_1,...,s_n\}$ símbolos distintos tais que $\forall i.
      s_i\not\in\Sigma$. A gramática que gera $\overline{G_x}$ é
      $(\{P,X\}, \Sigma \cup \{s_1,...,s_n\}, R, P)$, em que o conjunto de
      regras $R$ é dado por:
      \[
        \begin{array}{lcll}
          P & \to & x_i P s_i & \textit{para }1 \leq i \leq n \\
          P & \to & Xa        & a \in \Sigma \\
          P & \to & s_i X     & \textit{ para } 1 \leq i \leq n \\
          P & \to & ya X s_i        & \textit{ para }1 \leq i \leq n \textit{ }y
                                \textit{ é prefixo de }x_i, |ya| \leq |x_i|,\\ &
                  & &
                                      ya \neq x_i, a \in \Sigma \cup\{s_1,...,s_n\} \\
          X & \to a X         & a \in \Sigma \cup \{s_1,..., s_n\} \\
          X & \to \lambda \\
        \end{array}
      \]
    \end{Definition}

    Usando a construção acima, podemos mostrar a indecidibilidade de outro
    problema envolvendo GLCs.

    \begin{Theorem}
      O problema de determinar se $L(G) = \Sigma^*$ é indecidível para uma
      GLC $G$ arbitrária.
    \end{Theorem}
    \begin{proof}
      Sejam $G_x$ e $G_y$ as gramáticas correspondentes a um SCP $S$.
      Construiremos uma gramática $G$ tal que $L(G) = \Sigma^*$ se, e somente
      se, o SCP $S$ não possui solução. Seja $L(G) = \overline{L(G_x)} \cup
      \overline{L(G_y)}$. Temos:
      \[
        \begin{array}{lc}
          \overline{L(G_x)} \cup \overline{L(G_y)} = \Sigma^* & \leftrightarrow \\
          \overline{\overline{L(G_x)} \cup \overline{L(G_y)}} =
          \overline{\Sigma^*} & \leftrightarrow \\
          \overline{\overline{L(G_x)}} \cap \overline{\overline{L(G_y)}} =
          \emptyset & \leftrightarrow \\
          L(G_x) \cap L(G_y) = \emptyset & \\
        \end{array}
      \]
      Porém, conforme provado anteriormente, se um SCP $S$ possui solução, então
      $L(G_x) \cap L(G_y) \neq \emptyset$. Dessa forma, temos que o SCP $S$ não
      possui solução se $L(G_x) \cap L(G_y) = \emptyset$. Dessa forma, temos que
      o problema de determinar se $L(G) = \Sigma^*$ é indecidível. 
    \end{proof}

    Encerramos essa aula apresentando a demonstração de que o problema de
    determinar se uma GLC é ambígua é também indecidível.

    \begin{Theorem}
      O problema de determinar se uma GLC arbitrária $G$ é ambígua é
      indecidível.
    \end{Theorem}
    \begin{proof}
      Sejam $G_x$ e $G_y$ as gramáticas geradas a partir de um SCP $S$ e $P_x$ e
      $P_y$ os símbolos de partida de $G_x$ e $G_y$, respectivamente.
      Considere a seguinte gramática $G = (\{P,P_x,P_y\},
      \Sigma\cup\{s_1,...,s_n\},R_x \cup R_y \cup \{P \to P_x \,|\, P_y\}, P)$.
      Note que $G$ será ambígua exatamente nas situações em que $L(G_x) \cap
      L(G_y) \neq \emptyset$, ou seja, quando o PCP possuir solução.
    \end{proof}

    \section*{Exercícios}

    \begin{enumerate}
      \item Construa as gramáticas $G_x$ e $G_y$ para o seguinte SCP $S =
        (\{0,1\}, P)$, em que $P = [(01,011), (001,01), (10,00)]$.
      \item Prove que os problemas seguintes são decidíveis ou indecidíveis.
        \begin{enumerate}
          \item Determinar se $L(G_1) \cap L(G_2) = \emptyset$ quando $G_1$ é
            uma gramática livre de contexto e $G_2$ uma gramática regular.
          \item Determinar se $L(G_1) \cap L(G_2)$ é finito, quando $G_1$ e
            $G_2$ são gramáticas livres de contexto.
          \item Determinar se $L(G_1) \cup L(G_2)$ é finito, quando $G_1$ e
            $G_2$ são gramáticas livres de contexto.
          \item Determinar se $L(G_1) = L(G_2)$ é finito, quando $G_1$ e
            $G_2$ são gramáticas livres de contexto.
        \end{enumerate}
    \end{enumerate}
\end{document}

