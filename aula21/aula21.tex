\documentclass[a4paper]{article}

\usepackage[portuguese]{babel}
\usepackage[utf8]{inputenc}
\usepackage{graphicx,hyperref}
\usepackage{float}
\usepackage{listings}
\usepackage{proof,tikz}
\usepackage{algorithm2e}
\usepackage{amssymb,amsthm,stmaryrd}


\usepackage[edges]{forest}
\usetikzlibrary{automata, positioning, arrows}


\newtheorem{Lemma}{Lema}
\newtheorem{Theorem}{Teorema}
\theoremstyle{definition}
\newtheorem{Example}{Exemplo}
\newtheorem{Definition}{Definição}
\newtheorem{Fact}{Fato}


\usepackage{fancyhdr}
  \pagestyle{fancy}
  \fancyhf{}
  \lhead{Teoria da Computação}
  \rhead{Aula 21}
  \lfoot{Prof. Rodrigo Ribeiro}
  \rfoot{\thepage}
  \renewcommand{\footrulewidth}{0.4pt}
  \pagestyle{fancy}

\tikzset{
        ->,  % makes the edges directed
        >=stealth', % makes the arrow heads bold
        node distance=3cm,
        every state/.style={thick, fill=gray!10},
        initial text=$\,$
        }
  

\begin{document}

\title{Aula 21 - Conjuntos Enumeráveis e o Teorema de Cantor}
  \author{Rodrigo Ribeiro}

  \maketitle

  \pagestyle{fancy}


  \section*{Objetivos}

  \begin{itemize}
     \item Apresentar os conceito de conjunto enumerável e alguns
       exemplos.
     \item Apresentar o Teorema de Cantor e sua aplicação para
       mostrar que alguns conjuntos não são enumeráveis.
     \item Discutir a relação entre o Teorema de Cantor e a
       decidibilidade de problemas.
  \end{itemize}

  \section{Uma pequena revisão sobre funções}

  Nesta aula faremos uso de diversos conceitos vistos anteriormente por vocês.
  Apresentaremos as definições necessárias de maneira breve.

  \begin{Definition}
    Sejam $A,B$ conjuntos quaisquer. Dizemos que $f \subseteq A \times B$ é uma
    função se $\forall x. x \in A \to \exists ! y . y \in B \land (x,y) \in f$.
    Usamos a notação $f : A \to B$ para representar o fato de que $f \subseteq
    A\times B$ é uma função. Usamos a notação $f(x) = y$ para representar
    $(x,y) \in f$.
  \end{Definition}

  \begin{Definition}
    Seja $f : A\to B$ uma função. Dizemos que $f$ é uma função injetora se
    \[
      \forall a_1.\forall a_2.  a_1 \in A \land a_2 \in A \to f(a_1) = f(a_2)
      \to a_1 = a_2
    \]
  \end{Definition}

  \begin{Definition}
    Seja $f : A\to B$ uma função. Dizemos que $f$ é uma função sobrejetora se
    \[
      \forall y . y \in B \to \exists x. x \in A \land f(x) = y.
    \]
  \end{Definition}
  
  \begin{Definition}
    Seja $f : A\to B$ uma função. Dizemos que $f$ é uma função bijetora se $f$
    for injetora e sobrejetora.
  \end{Definition}

  \begin{Fact}
    Se $f : A \to B$ é uma função bijetora, então sua inversa $f^{-1} : B \to A$
    também é uma função bijetora.
  \end{Fact}

  \begin{Fact}
    Sejam $f : A \to B$ e $g : B \to C$ duas funções bijetoras. Então,
    $g \circ f : A \to C$ também é uma função bijetora.
  \end{Fact}
  
  
  \section{Conjuntos enumeráveis}

  \begin{Definition}[Conjuntos de mesma cardinalidade]
    Sejam $A$ e $B$ dois conjuntos quaisquer. Dizemos que $A$ e $B$ possuem
    a mesma cardinalidade, $A \sim B$, se existe uma função bijetora $f : A \to B$.
  \end{Definition}

  Usando a definição de conjuntos de mesma cardinalidade, podemos definir
  o conceito de conjunto infinito e finito.

  \begin{Definition}[Conjuntos finitos e infinitos]
    Seja $n \in \mathbb{N}$. Definimos $I_n =\{i \in \mathbb{N}^+\,|\, i \leq
    n\}$. Dizemos que um conjunto $A$ é finito se existe $n \in \mathbb{N}$, tal
    que $A \sim I_n$. Dizemos que um conjunto é infinito se não existe $n \in
    \mathbb{N}$ tal que $f : A \to I_n$.
  \end{Definition}

  As definições anteriores tem algumas consequências interessantes. Por exemplo,
  podemos mostrar que o conjunto dos números inteiros, $\mathbb{Z}$, possui a
  mesma cardinalidade dos números naturais, $\mathbb{N}$. Note que o bizarro
  deste fato é que $\mathbb{N} \subset \mathbb{Z}$! A seguinte função bijetora
  (prove este fato!) mostra que $\mathbb{N} \sim \mathbb{Z}$.

  \[
    f(n) = \left\{
      \begin{array}{ll}
        \frac{n}{2} & \textit{se }n\textit{ é par.}\\
        \frac{1 - n}{2} & \textit{se }n\textit{ é ímpar}\\
      \end{array}
           \right.
  \]

  Pode-se mostrar que a relação $\sim$ é uma relação de equivalência, isto é,
  que esta é uma relação reflexiva, transitiva e simétrica.

  \begin{Definition}
    Dizemos que um conjunto $A$ é contável (enumerável) se este é finito ou se $A
    \sim\mathbb{N}$. Caso contrário, este é dito não contável.
  \end{Definition}

  \begin{Example}
    Mostraremos que o conjunto $(\mathbb{N},\mathbb{N})$ é contável. Para isso,
    vamos utilizar a seguinte função:
    \[
      f(i,j) = \frac{(i + j - 2)(i + j - 1)}{2} + i
    \]
    Pode-se mostrar que essa função $f : \mathbb{N}\times\mathbb{N}\to
    \mathbb{N}$ é uma função bijetora.
  \end{Example}

  \begin{Theorem}
    Suponha que $A$ e $B$ são conjuntos enumeráveis. Então, $A\times B$ é
    enumerável e $A \cup B$ é enumerável.
  \end{Theorem}
  \begin{proof}
    Como $A$ e $B$ são enumeráveis, existem funções $f : A \to \mathbb{N}$ e
    $g : B \to \mathbb{N}$ bijetoras. As seguintes funções mostram que tanto o
    produto cartesiano quanto a união produzem conjuntos enumeráveis.
    \[
      \begin{array}{lcl}
        h(x,y) & = & (f(x),g(y))\\ \\
        h(x) & = & \left\{
                   \begin{array}{ll}
                     f(x)  & \textit{se }x\in A\\
                     -g(x) & \textit{se }x \in B\\
                   \end{array}
                   \right.
      \end{array}
    \]
    Observe que no caso da união, estamos definindo uma função $h : A \cup B \to
    \mathbb{Z}$. Porém, como apresentado anteriormente, $\mathbb{Z}\sim
    \mathbb{N}$. Como $\sim$ é uma relação transitiva, temos que $A \cup B \sim
    \mathbb{N}$ conforme requerido.
  \end{proof}

  \begin{Theorem}
    Seja $\mathcal{F}$ uma família de conjuntos tal que $\mathcal{F}$ é
    enumerável e todo elemento de $\mathcal{F}$ é enumerável. Então
    $\bigcup \mathcal{F}$ é enumerável.
  \end{Theorem}
  \begin{proof}
    Como $\mathcal{F}$ é contável, é possível indexar cada um dos conjuntos que
    compõe $\mathcal{F}$ por um número natural:
    \[
      \mathcal{F} =\{A_1,A_2,...\}
    \]
    Porém, sabemos que cada elemento de $\mathcal{F}$ é também enumerável.
    Assim, temos que podemos indexá-los por número inteiro:
    \[
      \begin{array}{l}
        A_1 = \{a^1_1, a^2_1,a^3_1...\}\\
        A_2 = \{a^1_2, a^2_2,a^3_2...\}\\
        A_3 = \{a^1_3, a^2_3,a^3_3...\}\\
        \vdots
      \end{array}
    \]
    Note que $\bigcup\mathcal{F} = \{a^i_j\,|\,i,j\in\mathbb{N}\}$. Seja $f :
    \mathbb{N}\times\mathbb{N} \to \bigcup{F}$ definida como:
    \[
      f(i,j) = a^i_j
    \]
    Como sabemos que $\mathbb{N} \times \mathbb{N}$ é enumerável, temos que
    existe uma função bijetora $g : \mathbb{N} \to \mathbb{N} \times
    \mathbb{N}$. Dessa forma $f \circ g : \mathbb{N} \to \bigcup\mathcal{F}$
    é também uma função bijetora, concluindo a demonstração.
  \end{proof}

  \begin{Definition}
    Uma sequência de elementos sobre um conjunto $A$ é uma função $f : I_n\to
    A$, em que $n \in \mathbb{N}$ é o tamanho da sequência $f$.
  \end{Definition}

  Intuitivamente, sequências são o equivalente matemática de listas presentes em
  linguagens de programação. Quando consideramos o conjunto $A$ como sendo um
  alfabeto finito, temos que sequências nada mais são que strings sobre esse
  determinado alfabeto. A seguir apresentamos que o conjunto de sequências
  finitas sobre um conjunto $A$ é enumerável, se $A$ é um conjunto enumerável.
  Para mostrar esse fato, primeiramente provaremos um resultado auxiliar.

  \begin{Lemma}
    Seja $S_n$ o conjunto de todas as sequências de tamanho $n\in\mathbb{N}$ de
    um conjunto enumerável $A$. Então $S_n$ é enumerável.
  \end{Lemma}
  \begin{proof}
    No caso base, temos $n = 0$. Note que $I_0 = \emptyset$ e assim uma
    sequência de tamanho $n = 0$ é uma função $f : \emptyset \to A$, que
    é exatamente o conjunto $\emptyset$. Logo, $S_0 = \{\emptyset\}$ é
    enumerável. Para o passo indutivo, suponha $n\in\mathbb{N}$ arbitrário e que
    $S_n$ é enumerável. Seja $F : A \times S_n \to S_{n + 1}$ definida como:
    \[
      F(a,f) = f \cup \{(n+1,a)\}
    \]
    Note que $F(a,f)$ é exatamente a operação de inserir $a$ na sequência $f$.
    Logo, $F$ é uma função bijetora. Dessa forma, temos que $A \times S_n \sim
    S_{n + 1}$. Porém, como $A$ e $S_n$ são contáveis e o produto cartesiano de
    dois conjuntos contáveis é também contável, temos que $S_{n + 1}$ é contável.
  \end{proof}

  \begin{Theorem}
    O conjunto de todas as sequências finitas de um conjunto contável $A$ é
    também contável.
  \end{Theorem}
  \begin{proof}
    Note que o conjunto de todas as sequências finitas é
    $\bigcup_{n\in\mathbb{N}}S_n$. Pelo Lema 1, temos que $S_n$ é contável. Pelo
    teorema 2, temos que a união de famílias enumeráveis de conjuntos
    enumeráveis é também enumerável.
  \end{proof}

  Note que o teorema 3 implica que $\Sigma^*$ é um conjunto enumerável.

  \section{O teorema de Cantor}

  A partir dos resultados apresentados acima pode-se pensar que todos os
  conjuntos são enumeráveis e que qualquer operação preserva a cardinalidade
  de conjuntos. Porém, não é esse o caso. A operação de conjunto potência não
  preserva a cardinalidade de um conjunto. Esse fato é conhecido como teorema
  de Cantor e é apresentado a seguir.

  \begin{Theorem}
    O conjunto $\mathcal{P}(\mathbb{N})$ não é enumerável.
  \end{Theorem}
  \begin{proof}
    A demonstração consiste em mostrar que não existe uma função $f : \mathbb{N}
    \to \mathcal{P}(\mathbb{N})$ bijetora. Para isso, mostraremos que
    $f : \mathbb{N} \to \mathcal{P}(\mathbb{N})$ não pode ser sobrejetora o que
    é suficiente para mostrar que $\mathcal{P}(\mathbb{N})$ não é contável.
    Dizer que $f$ não é sobrejetora pode ser descrito como:
    \[
      \neg \forall D . D \in \mathcal{P}(\mathbb{N}) \to \exists n. n \in \mathbb{N} \land f(n) = D.
    \]
    Usando álgebra Booleana, temos que a fórmula acima é equivalente a:
    \[
      \exists D. D \in \mathcal{P}(\mathbb{N}) \land \forall n. n\in\mathbb{N}
      \to f(n) \neq D
    \]
    Logo, o maior problema é encontrar um conjunto $D \subset \mathbb{N}$ tal
    que $f(n) \neq D$ para todo $n$. Basta fazer $D$ ser o seguinte conjunto.
    \[
      D = \{n \in \mathbb{N}\,|\,n \not\in f(n)\}
    \]
    Pela definição de $D$, temos que $D\subseteq \mathbb{N}$ e, portanto,
    $D \in \mathcal{P}(\mathbb{N})$. Suponha $x \in \mathbb{N}$ arbitrário e
    considere os seguintes casos.
    \begin{enumerate}
      \item $x \in f(x)$: Pela definição de $D$, podemos concluir que $x \not\in
        D$. Como $x \in f(x)$ e $x\not\in D$, temos que $f(x) \neq D$.
      \item $x \not\in f(x)$: Pela definição de $D$, podemos concluir que $x \in
        D$. Como $x\not\in f(x)$ e $x\in D$, temos que $f(x) \neq D$.
    \end{enumerate}
    Como $x\in\mathbb{N}$ é arbitrário temos que $\forall x. x\in\mathbb{N} \to
    D \neq f(x)$ e, portanto, $f$ não pode ser sobrejetora. Dessa forma, temos
    que não existe função $f : \mathbb{N} \to \mathcal{P}(\mathbb{N})$ bijetora,
    o que nos leva a concluir que  $\mathcal{P}(\mathbb{N})$ não é enumerável.
  \end{proof}

  Observe que o conjunto de todas as linguagens sobre um determinado alfabeto
  não é enumerável. 

  \section{Teorema de Cantor e Indecidibilidade}

  É fácil perceber que $|A| < \mathcal{P}(A)$, quando $A$ é finito. Porém, uma
  consequência do teorema de Cantor, é que a mesma idéia vale para conjuntos
  infinitos. Se $A$ é um conjunto enumerável infinito, então
  $A \not\sim \mathcal{P}(A)$. De maneira informal, isso quer dizer que o
  conjunto potência sempre terá maior cardinalidade que o conjunto que o
  originou.
  
  Para entender a relação entre o teorema de Cantor e a decidibilidade de
  problemas, considere os seguintes fatos:
  \begin{itemize}
     \item Podemos considerar o código ASCII ou o Unicode como um alfabeto.
     \item Programas expressos em uma linguagem de programação qualquer são
           palavras sobre um alfabeto.
  \end{itemize}
  Como apresentado anteriormente, o conjunto palavras (sequências) sobre um
  alfabeto é enumerável. Dessa forma, o conjunto de todos os programas expressos
  em uma linguagem de programação é enumerável.

  Se conseguimos expressar programas como sequências, podemos pensar que
  entradas para programas também podem ser representadas de maneira similar.
  Porém, cada programa possui seu próprio conjunto de entradas. Logo, podemos
  associar a um problema (a ser resolvido por um algoritmo) um conjunto que
  representa todas as suas possíveis entradas. Assim sendo, temos que para cada
  programa, existe uma linguagem correspondente as entradas por ele demandadas.
  Logo, o conjunto de todas as possíveis entradas para problemas é o conjunto
  de todas as possíveis linguagens sobre um dado alfabeto, isto é,
  $\mathcal{P}(\Sigma^*)$.

  
  O que isso quer dizer? De maneira intuitiva, isso representa que o conjunto
  $\mathcal{P}(\Sigma^*)$ possui uma cardinalidade maior que $\Sigma^*$.
  Conforme discutido anteriormente, podemos representar programas como palavras
  em $\Sigma^*$ e problemas como o conjunto de suas entradas, isto é,
  linguagens (elementos de $\mathcal{P}(\Sigma^*)$). Disso segue que
  \textbf{existem mais problemas que programas aptos a resolvê-los}. Logo,
  existem problemas para os quais não há solução.
  
  \section{Exercícios}

  \begin{enumerate}
    \item Prove que a relação $\sim$ é uma relação de equivalência.
    \item Prove que o conjunto de todos os números naturais ímpares é contável.
  \end{enumerate}
\end{document}
