\documentclass[a4paper]{article}

\usepackage[portuguese]{babel}
\usepackage[utf8]{inputenc}
\usepackage{graphicx,hyperref}
\usepackage{float}
\usepackage{listings}
\usepackage{proof,tikz}
\usepackage{amssymb,amsthm,stmaryrd}


\usepackage[edges]{forest}
\usetikzlibrary{automata, positioning, arrows}


\newtheorem{Lemma}{Lema}
\newtheorem{Theorem}{Teorema}
\theoremstyle{definition}
\newtheorem{Example}{Exemplo}
\newtheorem{Definition}{Definição}


\usepackage{fancyhdr}
  \pagestyle{fancy}
  \fancyhf{}
  \lhead{Teoria da Computação}
  \rhead{Aula 3}
  \lfoot{Prof. Rodrigo Ribeiro}
  \rfoot{\thepage}
  \renewcommand{\footrulewidth}{0.4pt}
  \pagestyle{fancy}

\tikzset{
        ->,  % makes the edges directed
        >=stealth', % makes the arrow heads bold
        node distance=3cm,
        every state/.style={thick, fill=gray!10},
        initial text=$\,$
        }
  

\begin{document}

  \title{Aula 3 - Minimização de AFDs}
  \author{Rodrigo Ribeiro}

  \maketitle


  \pagestyle{fancy}


  \section*{Objetivos}

  \begin{itemize}
     \item Apresentar o conceito de equivalência entre estados de um AFD.
     \item Apresentar a construção do AFD mínimo equivalente. 
     \item Demonstrar que o AFD produzido pela minimização é equivalente ao
           AFD original.
  \end{itemize}

  \section{Introdução}

  \begin{Example}[Dois AFDs equivalentes]
    Considere a seguinte linguagem $L\subseteq \{0,1\}^*$:
    \[
      L = \{w \in\{0,1\}^*\,|\, \exists k. |w| = 2k + 1\}
    \]
    isto é, palavras de tamanho ímpar. Abaixo, apresentamos
    dois AFDs que aceitam essa linguagem.
    \begin{figure}[ht]
      \begin{minipage}{0.45\textwidth}
        \begin{tikzpicture}
          \node[state,initial](p){$P$} ;
          \node[state, accepting, right of=p] (i){$I$} ;
          
          \draw (p) edge[bend left, above] node{$0,1$} (i)
                (i) edge[bend left, below] node{$0,1$} (p) ;
        \end{tikzpicture}
      \end{minipage}\hfill
      \begin{minipage}{0.45\textwidth}
      \begin{tikzpicture}[node distance=2cm]
        \node[state, initial]                (pp){$A$} ;
        \node[state, accepting, right of=pp] (pi){$B$} ;
        \node[state,accepting,  below of=pp] (ip){$C$} ;
        \node[state,  below of=pi] (ii){$D$} ;

        \draw (pp) edge[bend left, above]     node{$1$} (pi)
              (pi) edge[bend left, below]     node{$1$} (pp)
              (pp) edge[bend left, left]      node{$0$} (ip)
              (ip) edge[bend left, left]      node{$0$} (pp)
              (pi) edge[bend left, left]     node{$0$} (ii)
              (ii) edge[bend left, left]     node{$0$} (pi)
              (ip) edge[bend left, above]    node{$1$} (ii)
              (ii) edge[bend left, below]    node{$1$} (ip) ; 
      \end{tikzpicture}        
      \end{minipage}
    \end{figure}

    O objetivo da aula de hoje é mostrar que dados dois ou mais
    AFDs equivalentes é possível construir um AFD mínimo (isto é,
    com o menor número possível de estados) equivalente. Além disso,
    esse AFD mínimo é único, a menos de isomorfismo de grafos.
  \end{Example}

  \section{Construção do AFD mínimo}

  \begin{Definition}[Equivalência de estados]
    Seja $M = (E,\Sigma,\delta,i,F)$ um AFD. Dizemos que dois estados
    $e_1,e_2 \in E$ são equivalentes, $e_1 \approx e_2$, se a seguinte
    fórmula é verdadeira:
    \[
      \forall w. w\in\Sigma^* \to \widehat{\delta}(e_1,w) \in F \leftrightarrow
      \widehat{\delta}(e_2,w) \in F
    \]
    A equivalência de estados é uma relação de equivalência: ela é reflexiva,
    transitiva e simétrica. Logo, essa introduz uma partição sobre o conjunto de
    estados. O AFD mínimo equivalente consiste em considerar como novo conjunto
    estados o conjunto de classes de equivalência induzida pela relação $\approx$.
  \end{Definition}

  \begin{Definition}[Autômato finito mínimo equivalente].\label{afdmin}
    Seja $M = (E,\Sigma,\delta,i,F)$ um AFD. O AFD mínimo equivalente a M é
    $M' = (E',\Sigma,\delta',i', F')$, em que:
    \begin{itemize}
      \item $E' = \{[e]_\approx \,|\, e \in E\}$, isto é, o conjunto de classes
        de equivalência induzidas por $\approx$ em $E$.
      \item $\delta'([e]_\approx,a) = [\delta(e,a)]_\approx$.
      \item $i' = [i]_\approx$.
      \item $F' = \{[e]_\approx\,|\,e\in F\}$.
    \end{itemize}
    Visto a definição acima, temos que o problema central em minimizar um AFD é
    encontrar a partição de seu conjunto de estados.
  \end{Definition}

  \subsection{Correção da construção do AFD mínimo}
  
  \begin{Lemma}
    Se $e_1 \approx e_2$, então $\delta(e_1,a) \approx \delta(e_2, a)$.
  \end{Lemma}
  \begin{proof}
    Suponha que $e_1\approx e_2$.Suponha $a\in \Sigma$, $y\in\Sigma^*$
    arbitrários. Temos que:
    \[
      \begin{array}{lcl}
        \widehat{\delta}(\delta(e_1,a),y)\in F & \leftrightarrow \\
        \widehat{\delta}(e_1,ay) \in F         & \leftrightarrow & \{ e_1 \equiv
                                                                   e_2\} \\
        \widehat{\delta}(e_2,ay) \in F         & \leftrightarrow \\
        \widehat{\delta}(\delta(e_2,a),y)\in F \\
        
      \end{array}
    \]
    Como $y \in \Sigma^*$ é arbitrário, temos que $[\delta(e_1,a)] \approx [\delta(e_2,a)]$.
  \end{proof}

  \begin{Lemma}\label{lemma136}
    $e \in F$ se e somente se $[e]_\approx \in F'$.
  \end{Lemma}
  \begin{proof}$\,\,\,$\\
    $(\to)$ Imediato pela definição de $F'$. \\
    $(\leftarrow)$ Suponha que $[e]_\approx \in F'$. Como $[e]_\approx \in F'$, pela definição
    de $F'$, existe $e' \in F$ tal que $e \approx e'$. Uma vez que $e \approx
    e'$ e $e' \in F$, temos que $e \in F$.
  \end{proof}

  \begin{Lemma}\label{lemma137}
    Para todo $w \in \Sigma^*$, $e\in E$, temos que
    $\widehat{\delta'}([e],w) = [\widehat{\delta}(e,w)]$
  \end{Lemma}
  \begin{proof}
    Suponha $w\in\Sigma^*$ arbitrário. Procederemos por indução sobre $w$.
    \begin{itemize}
    \item Caso base: Para $w = \lambda$, temos:
      \[
        \begin{array}{lc}
          \widehat{\delta'}([e],w) & = \\
         \lbrack e \rbrack & = \\
          \lbrack \widehat{\delta}(e,\lambda) \rbrack
        \end{array}
      \]
     \item Passo indutivo. Suponha que para todo $e\in E$, $\widehat{\delta'}([e],w) =
       [\widehat{\delta}(e,w)]$. Suponha $a \in \Sigma$ arbitrário. Temos:
       \[
         \begin{array}{lcl}
           \widehat{\delta'}([e],aw) & = & \{\textit{def. de }\widehat{\delta}.\}\\
           \widehat{\delta'}(\delta'([e],a),w) & = &  \{\textit{def. de } \delta'\}\\
           \lbrack\widehat{\delta'}([\delta(e,a)],w)\rbrack & = & \{H.I.\}\\ 
           \lbrack\widehat{\delta}(\delta(e,a),w)\rbrack  & = & \{\textit{def. de }\widehat{\delta}.\}\\
           \lbrack\widehat{\delta}(e,aw)\rbrack \\
         \end{array}
       \]
    \end{itemize}
  \end{proof}

  \begin{Theorem}[Correção da construção de AFD mínimo]
    Seja $M$ um AFD qualquer e $M'$ o seu AFD mínimo construído usando a
    Definição \ref{afdmin}. Temos que $L(M') = L(M)$.
  \end{Theorem}
  \begin{proof}
    Suponha $w\in \Sigma^*$ arbitrário. Temos:
    \[
      \begin{array}{lcl}
        w \in L(M')                   & \leftrightarrow & \{\textit{definição de L(M')}\}\\
        \widehat{\delta}(i',w) \in F' & \leftrightarrow & \{\textit{definição de i'}\} \\
        \widehat{\delta}([i],w)\in F'  & \leftrightarrow & \{\textit{Lema }\ref{lemma137}\}\\
        \lbrack\widehat{\delta}(i,w)\rbrack \in F' & \leftrightarrow & \{\textit{Lema }\ref{lemma136}\}\\
        \lbrack\widehat{\delta}(i,w)\rbrack \in F  & \leftrightarrow &
                                                                       \{\textit{definição
                                                                       de L(M)}\}\\
        w \in L(M)  
      \end{array}
    \]
  \end{proof}

  \subsection{Construção da partição de estados}

  \begin{Definition}[Estados $i$-equivalentes]\label{iequiv}
    Seja $M = (E,\Sigma,\delta, i, F)$ um  AFD. A relação $\approx_i$
    (i-equivalência) é definida como:
    \begin{itemize}
      \item $e \approx_0 e'$ se, e somente se, $e,e' \in F \lor e,e' \in (E - F)$.
      \item $e \approx_{n + 1} e'$ se, e somente se, $e \approx_n e'$ e, para
        todo $a\in \Sigma$ temos que $\delta(e,a)\approx_n\delta(e',a)$.
      \end{itemize}
  \end{Definition}

  \begin{Definition}{Partições sucessivas}
    Seja $M = (E,\Sigma,\delta, i, F)$ um  AFD. Definimos as partições
    sucessivas de $E$ com respeito a $\approx_i$ como:
    \[
      \begin{array}{lcl}
        [e]_0 & = & \left\{
                    \begin{array}{ll}
                      F & \textit{se }e\in F \\
                      E - F & \textit{caso contrário} \\
                    \end{array}
        \right. \\ \\
        \lbrack e\rbrack_{n+1} & = & \{e' \in [e]_n\,|\,\forall a. a \in \Sigma \to
                        \lbrack\delta(e',a)\rbrack = \lbrack\delta(e,a)\rbrack\}\\
      \end{array}
    \]
    A construção iterativa da partição continua até que $[e]_n = [e]_{n + 1}$.
  \end{Definition}

  \begin{Example}
    Para exemplificar as diversas definições apresentadas, vamos considerar o
    AFD apresentado no início destas notas.
    \begin{figure}[H]
      \centering
      \begin{tikzpicture}[node distance=2cm]
        \node[state, initial]                (pp){$A$} ;
        \node[state, accepting, right of=pp] (pi){$B$} ;
        \node[state,accepting,  below of=pp] (ip){$C$} ;
        \node[state,  below of=pi] (ii){$D$} ;

        \draw (pp) edge[bend left, above]     node{$1$} (pi)
              (pi) edge[bend left, below]     node{$1$} (pp)
              (pp) edge[bend left, left]      node{$0$} (ip)
              (ip) edge[bend left, left]      node{$0$} (pp)
              (pi) edge[bend left, left]     node{$0$} (ii)
              (ii) edge[bend left, left]     node{$0$} (pi)
              (ip) edge[bend left, above]    node{$1$} (ii)
              (ii) edge[bend left, below]    node{$1$} (ip) ; 
      \end{tikzpicture}              
    \end{figure}

    Para encontrar o AFD mínimo equivalente ao acima apresentado, devemos
    construir a partição de seu conjunto de estados. Para isso, utilizaremos
    uma tabela para armazenar a configuração intermediária da partição.
    No primeiro momento, apresentamos a partição para $n = 0$.
    \begin{figure}[H]
      \begin{tabular}{|c|l|}
        \hline
        $n$ & Partição \\ \hline
        0   & $\{A,D\}\{B,C\}$ \\\hline
      \end{tabular}
      \centering
    \end{figure}
    Note que na primeira iteração, dividimos o conjunto de estados em dois
    subconjuntos contendo os estados finais e não finais. Nessa etapa,
    podemos afirmar que $A \approx_0 D$ e que $B\approx_0 C$.

    Para a iteração 1, devemos verificar se $A \approx_1 D$. Para isso, vamos
    recorrer a Definição \ref{iequiv}, que especifica:
    \[
      A \approx_1 D \leftrightarrow A\approx_0 D \land \forall a \in \Sigma.
      \delta(A,a) \approx_0 \delta(D,a).
    \]
    Isto é, para $A$ e $D$ serem $1$-equivalentes eles devem: 1) ser
    0-equivalentes e; 2) o resultado de toda transição saindo desses estados
    também deve ser 0-equivalente. Uma vez que $A \approx_0 B$, basta verificar
    se o resultado de cada uma das transições também é 0-equivalente.

    Para $a = 0$, temos que $\delta(A,0) = C \approx_0 B = \delta(D,0)$. Por sua
    vez, para $a = 1$, temos que $\delta(A,1) = B \approx_0 C = \delta(D,1)$.
    Dessa forma, temos que $A \approx_1 D$.

    Seguindo um raciocínio similar, podemos constatar que $B \approx_1 C$. Logo,
    chegamos no seguinte resultado.
    \begin{figure}[H]
      \begin{tabular}{|c|l|}
        \hline
        $n$ & Partição \\ \hline
        0   & $\{A,D\}\{B,C\}$ \\\hline
        1   & $\{A,D\}\{B,C\}$ \\\hline        
      \end{tabular}
      \centering
    \end{figure}
    Como $[e]_n = [e]_{n + 1}$, temos que a partição de estados para o AFD é
    $\{\{A,D\},\{B,C\}\}$.

    Finalmente, aplicando a construção da Definição \ref{afdmin}, obtemos o
    seguinte AFD:
    \begin{figure}[H]
      \begin{tikzpicture}
          \node[state,initial](p){$\{A,D\}$} ;
          \node[state, accepting, right of=p] (i){$\{B,C\}$} ;
          
          \draw (p) edge[bend left, above] node{$0,1$} (i)
                (i) edge[bend left, below] node{$0,1$} (p) ;
      \end{tikzpicture}
      \centering
    \end{figure}
  \end{Example}

  \begin{Example}
    Nesse exemplo vamos construir o AFD mínimo equivalente ao seguinte AFD:
    \begin{figure}[H]
      \begin{tikzpicture}[node distance=2cm]
        \node[state,initial](s0){$\lambda$};
        \node[state,accepting, below right of=s0](s1){$c$};
        \node[state, above right of=s0](e){$e$} ;

        \draw (s0) edge [above] node{$0$} (s1)
              (s0) edge [above] node{$1$} (e)
              (s1) edge [loop right] node{$0,1$} (s1)
              (e) edge [loop right] node{$0,1$} (e)  ;            
      \end{tikzpicture}
      \centering
    \end{figure}
    Construindo a partição para $n = 0$.
    \begin{figure}[H]
      \begin{tabular}{|c|l|}
        \hline
        $n$ & Partição \\ \hline
        0   & $\{c\}\{\lambda, e\}$ \\\hline
      \end{tabular}
      \centering
    \end{figure}
    Note que a única possibilidade de evolução dessa partição é pela verificação
    de $\lambda \approx_1 e$. Porém, veja que $\delta(\lambda,0) = C
    \not\approx_0 e = \delta(e,0)$. Dessa forma, temos que $\lambda
    \not\approx_1 e$ e, portanto, esses estados devem estar em conjuntos
    diferentes na partição.
    \begin{figure}[H]
      \begin{tabular}{|c|l|}
        \hline
        $n$ & Partição \\ \hline
        0   & $\{c\}\{\lambda, e\}$ \\\hline
        1   & $\{c\}\{\lambda\}\{e\}$ \\ \hline
      \end{tabular}
      \centering
    \end{figure}
    Observe que na iteração $1$ obtivemos uma partição que é formada
    apenas por conjuntos unitários. O que esse fato quer dizer? Observe que
    cada conjunto de estados de uma partição contém todos os estados
    equivalentes entre si. Se a partição é formada apenas por conjuntos
    unitários, isso evidencia que o AFD em questão já é o mínimo.
  \end{Example}

  \begin{Example}
    Construa o AFD mínimo equivalente ao AFD apresentado a seguir.
    \begin{figure}[H]
      \begin{tikzpicture}[node distance=2cm]
        \node[state, initial](s0){$0$} ;
        \node[state, above right of=s0] (s1){$1$} ;
        \node[state, below right of=s0] (s2){$2$} ;
        \node[state, accepting, right of=s1] (s3) {$3$} ;
        \node[state, accepting, right of=s2] (s4) {$4$} ;
        \node[state, accepting, right=3.75 of s0](s5) {$5$} ;

        \draw (s0) edge[above] node{$a$} (s1)
              (s0) edge[above] node{$b$} (s2)
              (s1) edge[above] node{$a$} (s3)
              (s1) edge[right] node{$b$} (s4)
              (s2) edge[above] node{$a$} (s4)
              (s2) edge[left]  node{$b$} (s3)
              (s3) edge[left] node{$a,b$} (s5)
              (s4) edge[left] node{$a,b$} (s5)
              (s5) edge[loop right] node{$a,b$} (s5) ;
      \end{tikzpicture}
      \centering
    \end{figure}
    Temos a seguinte partição inicial dos estados:
    \begin{figure}[H]
      \begin{tabular}{|c|l|}
        \hline
        $n$ & Partição \\ \hline
        0   & $\{0,1,2\}\{3,4,5\}$ \\\hline
      \end{tabular}
      \centering
    \end{figure}
    Como temos mais estados nos conjuntos, temos mais possibilidades para o
    mesmo conjunto. Por exemplo, temos que verificar, no primeiro conjunto, se
    $0 \approx_1 1$ e $1\approx_1 2$.
    
    Primeiramente, para $0 \approx_1 1$ temos que $\delta(0,a) = 1 \not\approx_0
    3 = \delta(1,a)$. Logo, $0\not\approx_1 1$ e, portanto, esses dois estados
    não devem ficar no mesmo conjunto. Por sua vez, temos que $1 \approx_1 2$,
    uma vez que $\delta(1,a) = 3 \approx_0 4 = \delta(2,a)$ e
    $\delta(1,b) = 4 \approx_0 3 = \delta(2,b)$. Dessa forma, temos que o estado
    $0$ deve ser removido desse conjunto. Com isso, temos a seguinte partição
    intermediária. Omitiremos a verificação de que os estados $3,4$ e $5$ são
    equivalentes, pois essa será repetida no próximo passo.
    \begin{figure}[H]
      \begin{tabular}{|c|l|}
        \hline
        $n$ & Partição \\ \hline
        0   & $\{0,1,2\}\{3,4,5\}$ \\\hline
        1   & $\{0\},\{1,2\},\{3,4,5\}$ \\ \hline
      \end{tabular}
      \centering
    \end{figure}
    Para construir a partição da iteração 2, pode-se ver que os estados $1$ e
    $2$ são equivalantes. Logo, para finalizar a partição basta verificar se
    os estados $3,4$ e $5$ são equivalentes. Note que os estados 3 e 4 são
    equivalentes pois ambos tem transições apenas para o estado 5. Logo,
    $3 \approx_2 4$. Da mesma forma, temos que $3 \approx_2 5$ pois o 5 tem
    apenas um ciclo. Logo os estados 3, 4 e 5 são equivalentes. Com isso, temos
    a seguinte partição final.
    \begin{figure}[H]
      \begin{tabular}{|c|l|}
        \hline
        $n$ & Partição \\ \hline
        0   & $\{0,1,2\}\{3,4,5\}$ \\\hline
        1   & $\{0\},\{1,2\},\{3,4,5\}$ \\ \hline
        2   & $\{0\},\{1,2\},\{3,4,5\}$ \\ \hline
      \end{tabular}
      \centering
    \end{figure}
    A seguir, apresentamos a construção do AFD mínimo usando a partição obtida.
    \begin{figure}[H]
      \begin{tikzpicture}[node distance=3cm]
        \node[state, initial](s0){$0$} ;
        \node[state, right of=s0](s1){$\{1,2\}$} ;
        \node[state, accepting, right of=s1](s2){$\{3,4,5\}$} ;
        \draw (s0) edge[above] node{$a,b$} (s1)
              (s1) edge[above] node{$a,b$} (s2)
              (s2) edge[loop right] node{$a,b$} (s2) ;
      \end{tikzpicture}
      \centering
    \end{figure}
  \end{Example}
  
  \section{Exercícios}

  \begin{enumerate}
    \item Apresente AFDs para as seguintes linguagens sobre $\Sigma = \{0,1\}$ e
      o respectivos AFDs mínimos.
      \begin{enumerate}
        \item Palavras cujo tamanho é múltiplo de 4.
        \item Palavras que terminam em 11.
        \item Palavras que não possuem nenhuma ocorrência de 000.
        \item Palavras que possuem um número par de 0's.
        \item Palavras que possuem um número par de 0's e ímpar de 1's.
      \end{enumerate}
    \item Prove que $\approx$ é uma relação de equivalência.
    \item O Prof. Mipha Laram argumenta que pode-se obter AFDs mais eficientes
      por executar o processo de minimização de AFDs uma segunda vez. Mostre,
      utilizando as definições apresentadas nesta aula, que o argumento do Prof.
      Mipha Laram é incorreto.
  \end{enumerate}
\end{document}
