\documentclass[a4paper]{article}

\usepackage[portuguese]{babel}
\usepackage[utf8]{inputenc}
\usepackage{graphicx,hyperref}
\usepackage{float}
\usepackage{listings}
\usepackage{proof,tikz}
\usepackage{amssymb,amsthm,stmaryrd}


\usepackage[edges]{forest}
\usetikzlibrary{automata, positioning, arrows}


\newtheorem{Lemma}{Lema}
\newtheorem{Theorem}{Teorema}
\theoremstyle{definition}
\newtheorem{Example}{Exemplo}
\newtheorem{Definition}{Definição}


\usepackage{fancyhdr}
  \pagestyle{fancy}
  \fancyhf{}
  \lhead{Teoria da Computação}
  \rhead{Aula 18}
  \lfoot{Prof. Rodrigo Ribeiro}
  \rfoot{\thepage}
  \renewcommand{\footrulewidth}{0.4pt}
  \pagestyle{fancy}

\tikzset{
        ->,  % makes the edges directed
        >=stealth', % makes the arrow heads bold
        node distance=3cm,
        every state/.style={thick, fill=gray!10},
        initial text=$\,$
        }
  

\begin{document}

\title{Aula 18 - Introdução às Máquinas de Turing}
  \author{Rodrigo Ribeiro}

  \maketitle


  \pagestyle{fancy}


  \section*{Objetivos}

  \begin{itemize}
     \item Apresentar o conceito de máquina de Turing.
     \item Apresentar os conceitos de linguagens recursivas
       e recursivamente enumeráveis.
     \item Apresentar os conceitos de aceitação por parada em
           estado final e por parada.
  \end{itemize}

  \section{Introdução}

  Conforme apresentado em aulas passadas, linguagens regulares e livres de
  contexto não são capazes de representar a estrutura de linguagens muito
  simples, como por exemplo, $\{0^n1^n2^n\,|\,n\geq 0\}$. Nesta aula
  apresentaremos as máquinas de Turing (MT), um modelo matemático de máquina
  proposta por Alan Turing na década de 30. As MTs são tão
  expressivas, que até hoje não se conhece um formalismo mais poderoso.

  Ao contrário das máquinas que vimos até o presente momento, a MT
  permite sobreescrever a palavra de entrada. Esse fato aliado a
  uma memória infinita proporciona maior expressividade que os autômatos de
  pilha. Em MTs consideramos que o próximo símbolo a ser
  processado é o que está sobre o \emph{cabeçote}, que pode mover para a
  direita ou esquerda durante transições. O conteúdo da fita de uma MT é
  da forma $\langle w \sqcup^*$, em que $\langle$ é o marcador de início de fita
  e $\sqcup$ é o símbolo de ``branco'', que preenche todas as posições da fita
  de uma MT após o fim da palavra de entrada. Note que se a MT inicia com a
  palavra $\lambda$ ela será da forma $\langle \sqcup^*$.

  Abaixo, apresentamos o formato geral de uma transição em uma máquina de Turing:
  \begin{figure}[H]
    \begin{tikzpicture}[node distance=3cm]
      \node[state](s0){$A$};
      \node[state, right of=s0](s1){$B$};
      \draw (s0)edge[above]node{$a/b\,d$}(s1);
    \end{tikzpicture}
    \centering
  \end{figure}
  A transição acima pode ser descrita formalmente como $\delta(A,a) = [B,b,d]$,
  em que $d \in \{D,E\}$ e, pode ser entendida como: a máquina estando no estado
  $A$ com símbolo sobre o cabeçote $a$, escreve $b$ na posição atual e move o
  cabeçote na direção $d$.

  A MT utiliza uma fita, infinita à direita, possuindo células capazes de
  armazenar um símbolo de alfabeto cada. No próximo exemplo, apresentaremos
  uma MT simples e ilustraremos como a computação funciona nesse tipo de
  máquina.

  \begin{Example}
    Considere o seguinte diagrama de estados para uma MT que calcula a operação
    de not bit-a-bit.
    \begin{figure}[H]
      \begin{tikzpicture}
        \node[state, initial](s0){$A$} ;
        \node[state, right of=s0](s1){$B$};
        \node[state, right of=s1](s2){$C$};
        \draw (s0) edge[loop above]node{$\begin{array}{c}0/1\:D\\1/0\:D\\\end{array}$}(s0)
              (s0) edge[above]node{$\sqcup/\sqcup\:E$}(s1)
              (s1) edge[loop above]node{$\begin{array}{c}0/0\:E\\1/1\:E\\\end{array}$}(s1)
              (s1) edge[above]node{$\langle / \langle\: D$} (s2);
      \end{tikzpicture}
      \centering
    \end{figure}
    Para exemplificar o funcionamento desta MT, vamos utilizar a noção de
    configuração instântanea, que consiste de um par formado pelo estado atual e
    pelo conteúdo da fita. Vamos mostrar o processamento desta MT sobre a
    palavra $0011$:
    \[
      \begin{array}{lcl}
        \lbrack A,\langle \underline{0}011 \rbrack & \vdash & \{\delta(A,0) =
                                                              [A,1,D]\}\\
        \lbrack A, \langle 1 \underline{0}11 \rbrack & \vdash & \{\delta(A,0) =
                                                                [A,1,D]\} \\
        \lbrack A, \langle  11 \underline{1}1 \rbrack & \vdash & \{\delta(A,1) =
                                                                 [A,0,D]\} \\
        \lbrack A, \langle  110\underline{1}\rbrack & \vdash & \{\delta(A,1) =
                                                               [A,0,D]\} \\
        \lbrack A , \langle 1100 \underline{\sqcup} \rbrack & \vdash &
                                                              \{\delta(A,\sqcup)
                                                              = [B,\sqcup, E]\}\\
        \lbrack B, \langle 110\underline{0} \rbrack & \vdash & \{\delta(B,0) =
                                                               [B,0, E]\} \\
        \lbrack B, \langle 11\underline{0}0 \rbrack & \vdash & \{\delta(B,0) =
                                                               [B,0, E]\} \\
        \lbrack B, \langle 1\underline{1}00 \rbrack & \vdash & \{\delta(B,1) =
                                                               [B,1, E]\} \\
        \lbrack B, \langle \underline{1}100 \rbrack & \vdash & \{\delta(B,1) =
                                                               [B,1, E]\} \\
        \lbrack B, \langle \underline{\langle}1100 \rbrack & \vdash & \{\delta(B,\langle) =
                                                               [C,\langle, D]\}
        \\
        \lbrack C , \langle  \underline{1}100 \rbrack & &
      \end{array}
    \]
    Com isso, a MT pára contendo $1100$ que é o resultado do not bit-a-bit de
    $0011$, que era a palavra original de entrada. Note que, como o objetivo
    desta MT não é a aceitação de uma palavra, não faz sentido possuir estados
    finais. O próximo exemplo ilustra uma MT que aceita uma linguagem.
 \end{Example}

 \begin{Example}
   A seguinte MT aceita a linguagem regular $0(0+1)^*$:
    \begin{figure}[H]
      \begin{tikzpicture}
        \node[state, initial](s0){$A$} ;
        \node[state, accepting, right of=s0](s1){$B$};
        \draw (s0) edge[above]node{$0/ 0\:D$}(s1) ;
      \end{tikzpicture}
      \centering
    \end{figure}
    Essa MT apresenta uma característica importante: uma MT não precisa processar
    uma palavra inteira para aceitar uma palavra. Note que se a palavra não inicia
    com o símbolo $0$, a MT permanece no estado inicial $A$, que é não final.
    Caso inicie com $0$, a MT faz a transição para o estado $B$ e pára (pois
    este não possui transições) em estado final.
    Dizemos que uma MT aceita uma palavra por parada em estado final se o
    processamento de uma palavra faz com que a MT pare em um estado final.

    Considere, agora outra MT que aceita a mesma linguagem usando outro critério
    de aceitação:
    \begin{figure}[H]
      \begin{tikzpicture}
        \node[state, initial](s0){$A$} ;
        \node[state, right of=s0](s1){$B$};
        \draw (s0) edge[above, bend left]node{$\begin{array}{l}1 / 1 \: D\\ \sqcup / \sqcup\: D \end{array}$}(s1)
              (s1) edge[below, bend left]node{$\begin{array}{l}1 / 1 \: E\\ 0/ 0\: E\\ \sqcup / \sqcup\: E \end{array}$} (s0) ;
      \end{tikzpicture}
      \centering
    \end{figure}

    Observe que essa MT entra em loop para qualquer palavra que não inicia com
    $0$. Dizemos que uma MT aceita uma palavra por parada se o processamento da
    MT pára em algum estado e a recusa se a MT entra em loop. Logo, a MT
    anterior, aceita $0(0+1)^*$ por parada.
 \end{Example}

 \section{Máquinas de Turing}

 Segue a definição formal de uma MT.
 
 \begin{Definition}[Máquina de Turing]
   Uma MT $M = (E,\Sigma,\Gamma,\langle, \sqcup, \delta, i, F)$ é uma óctupla,
   em que:
   \begin{itemize}
      \item $E$: conjunto de estados.
      \item $\Sigma$: alfabeto de entrada.
      \item $\Gamma$: alfabeto de fita.
      \item $\langle \in \Gamma - (\{\sqcup\} \cup \Sigma)$: marcador de início
        de fita.
      \item $\sqcup \in \Gamma - (\{\langle\} \cup \Sigma)$: símbolo de branco.
      \item $\delta : E \times \Gamma \to E \times \Gamma \times \{E,D\}$,
        função de transição. Uma função parcial.
      \item $i \in E$: estado inicial.
      \item $F\subseteq E$: conjunto de estados finais.
    \end{itemize}
    Uma MT que segue essa definição é dita ser uma MT padrão. 
 \end{Definition}

 \begin{Example}
   Para ilustrar a definição acima, vamos considerar novamente a MT que aceita
   $0(0+1)^*$:
    \begin{figure}[H]
      \begin{tikzpicture}
        \node[state, initial](s0){$A$} ;
        \node[state, accepting, right of=s0](s1){$B$};
        \draw (s0) edge[above]node{$0/ 0\:D$}(s1) ;
      \end{tikzpicture}
      \centering
    \end{figure}
    Temos que:
    \begin{itemize}
      \item $E =\{A,B\}$
      \item $\Sigma =\{0,1\}$
      \item $\Gamma = \{\langle, \sqcup, 0, 1\}$
      \item A função de transição é formada por uma única equação $\delta(A,0) =
        [B,0,D]$.
      \item $i = A$
      \item $F =\{B\}$
    \end{itemize}   
 \end{Example}

 \section{Linguagem aceita por MT padrão}
 
  Para definir a linguagem aceita por uma MT, é necessário uma função que
  elimina os símbolos de branco presentes à direita da palavra de entrada na
  fita. A função $\pi : \Gamma^* \to \Gamma^*$ é definida como:
  \[
    \pi(w) =\left\{  \begin{array}{ll}
                       \lambda & \textit{se } w \in \{\sqcup\}^*\\
                       xa      & \textit{se } w=xay \land a \neq \sqcup \land y \in \{\sqcup\}^*\\
                     \end{array}
             \right.
  \]

  
  Usando a função $\pi$, podemos definir a relação de transição entre
  configurações instantâneas de uma MT.

  \begin{Definition}[Transição entre configurações intantâneas]
    Seja $M = (E,\Sigma,\Gamma, \langle, \sqcup, \delta, i, F)$ uma MT. A
    relação $\vdash \subseteq (E \times \Gamma^+)^2$, para $M$, é tal que
    para todo $e \in E$ e $a \in \Gamma$:
    \begin{enumerate}
      \item Se $\delta(e,a) = [e',b,D]$ então $[e,x\underline{a}cy] \vdash
        [e',xb\underline{c}y]$ para $c \in \Gamma$ e,
        $[e,x\underline{a}]\vdash[e',xb\underline{\sqcup}]$;
      \item Se $\delta(e,a) = [e',b,E]$ então $[e,xc\underline{a}y] \vdash [e',
        x\underline{c}\pi(by)]$ para $c \in \Gamma$ ;
      \item Se $\delta(e,a)$ é indefinido então não existe $f$ tal eque
        $[e,x\underline{a}y] \vdash f$.
    \end{enumerate}
    Como usual, $\vdash^*$ denotará o fecho reflexivo e transitivo de $\vdash$.  
  \end{Definition}

  \begin{Definition}[Linguagem aceita por parada em estado final]
    Seja $M = (E,\Sigma,\Gamma,\langle, \sqcup, \delta, i, F)$ um MT. A
    linguagem aceita por $M$ por parada em estado final é:
    \[
      L(M) = \{w \in \Sigma^*\,|\,[i,\langle w] \vdash^* [e', x\underline{a}y],
      \delta(e,a) = \bot \land e' \in F\}
    \]
    A notação $\delta(e,a) = \bot$ representa que a transição $\delta(e,a)$ é
    indefinida, isto é, que a MT pára.
  \end{Definition}

  \begin{Definition}[Linguagem recursivamente enumerável]
    Dizemos que $L \subseteq \Sigma^*$ é uma linguagem recursivamente enumerável (LRE)
    se existe uma MT que a reconhece.
  \end{Definition}

  \begin{Definition}[Linguagem recursiva]
    Dizemos que $L \subseteq \Sigma^*$ é uma linguagem recursiva (LRec) se
    existe uma MT que a reconhece e pare para todas as palavras do alfabeto de entrada.
  \end{Definition}

  \begin{Definition}[Linguagem aceita por estado final]
    Seja $M = (E,\Sigma,\Gamma,\langle, \sqcup, \delta, i, F)$ um MT. A
    linguagem aceita por $M$ por estado final é:
    \[
      L_F(M) = \{w \in \Sigma^*\,|\,[i,\langle w] \vdash^* [e', x\underline{a}y]
      \land e' \in F\}
    \]
    Note que nesse critério de reconhecimento basta existir um único estado
    final e este não deve possuir transições.
  \end{Definition}

  \begin{Definition}[Linguagem aceita por parada]
    Seja $M = (E,\Sigma,\Gamma,\langle, \sqcup, \delta, i)$ um MT. A
    linguagem aceita por $M$ por parada é:
    \[
      L_P(M) = \{w \in \Sigma^*\,|\,[i,\langle w] \vdash^* [e', x\underline{a}y],
      \delta(e,a) = \bot \}
    \]
  \end{Definition}

  Todos esses 3 critérios de aceitação são equivalentes entre si, conforme
  mostrado pelo teorema a seguir.

  \begin{Theorem}
    Seja $L$ uma linguagem. As seguintes afirmativas
    são equivalentes:
    \begin{enumerate}
       \item $L$ é uma linguagem recursivamente enumerável.
       \item $L$ pode ser reconhecida por uma MT que aceita por estado final.
       \item $L$ pode ser reconhecida por uma MT que aceita por parada.
    \end{enumerate}
  \end{Theorem}
  \begin{proof}$\:$\\
    $1 \to 2)$: Suponha que $L$ é uma LRE. Se $L$ é uma LRE, existe uma MT
    $M = (E,\Sigma, \Gamma, \langle, \sqcup, \delta, i, F)$ tal que $L(M) = L$.
    Seja $f \not\in E$ um novo estado. A MT $M' = (E \cup \{f\}, \Sigma, \Gamma,
    \langle, \sqcup, \delta',i, \{f\})$, em que $\delta'$ é definida como:
    \begin{itemize}
      \item Se $\delta(e,a)$ é definido, então $\delta'(e,a) = \delta(e,a)$.
      \item Para todo $(e,a) \in F \times \Gamma$, se $\delta(e,a) = \bot$,
        então $\delta'(e,a) = [f, a, D]$.
      \item Para todo $(e,a) \in (E - F) \times \Gamma$, se $\delta(e,a) =
        \bot$, então $\delta'(e,a) = \bot$.
      \item Para todo $a \in \Gamma$, $\delta'(f, a) = \bot$.
    \end{itemize}
    $2 \to 3)$: Suponha que $L$ pode ser reconhecida por uma MT que aceita por
    estado final e seja $M = (E,\Sigma, \Gamma, \langle, \sqcup, \delta, i, F)$
    tal MT. A MT $M' = (E\cup \{l\},\Sigma,\Gamma,\langle, \sqcup, \delta',i)$,
    $l\not\in E$ é equivalente a $M$, e aceita por parada, em que $\delta'$ é
    definida como:
    \begin{itemize}
      \item para todo $(e,a) \in (E - F) \times \Gamma$, se $\delta(e,a)$ é
        definido, temos que $\delta'(e,a) = \delta(e,a)$. Se $\delta(e,a) =
        \bot$, então $\delta'(e,a) = [l,a,D]$.
      \item Para todo $a\in \Gamma$, $\delta'(l,a) = [l,a,D]$.
      \item Para todo $(e,a) \in F \times \Gamma$, $\delta'(e,a) = \bot$. 
    \end{itemize}
    $3 \to 1)$: Seja $M = (E,\Sigma, \Gamma, \langle, \sqcup, \delta, i)$ uma MT
    que aceita $L$ por parada. A MT $M' = (E,\Sigma, \Gamma, \langle, \sqcup,
    \delta, i, E)$ aceita $L$ por parada em estado final.
  \end{proof}
  
  \section{Exercícios}

  \begin{enumerate}
     \item Construa uma MT padrão que aceita a linguagem $\{0^{2n}\,|\,n\geq
       0\}$.
     \item Construa uma MT padrão que aceita a linguagem $\{a^nb^n\,|\,n\geq
       0\}$.
     \item Utilize o Teorema 1 para construir MTs que aceitem por parada as
       linguagens dos exercícios 1 e 2.
  \end{enumerate}
\end{document}
