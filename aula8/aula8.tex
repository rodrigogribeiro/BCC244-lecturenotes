\documentclass[a4paper]{article}

\usepackage[portuguese]{babel}
\usepackage[utf8]{inputenc}
\usepackage{graphicx,hyperref}
\usepackage{float}
\usepackage{listings}
\usepackage{proof,tikz}
\usepackage{amssymb,amsthm,stmaryrd}


\usepackage[edges]{forest}
\usetikzlibrary{automata, positioning, arrows}


\newtheorem{Lemma}{Lema}
\newtheorem{Theorem}{Teorema}
\theoremstyle{definition}
\newtheorem{Example}{Exemplo}
\newtheorem{Definition}{Definição}


\usepackage{fancyhdr}
  \pagestyle{fancy}
  \fancyhf{}
  \lhead{Teoria da Computação}
  \rhead{Aula 8}
  \lfoot{Prof. Rodrigo Ribeiro}
  \rfoot{\thepage}
  \renewcommand{\footrulewidth}{0.4pt}
  \pagestyle{fancy}

\tikzset{
        ->,  % makes the edges directed
        >=stealth', % makes the arrow heads bold
        node distance=2cm,
        every state/.style={thick, fill=gray!10},
        initial text=$\,$
        }
  

\begin{document}

  \title{Aula 8 - Expressões regulares}
  \author{Rodrigo Ribeiro}

  \maketitle


  \pagestyle{fancy}


  \section*{Objetivos}

  \begin{itemize}
     \item Apresentar expressões regulares, sua sintaxe e semântica.
     \item Apresentar a equivalência entre expressões regulares e
           autômatos finitos.
  \end{itemize}

  \section{Expressões regulares}

  \begin{Definition}[Sintaxe e semântica de expressões regulares]
    O conjunto de expressões regulares (ERs) sobre um alfabeto $\Sigma$ é
    definido recursivamente como:
    \begin{itemize}
       \item $\emptyset$ denota a linguagem regular $\emptyset$.
       \item $\lambda$ denota a linguagem regular $\{\lambda\}$.
       \item $a$, $a\in\Sigma$, denota a linguagem regular $\{a\}$.
       \item $e_1 + e_2$ denota a linguagem regular $L(e_1) \cup L(e_2)$.
       \item $e_1\,e_2$ denota a linguagem regular $L(e_1) \:L(e_2)$.
       \item $e^*$ denota a linguaegm regular $L(e)^*$.
    \end{itemize}
  \end{Definition}

  \begin{Example}
    Abaixo apresentamos alguns exemplos de expressões regulares e os respectivos
    conjunto que elas denotam.
    \begin{itemize}
       \item $\emptyset$ denota a linguagem $\emptyset$.
       \item $01$ denota a linguagem $\{0\}\{1\} = \{01\}$.
       \item $(0 + 1)1$ denota a linguagem $(\{0\} \cup \{1\})\{1\} =
         \{0,1\}\{1\} = \{01,11\}$.
       \item $0^*$ denota a linguagem $\{0\}^*$.
    \end{itemize}
  \end{Example}

  \section{Equivalência de ERs e AFs}

  \subsection{Construindo um AF a partir de uma ER}
  
  \begin{Theorem}\label{teorema1}
    Toda expressão regular denota uma linguagem regular.
  \end{Theorem}
  \begin{proof}
    Suponha $e$ uma expressão regular arbitrária. Para mostrar que $e$ denota
    uma linguagem regular, devemos mostrar como construir, a partir de $e$ um
    AF. Para isso, vamos proceder por indução sobre a estrutura de $e$.
    \begin{itemize}
      \item Caso $e = \emptyset$. O AF correspondente a $e = \emptyset$ é:
        \begin{figure}[H]
          \begin{tikzpicture}
            \node[state, initial](s0){$A$};
          \end{tikzpicture}
          \centering
        \end{figure}
      \item Caso $e = \lambda$. O AF correspondente a $e = \lambda$ é:
        \begin{figure}[H]
          \begin{tikzpicture}
            \node[state, initial, accepting](s0){$A$};
          \end{tikzpicture}
          \centering
        \end{figure}
      \item Caso $e = a$, $a\in\Sigma$. O AF correspondente é:
        \begin{figure}[H]
          \begin{tikzpicture}
            \node[state,initial](s0){$A$} ;
            \node[state,accepting, right of=s0](s1){$B$} ;
            \draw (s0) edge[above] node{$a$}(s1);
          \end{tikzpicture}
          \centering
        \end{figure}
      \item Caso $e = e_1 + e_2$. Pela hipótese de indução, temos que $M_1$ é o
        AF correspondente a $e_1$ e de maneira similar $M_2$ é o AF para $e_2$.
        O AF para $e$ é obtido pela construção de produto para união.
      \item Caso $e = e_1\,e_2$. Pela hipótese de indução, temos que $M_1$ é o
        AF correspondente a $e_1$ e de maneira similar $M_2$ é o AF para $e_2$.
        O AF para $e$ é obtido pela propriedade de fechamento para concatenação.
      \item Caso $e = e_1^*$. Pela hipótese de indução, temos que $M_1$ é o
        AF correspondente a $e_1$.
        O AF para $e$ é obtido pela propriedade de fechamento para o fecho de Kleene.
    \end{itemize}
  \end{proof}

  \subsection{Construindo uma ER a partir de um AF}

  Antes de se apresentar a construção de uma ER a partir de um AF, faz-se
  necessária a seguinte definição.

  \begin{Definition}
    Um diagrama ER sobre $\Sigma$ é um diagrama de estados em que as arestas são
    rotuladas com ER ao invés de símbolos de $\Sigma$.
  \end{Definition}

  \begin{Definition}[Eliminando estados de um diagrama ER]
    Pode-se eliminar estados de diagramas ER preservando a linguagem aceita por
    esses usando os seguintes esquemas.
    \begin{figure}[H]
      \begin{minipage}{.45\textwidth}
        \begin{tikzpicture}
          \node[state](s0){$0$} ;
          \node[state, right of=s0](s1) {$1$} ;
          \node[state, right of=s1](s2) {$2$} ;
          \draw (s0) edge[above] node{$e_1$} (s1)
                (s1) edge[loop above] node{$e_2$} (s1)
                (s1) edge[above] node{$e_3$} (s2) ;
        \end{tikzpicture}
      \end{minipage}
      \hfill
      \begin{minipage}{.45\textwidth}
        \begin{tikzpicture}
          \node[state](s0){$0$} ;
          \node[state, right of=s1](s2) {$2$} ;
          \draw (s0) edge[above] node{$e_1e_2^*e_3$} (s2) ;
        \end{tikzpicture}        
      \end{minipage}
      \centering
      \caption{Esquema de eliminação de estados.}
    \end{figure}

    Outro esquema de eliminação de estados é apreseentado a seguir.

    \begin{figure}[H]
      \begin{minipage}{.45\textwidth}
        \begin{tikzpicture}[node distance=3cm]
          \node[state](s0){$0$} ;
          \node[state, right of=s0](s1) {$1$} ;
          \draw (s0) edge[bend left, above] node{$e_1$} (s1)
                (s1) edge[loop above] node{$e_2$} (s1)
                (s1) edge[bend left, below]node{$e_3$}(s0);
        \end{tikzpicture}
      \end{minipage}
      \hfill
      \begin{minipage}{.45\textwidth}
        \begin{tikzpicture}
          \node[state](s0){$0$} ;
          \draw (s0) edge[loop above] node{$e_1e_2^*e_3$} (s0) ;
        \end{tikzpicture}        
      \end{minipage}
      \centering
      \caption{Esquema de eliminação de estados.}
    \end{figure}    
  \end{Definition}

  \begin{Definition}[Diagramas ER básicos]
    Dizemos que um diagrama ER é básico se este é formado apenas por um estado
    inicial e final. A partir de um diagrama ER básico é imediato obter a ER
    que o denota. Abaixo apresentamos os dois possíveis casos.
    \begin{figure}[H]
        \begin{tikzpicture}[node distance=3cm]
          \node[state, initial, accepting](s0){$0$} ;
          \draw (s0) edge[loop above] node{$e$} (s0) ;
        \end{tikzpicture}
        \centering
        \caption{Diagrama ER para $e^*$.}
      \end{figure}
      \begin{figure}[H]
        \begin{tikzpicture}
          \node[state, initial](s0){$0$} ;
          \node[state, accepting, right of=s0](s1){$1$} ; 
          \draw (s0) edge[loop above] node{$e_1$} (s0)
                (s0) edge[bend left, above] node{$e_2$} (s1)
                (s1) edge[loop above] node{$e_3$}(s1)
                (s1) edge[bend left, below]node{$e_4$}(s0);
         \end{tikzpicture}    
      \centering
      \caption{Diagrama ER para $e_1^*e_2(e_3 + e_4e_1^*e_2)^*$.}
    \end{figure} 
  \end{Definition}

  \begin{Theorem}[Construção de ER a partir de AFs]\label{teorema2}
    Toda linguagem regular é denotada por uma expressão regular.
  \end{Theorem}
  \begin{proof}
    Suponha $M = (E,\Sigma,\delta,i,\{f_1,...,f_k\})$ um AF para uma linguagem
    regular $L$. Pela definição de aceitação de um AF, temos que
    $L = L_1 \cup ... \cup L_k$, em que
    \[
      L_j = \{w \in \Sigma^*\,|\,\widehat{\delta}(i,w) = f_j\}
    \]
    isto é, cada $L_j$ denota a sub-linguagem cujas palavras
    terminam seu processamento em $f_j$. Dessa forma, podemos encontrar a
    ER correspondente a cada $f_j$ eliminando todos os estados, exceto $i$ e
    $f_j$. Após essa eliminação de estados, pode-se ober uma ER $e_j$ usando
    os diagramas ER básicos. De posse das ER para cada um dos estados $f_j$,
    basta combiná-las para formar a ER do AF como um todo. A ER correspondente a
    $M$ é $e = e_1 + ... + e_k$.
  \end{proof}

  \begin{Example}
    Para exemplificar a construção de uma ER a partir de um AF, considere o seguinte AF:
    \begin{figure}[H]
      \begin{tikzpicture}
        \node[state,initial](s0){$0$};
        \node[state, accepting, right of=s0](s1){$1$};
        \node[state, below of=s0](s2){$2$} ;
        \node[state, right of=s2](s3){$3$} ;
        \draw (s0)edge[bend left, right]node{$a$}(s2)
              (s1)edge[bend left, right]node{$a$}(s3)
              (s3)edge[bend left, left]node{$a$}(s1)        
              (s0)edge[above]node{$b$}(s1)
              (s2)edge[above]node{$b$}(s3)
              (s2)edge[bend left, left]node{$a$}(s0) ;
       
      \end{tikzpicture}
      \centering
    \end{figure}
    Removendo o estado 2, temos o seguinte diagrama ER.
     \begin{figure}[H]
      \begin{tikzpicture}
        \node[state,initial](s0){$0$};
        \node[state, accepting, right of=s0](s1){$1$};
        \node[state, right of=s2](s3){$3$} ;
        \draw (s0)edge[loop above]node{$aa$}(s0)
              (s0)edge[left]node{$ab$}(s3)
              (s1)edge[bend left, right]node{$a$}(s3)
              (s3)edge[bend left, left]node{$a$}(s1)        
              (s0)edge[above]node{$b$}(s1) ;       
      \end{tikzpicture}
      \centering
    \end{figure}
    Eliminando o estado 3 obtemos:
     \begin{figure}[H]
      \begin{tikzpicture}[node distance=3cm]
        \node[state,initial](s0){$0$};
        \node[state, accepting, right of=s0](s1){$1$};
        \draw (s0)edge[loop above]node{$aa$}(s0)
              (s1)edge[loop above]node{$aa$}(s1)
              (s0)edge[above]node{$b + aba$}(s1) ;       
      \end{tikzpicture}
      \centering
    \end{figure}
   A partir deste diagrama ER básico, obtemos a ER correspondente ao AF
   original: $(aa)^*(b + aba)(aa)^*$.
  \end{Example}
  
  \section{Exercícios}

  \begin{enumerate}
     \item Para cada linguagem abaixo, apresente um AFD para a mesma. A partir
       do AFD apresentado, construa a ER usando a técnica do
       Teorema~\ref{teorema2}.
       \begin{enumerate}
         \item Palavras que começam e terminam com 1.
         \item Palavras que começam e terminam com 1 e tem pelo menos um 0.
         \end{enumerate}
      \item Utilizando a construção do Teorema \ref{teorema1}, apresente AFs
        para as seguintes ERs.
        \begin{enumerate}
          \item $(ab)^*ac$
          \item $(ab)^*(ba)^*$
          \item $((aa + bb)^*cc)*$
        \end{enumerate}
   \end{enumerate}
\end{document}
