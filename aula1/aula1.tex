\documentclass[a4paper]{article}

\usepackage[portuguese]{babel}
\usepackage[utf8]{inputenc}
\usepackage{graphicx,hyperref}
\usepackage{float}
\usepackage{listings}
\usepackage{proof,tikz}
\usepackage{amssymb,amsthm,stmaryrd}


\usepackage[edges]{forest}


\newtheorem{Lemma}{Lema}
\newtheorem{Theorem}{Teorema}
\theoremstyle{definition}
\newtheorem{Example}{Exemplo}
\newtheorem{Definition}{Definição}


\usepackage{fancyhdr}
  \pagestyle{fancy}
  \fancyhf{}
  \lhead{Teoria da Computação}
  \rhead{Aula 1}
  \lfoot{Prof. Rodrigo Ribeiro}
  \rfoot{\thepage}
  \renewcommand{\footrulewidth}{0.4pt}
  \pagestyle{fancy}


\begin{document}

  \title{Aula 1 - Introdução às linguagens formais}
  \author{Rodrigo Ribeiro}

  \maketitle


  \pagestyle{fancy}


  \section*{Objetivos}

  \begin{itemize}
     \item Apresentar os conceitos centrais da teoria de
       linguagens: alfabeto, palavra, linguagem e
       problemas de decisão.
     \item Especificação de linguagens usando conjuntos.
  \end{itemize}

  \section{Conceitos básicos}

  \begin{Definition}[Alfabeto]
    Um alfabeto é qualquer conjunto finito não vazio. Usaremos a letra grega
    sigma, $\Sigma$, para representar um alfabeto qualquer.
  \end{Definition}

  \begin{Example}
    Dois alfabetos importantes são $\Sigma_1= \{1\}$ e $\Sigma_2 =\{0,1\}$. Caso
    desejamos lidar com texto, um possível alfabeto é o UNICODE.
  \end{Example}

  \begin{Definition}[Palavra]
    Uma palavra sobre um alfabeto $\Sigma$ é uma sequência
    finita formada por elementos de $\Sigma$.
  \end{Definition}

  \begin{Definition}[Tamanho de uma palavra, palavra vazia]
    Seja $w$ uma palavra qualquer sobre um alfabeto $\Sigma$. Denomina-se
    por tamanho de $w$, $|w|$, o número de símbolos de $w$. Em especial,
    existe apenas uma palavra de tamanho $0$, e esta é representada pela
    letra grega $\lambda$.
  \end{Definition}

  \begin{Example}
    São exemplos de palavras sobre $\Sigma=\{0,1\}$: $0,1,00,1101,\lambda$.
    São exemplos de palavras sobre $\Sigma =\{1\}$: $\lambda,1,11,...$.
    Note que $\lambda$ é uma palavra sobre qualquer alfabeto.
  \end{Example}

  \begin{Definition}[Repetição]
    Denotamos por $a^n$, em que $a \in \Sigma$, a palavra formada por
    $n$ ocorrências do símbolo $a$. Formalmente:
    \[
      \begin{array}{lcl}
        a^0      & = & \lambda \\
        a^{n + 1} & = & a^n a\\
      \end{array}
    \]
  \end{Definition}

  \begin{Example}
    Considere $\Sigma = \{0,1\}$. Temos que $0^3 = 000$, $1^4 = 1111$ e $1^0 = \lambda$.
  \end{Example}

  \begin{Definition}[Concatenação]
    Sejam $x$ e $y$ duas palavras sobre um alfabeto $\Sigma$. A concatenação de
    $x$ e $y$ é a simples justaposição de $y$ ao fim da palavra $x$ e é
    representada por $xy$.
  \end{Definition}
  \begin{Example}
    Considere $\Sigma = \{0,1\}$, $x = 001$, $y = 01$ e $z = 11$. Temos que:
    \begin{itemize}
      \item $xy = 00101$.
      \item $yx = 01001$.
      \item $\lambda x = x \lambda = x$.
      \item $(xy)z = 0010111$ e $x(yz) = 0010111$.
    \end{itemize}
    Note que a concatenação é uma operação associativa, isto é, para quaisquer
    palavras $x$, $y$ e $z$ temos que $x(yz) = (xy)z$. Além disso, $\lambda$ é
    o elemento neutro da concatenação, isto é, para qualquer palavra $x$,
    $x\lambda = \lambda x = x$.
  \end{Example}
  \begin{Definition}[Concatenação repetida]
    Seja $w$ uma palavra qualquer sobre um alfabeto $\Sigma$. A concatenação
    repetida de $w$, $w^n$, é definida por recursão sobre $n$ como:
    \[
      \begin{array}{lcl}
        w^0 & = & \lambda\\
        w^{n + 1} & = & w^nw\\
      \end{array}
    \]
  \end{Definition}
  \begin{Example}
    Alguns exemplos da operação de concatenação repetida:
    \begin{itemize}
      \item $(01)^3 = 010101$; $(110)^0 = \lambda$ e $(110)^3 = 110110110$.
    \end{itemize}
  \end{Example}

  \section{Linguagens e operações}

  \begin{Definition}[Linguagem]
    Seja $\Sigma$ um alfabeto. Denominamos por linguagem um conjunto de palavras
    sobre $\Sigma$.
  \end{Definition}

  \begin{Example}
    Como linguagens são conjuntos de palavras, existem linguagens finitas e
    infinitas. Além disso, podemos nos valer de diversas operações sobre
    conjuntos para definir linguagens.
    \begin{itemize}
      \item $\emptyset$, é uma linguagem sobre qualquer alfabeto.
      \item $\{0, \lambda\}$ é uma linguagem sobre $\Sigma =\{0,1\}$.
      \item $\{(01)^n\,|\,n \geq 0\}$ é linguagem de palavras formadas por
        sequencias sobre $01$.
      \item $\{1^n| n \geq 1\}\,\{0^n| 1 \leq n \leq 10\}$ é a linguagem
           formada por um prefixo de $n \geq 0$ 1's seguido de um sufixo $1 \leq
           p \leq 10$ 1's.
    \end{itemize}
  \end{Example}

  \begin{Definition}[Outras operações sobre palavras]
    Seja $w = a_1a_2...a_n$ uma palavra sobre um alfabeto $\Sigma$. Denomina-se
    \emph{reverso} de $w$, $w^R$, a palavra $w^R = a_n a_{n - 1}...a_2a_1$.
    Em especial, diz-se que uma palavra $w$ é um \emph{palíndromo} se
    $w = w^R$.  Seja $w = xyz$, em que tanto $x$, $y$ e $z$ podem ser $\lambda$
    ou não. Dizemos que $y$ é uma \emph{subpalavra} de $w$, $x$ é um
    \emph{prefixo} de $w$ e $z$ é um \emph{sufixo} de $w$. Em particular, tanto
    $\lambda$ quanto $w$ são prefixo, sufixo e subpalavra de $w$.
  \end{Definition}

  \begin{Example}
    Seja $\Sigma = \{0,1\}$. Temos que $(100)^R = 001$ e a palavra $101$ é um
    palíndromo, visto que $101 = (101)^R$. Seja $w = 010$. Temos que
    $\lambda, 0,01$ e $010$ são prefixos de $w$. Por sua vez,
    $\lambda,1, 01$ e $010$ são sufixos de $w$. As subpalavras de $w$ são
    $\lambda, 0, 1, 01, 10, 010$.
  \end{Example}

  \begin{Definition}[Concatenação de linguagens]
    Sejam $L_1$ e $L_2$ duas linguagem sobre um alfabeto $\Sigma$. Denotamos
    a concatenação de $L_1$ com $L_2$, $L_1L_2$, como o seguinte conjunto:
    \[
      L_1L_2 = \{w_1w_2\,|\,w_1 \in L_1 \land w_2 \in L_2\}
    \]
  \end{Definition}
  \begin{Example}
    Seja $\Sigma = \{0,1\}$, $L_1 = \{0,01,11\}$ e $L_2 = {1,\lambda}$. Temos
    que $L_1L_2$ é o conjunto
    \[
      L_1L_2 =\{01, 0, 011, 111, 11\}
    \]
    Um outro exemplo: Considere $\{0,1\}\{0,1\}$ é igual a:
    \[
      \{0,1\}\{0,1\} = \{00,01,10,11\}
    \]
  \end{Example}

  \begin{Definition}[Concatenação repetida de linguagens]
    Seja $L$ uma linguagem sobre um alfabeto $\Sigma$. A concatenação
    repetida de $L$, $L^n$, é definida por recursão sobre $n$ como:
    \[
      \begin{array}{lcl}
        L^0 & = & \{\lambda\}\\
        L ^{n + 1} & = & L^n\,L
      \end{array}
    \]
  \end{Definition}

  \begin{Example}
    Seja $L = \{01,0\}$ uma linguagem sobre $\Sigma = \{0,1\}$. Temos que:
    \[
      \begin{array}{lcl}
        L^2 & = & L^{2 - 1} L \\
            & = & L^1 L \\
            & = & (L^0 L) L \\
            & = & (\{\lambda\} L) L \\
            & = & (\{\lambda\} \{01,0\})\{01,0\} \\
            & = & \{01,0\}\{01,0\} \\
            & = & \{0101, 010, 001, 00\}\\
      \end{array}
    \]
    Ainda considerando $\Sigma = \{0,1\}$, temos que $\Sigma^3$ é definido como:
    \[
      \begin{array}{lcl}
        \Sigma^3 & = & \Sigma^2 \Sigma \\
                 & = & (\Sigma^1 \Sigma) \Sigma \\
                 & = & ((\Sigma^0 \Sigma) \Sigma) \Sigma \\
                 & = & ((\{\lambda\}\Sigma)\Sigma) \Sigma \\
                 & = & (\Sigma \Sigma) \Sigma \\
                 & = & (\{0,1\}\{0,1\})\{0,1\} \\
                 & = & (\{00,01,10,11\})\{0,1\} \\
                 & = & \{000, 001, 010, 011, 100, 101, 110, 111\}
      \end{array}
    \]
    Observe que $\Sigma^n$ produz o conjunto de todas as palavras de tamanho $n$
    para um alfabeto $\Sigma$.
  \end{Example}

  \begin{Definition}{Fecho de Kleene}
    Seja $L$ uma linguagem sobre um alfabeto $\Sigma$. Definimos o fecho de
    Kleene de $L$, $L^*$, como:
    \[
      L^* = \bigcup_{n \in\mathbb{N}}L^n
    \]
    Isto é, $L^*$ é a união de todos os $L^n$ para $n \in \mathbb{N}$. Outra
    forma de definir $L^*$ é usando uma definição recursiva como a seguinte:
    \begin{itemize}
       \item $\lambda \in L^*$.
       \item Se $x \in L^*$ e $y \in L$ então $xy \in L^*$.
    \end{itemize}
    Em particular, define-se o fecho positivo de Kleene de $L$, $L^+$, como
    \[
      L^+ = \bigcup_{n \in\mathbb{N} - \{0\}}L^n
    \]
    Finalmente, note que $L^* = L^+ \cup \{\lambda\}$.
  \end{Definition}

  \begin{Example}
    Segue alguns exemplos do fecho de Kleene de algumas linguagens.
    \begin{itemize}
      \item $\emptyset^* = \{\lambda\}$ e $\emptyset^+ = \emptyset$.
      \item $\{\lambda\}^* = \{\lambda\}^+ = \{\lambda\}$.
      \item $\{1\}^* = \{1^n\,|\,n \in \mathbb{N}\}$.
      \item $\{\lambda,00,11\}^* = \{\lambda\} \cup \{00,11\}^*$.
    \end{itemize}
    Em particular, $\Sigma^*$ denota o conjunto de todas as palavras,
    de todos os tamanhos para um certo alfabeto $\Sigma$. Além disso,
    podemos dizer que $L$ é uma linguagem sobre um alfabeto $\Sigma$
    simplesmente especificando que $L \subseteq \Sigma^*$.
  \end{Example}

  \section{Especificando linguagens usando conjuntos}

  A seguir apresentamos alguns exemplos ilustrando o uso de operações
  e notações da teoria de conjuntos e operações recém definidas sobre
  linguagens para especifar novas linguagens.

  \begin{Example}
    A seguir apresentamos uma descrição textual de uma linguagem
    e em seguida apresentamos sua descrição como conjunto.
    \begin{itemize}
      \item Conjunto de palavras que começam com 0: $\{0\}\{0,1\}^*$.
            Observe que $\{0,1\}^*=\{\lambda,0,1,00,01...\}$. Logo,
            Concatenando a linguagem $\{0\}$ a $\{0,1\}^*$ irá produzir
            $\{0\}\{0,1\}^* = \{0,00,01, 000, ...\}$, como requerido.
      \item Conjunto de palavras que terminam em 00: $\{0,1\}^*\{00\}$.
      \item Conjunto de palavras que contém 00 ou 11:
        $\{0,1\}^*\{00,11\}\{0,1\}^*$.
      \item Conjunto de palavras de tamanho par: $(\{0,1\}^2)^*$.
      \item Conjunto de palavras de tamanho par que começam com 0:
            $(\{0,1\}^2)^* \cap \{0\}\{0,1\}^*$.
    \end{itemize}
  \end{Example}


  \section{Problemas de decisão}

  \begin{Definition}[Problema de decisão]
    Um problema de decisão é uma questão que faz referência a um
    conjunto finito de parâmetros, e que, para valores específicos dos
    parâmetros, tem como resposta sim ou não.
  \end{Definition}

  \begin{Example}
    A seguir apresentamos alguns exemplos de problemas de decisão.
    \begin{itemize}
      \item Determinar se $n \in \mathbb{N}$ é um número primo.
      \item Determinar se um grafo $G$ possui um ciclo.
      \item Determinar se um grafo possui um ciclo hamiltoniano.
      \item Determinar se um programa pára com uma determinada entrada.
    \end{itemize}
  \end{Example}

  \section{Exercícios}

  \begin{enumerate}
    \item Apresente uma definição recursiva para o fecho positivo de Kleene para
      uma linguagem $L$.
    \item Apresente definições, usando operações de conjuntos, para as seguintes
      linguagens. Considere $\Sigma = \{0,1\}$.
      \begin{enumerate}
         \item Palavras que contém pelo menos um 0.
         \item Palavras de tamanho ímpar.
         \item Palavras que não possuem nenhuma ocorrência de 00.
         \item Palavras em que todo 0 é seguido por pelo menos dois 1's.
           Exemplo: $\lambda, 1, 1111, 011, 11101101111,...$.
      \end{enumerate}
      \item Apresente definições recursivas para as seguintes linguagens.
      \begin{enumerate}
         \item $\{0\}^*\{1\}^*$.
         \item $\{0^n1^n\,|\,n\in\mathbb{N}\}$.
      \end{enumerate}
  \end{enumerate}
\end{document}
