\documentclass[a4paper]{article}

\usepackage[portuguese]{babel}
\usepackage[utf8]{inputenc}
\usepackage{graphicx,hyperref}
\usepackage{float}
\usepackage{listings}
\usepackage{proof,tikz}
\usepackage{algorithm2e}
\usepackage{amssymb,amsthm,stmaryrd}


\usepackage[edges]{forest}
\usetikzlibrary{automata, positioning, arrows}


\newtheorem{Lemma}{Lema}
\newtheorem{Theorem}{Teorema}
\theoremstyle{definition}
\newtheorem{Example}{Exemplo}
\newtheorem{Definition}{Definição}
\newtheorem{Fact}{Fato}


\usepackage{fancyhdr}
  \pagestyle{fancy}
  \fancyhf{}
  \lhead{Teoria da Computação}
  \rhead{Aula 25}
  \lfoot{Prof. Rodrigo Ribeiro}
  \rfoot{\thepage}
  \renewcommand{\footrulewidth}{0.4pt}
  \pagestyle{fancy}

\tikzset{
        ->,  % makes the edges directed
        >=stealth', % makes the arrow heads bold
        node distance=3cm,
        every state/.style={thick, fill=gray!10},
        initial text=$\,$
        }
  

\begin{document}

\title{Aula 25 - O Teorema de Rice}
  \author{Rodrigo Ribeiro}

  \maketitle

  \pagestyle{fancy}


  \section*{Objetivos}

  \begin{itemize}
    \item Apresentar e demonstrar o teorema de Rice.
    \item Apresentar como o teorema de Rice pode ser utilizado para
          provar que problemas são indecidíveis.
  \end{itemize}


  \section{Introdução}

  Na aula 24, vimos alguns problemas indecidíveis e suas respectivas provas de
  indecidibilidade. Porém, de maneira geral, todos os problemas apresentados
  possuem a seguinte estrutura, quando expressos como uma linguagem de códigos
  de MTs:

  \[
    \{R\langle M \rangle\,|\, L(M) \textit{ satisfaz a propriedade }P\}
  \]

  Como exemplo, para o problema da fita em branco, temos que a propriedade $P$ é
  $\lambda \in L(M)$. Nesta aula, apresentaremos o teorema de Rice, que
  generaliza a indecibilidade de problemas com essa estrutura. O teorema mostra
  que quase todas as propriedades sobre códigos de MTs resultam em linguagens
  que não são recursivas.

  \section{Teorema de Rice}

  O enunciado do teorema de Rice faz menção a propriedades não triviais, cuja
  definição é dada a seguir.

  \begin{Definition}
    Dizemos que uma propriedade $P$ sobre linguagens recursivamente enumeráveis
    é trivial se ela é verdadeira para todas ou nenhuma linguagem recursivamente
    enumerável. 
  \end{Definition}

  \begin{Example}
    Primeramente, note que existem apenas duas propriedades triviais: as funções
    constantes sobre valores booleanos. Quaisquer outras propriedades são não
    triviais. A seguir, apresentamos exemplos de propriedades não triviais de
    linguagens recursivamente enumeráveis.
    \begin{itemize}
       \item Determinar se $L(M) = \emptyset$.
       \item Determinar se $L(M) = \Sigma^*$.
       \item Determinar se $L(M)$ é finita.
       \item Determinar se $L(M)$ é regular.
    \end{itemize}
  \end{Example}

  A seguir apresentamos o enunciado e a demonstração do teorema de Rice.

  \begin{Theorem}
    Toda propriedade não trivial de linguagens recursivamente enumeráveis é
    indecidível.
  \end{Theorem}
  \begin{proof}
    Suponha que $P$ seja uma propriedade não trivial de linguagens
    recursivamente enumeráveis. Suponha, adicionalmente, que $P(\emptyset) =
    \bot$. Uma vez que $P$ é não trivial, deve existir uma linguagem $L$ tal que
    $P(L) = \top$. Seja $M_L$ uma MT que aceita a linguagem $L$. Agora,
    apresentaremos uma função $\sigma$ que reduz o problema da parada ao
    conjunto $\{R\langle M \rangle\,|\,P(L(M)) = \top\}$ mostrando que esse
    último não é uma linguagem recursiva. A função $\sigma$ cria uma MT
    $M'$ que sobre a entrada $y$, realiza os seguintes passos:
    \begin{enumerate}
       \item Escreve a palavra $w$, parte da entrada do problema da parada,
         depois de $y$, isto é, se a fita possui a forma $\langle y$ ela passará
         a ter o formato $\langle y \lbrack w $, em que $\lbrack$ é um novo
         símbolo que representará o início da fita de $M$, a MT fornecida como
         parte da entrada do problema da parada.
       \item $M'$ se comporta como $M$ sobre a fita $\lbrack w$.
       \item Se $M$ para com entrada $w$, então $M'$ executa a MT $M_L$ sobre a
         entrada $y$ e para se $M_L$ parar com $y$.
     \end{enumerate}
     Agora, consideramos os seguintes casos:
     \begin{enumerate}
        \item $M$ para com entrada $w$. Nessa situação, temos que $M'$ executa a
          MT $M_L$ sobre a entrada $y$ e, portanto irá parar se $M_L$ parar com
          entrada $y$. Dessa forma, podemos dizer que $L(M') = L(M_L) = L$ e,
          foi suposto que $P(L(M')) = P(L) = \top$.
        \item $M$ não para com entrada $w$. Nessa situação, $M'$ entra em loop
          junto com $M$ e não aceita a entrada $y$ e, dessa forma, temos que
          $L(M') = \emptyset$. Logo, $P(L(M')) = P(\emptyset) = \bot$.
        \end{enumerate}
        Dessa forma, $P(L(M)) = \top$ se, e somente se, $M$ para com entrada
        $w$, o que conclui a indecidibilidade deste problema.
      \end{proof}

      Dado o teorema de Rice, o questão de mostrar a indecidibilidade de um
      problema resume-se em provar que a propriedade que descreve esse problema
      é uma propriedade não trivial. Para isso, basta mostrarmos exemplos que
      tornam a propriedade verdadeira e exemplos que tornam a propriedade falsa.
      O exemplo seguinte prova a indecidibilidade de um problema usando o
      teorema de Rice.

      \begin{Example}
        O problema da fita em branco pode ser descrito pela seguinte linguagem:
        \[
          \{R\langle M \rangle\,|\, \lambda \in L(M)\}
        \]
        Para essa linguagem, temos que a propriedade $P(L(M)) = \lambda \in
        L(M)$. Note que a seguinte linguagem $L_1$ é recursivamente enumerável
        e $\lambda \in L_1$: $L_1 = \{0\}^*$. Além disso, a linguagem $L_2$ é
        também recursivamente enumerável: $L_2 = \{0\}^+$. Note que $\lambda
        \not\in L_2$. Dessa forma, a propriedade $P(L(M)) = \lambda \not\in
        L(M)$ é não trivial e, pelo teorema de Rice, a linguagem
        $\{R\langle M \rangle\,|\, \lambda \in L(M)\}$ não é recursiva, o que
        mostra que o problema da fita em branco é indecidível.
      \end{Example}
  \section{Exercícios}

  \begin{enumerate}
     \item Considere os problemas seguintes. Caso seja possível, demonstre a sua
       indecibilidade utilizando o teorema de Rice.
       \begin{enumerate}
          \item Dada uma MT $M$ e uma palavra $w$, determinar se $M$ move
            o cabeçote à esquerda durante o processamento de $w$.
          \item Dada uma MT $M$, uma palavra $w$ e um estado $e$, determinar
            se a computação de $w$ por $M$ atinge o estado $e$.
          \item Dada uma MT $M$, determinar se $M$ aceita alguma palavra
            iniciada com $0$.
          \item Data uma MT $M$, determinar se $M$ não aceita palavras de
            tamanho par.
       \end{enumerate}
  \end{enumerate}
\end{document}
