\documentclass[a4paper]{article}

\usepackage[portuguese]{babel}
\usepackage[utf8]{inputenc}
\usepackage{graphicx,hyperref}
\usepackage{float}
\usepackage{listings}
\usepackage{proof,tikz}
\usepackage{amssymb,amsthm,stmaryrd, amsmath}


\usepackage[edges]{forest}
\usetikzlibrary{automata, positioning, arrows}


\newtheorem{Lemma}{Lema}
\newtheorem{Theorem}{Teorema}
\theoremstyle{definition}
\newtheorem{Example}{Exemplo}
\newtheorem{Definition}{Definição}


\usepackage{fancyhdr}
  \pagestyle{fancy}
  \fancyhf{}
  \lhead{Teoria da Computação}
  \rhead{Aula 14}
  \lfoot{Prof. Rodrigo Ribeiro}
  \rfoot{\thepage}
  \renewcommand{\footrulewidth}{0.4pt}
  \pagestyle{fancy}

\tikzset{
        -,  % makes the edges directed
        >=stealth', % makes the arrow heads bold
        node distance=3cm,
        every state/.style={thick, fill=gray!10},
        initial text=$\,$
        }
  

\begin{document}

  \title{Aula 14 - Formas normais de Chomsky e Greibach}
  \author{Rodrigo Ribeiro}

  \maketitle


  \pagestyle{fancy}


  \section*{Objetivos}

  \begin{itemize}
     \item Apresentar a definição das formas normais de Chomsky e Greibach.
     \item Apresentar a construção de GLCs equivalentes nas formais normais
       a partir de uma GLC qualquer.     
  \end{itemize}


  \section{Forma normal de Chomsky}

  \begin{Definition}[Forma normal de Chomsky (FNC)]\label{FNC}
    Seja uma GLC $G=(V,\Sigma,R,P)$. Dizemos que $G$ está na forma normal de
    Chomsky (FNC) se todas as suas regras estão nas formas:
    \begin{itemize}
      \item $P \to \lambda$
      \item $X \to a$, $X \in V$ e $a \in\Sigma$
      \item $X \to YZ$, $X,Y,Z \in V$.
    \end{itemize}
  \end{Definition}

  \begin{Lemma}\label{lema1}
    Para qualquer GLC $G = (V,\Sigma,R,P)$ existe uma GLC equivalente em que
    todas as regras são das formas:
    \begin{itemize}
      \item $P \to \lambda$, se $\lambda \in L(G)$.
      \item $X \to a$, para $X \in V$ e $a \in \Sigma$.
      \item $X \to w$, para $X \in V$, $|w|\geq 2$ e $w \in (V\cup\Sigma)^*$.
    \end{itemize}
  \end{Lemma}
  \begin{proof}
    A GLC equivalente a $G$ com essas propriedades é a resultante de
    \begin{itemize}
       \item eliminar variáveis anuláveis.
       \item eliminar variáveis encadeadas.
       \item eliminar símbolos inúteis.
    \end{itemize}
    da gramática $G_1 = (V \cup \{P_1\}, \Sigma, R\cup \{P_1 \to P\}, P_1)$.   
  \end{proof}
  \begin{Theorem}\label{thmfnc}
    Seja uma GLC $G$. Existe uma GLC equivalente a $G$ na FNC.
  \end{Theorem}
  \begin{proof}
    Pelo Lema \ref{lema1}, temos uma GLC em que toda regra são das formas:
    \begin{itemize}
      \item $P \to \lambda$, se $\lambda \in L(G)$.
      \item $X \to a$, para $X \in V$ e $a \in \Sigma$.
      \item $X \to w$, para $X \in V$, $|w|\geq 2$ e $w \in (V\cup\Sigma)^*$.
    \end{itemize}
    Logo, para estar na FNC, basta lidar com as regras da forma $X \to w$, $|w|
    \geq 2$. Para isso:
    \begin{enumerate}
      \item Modificar cada regra $X \to w$, $|w|\geq 2$, de forma que ela
        contenha apenas variáveis. Para isso, substitua cada $a\in \Sigma$ em
        $w$ por uma variável $Y \to a$, caso essa regra exista. Caso não haja
        tal regra na gramática, criar uma nova variável $Z$ com a regra $Z \to
        a$ e substituir $a$ por $Z$ em $w$.
      \item Substituir cada regra $X \to Y_1Y_2...Y_n$, $n \geq 3$, em que cada
        $Y_i$ é uma variável, pelas regras:
        \[
          \begin{array}{l}
            X \to Y_1 Z_1\\
            Z_1 \to Y_2Z_2 \\
            \vdots \\
            Z_{n - 2} \to Y_{n - 1}Y_n\\
          \end{array}
        \]
        em que cada $Z_i$ é uma nova variável.
    \end{enumerate}
  \end{proof}

  \begin{Example}
    Considere a seguinte GLC:
    \[
      \begin{array}{l}
        L \to (S) \\
        S \to SE \,|\, \lambda\\
        E \to a \,|\, L \\
      \end{array}
    \]
    Vamos mostrar o passo-a-passo da obtenção da GLC equivalente na FNC.
    Primeiro, vamos obter uma GLC com um novo símbolo de partida, conforme
    especificado pelo Lema \ref{lema1}.
    \[
      \begin{array}{l}
        P \to L \\
        L \to (S) \\
        S \to SE \,|\, \lambda\\
        E \to a \,|\, L \\
      \end{array}
    \]
    O próximo passo consiste em eliminar variáveis anuláveis. O conjunto de
    variáveis anuláveis de $G$ é: $\mathcal{A}_G =\{S\}$. Eliminando as regras
    $\lambda$, temos:
    \[
      \begin{array}{l}
        P \to L \\
        L \to (S)\,|\,() \\
        S \to SE \,|\, E\\
        E \to a \,|\, L \\
      \end{array}
    \]
    Na sequência, devemos eliminar as variáveis encadeadas. O conjunto de
    variáveis encadeadas é apresentado a seguir.
    \[
      \begin{array}{l}
        enc(L) =\{L\} \\
        enc(E) =\{E,L\}  \\
        enc(S) = \{S, E, L\} \\
        enc(P) = \{P,L\}\\
      \end{array}
    \]
    Eliminando as variáveis encadeadas, obtemos a seguinte gramática:
    \[
      \begin{array}{l}
        P \to (S) \,|\, () \\
        L \to (S)\,|\,() \\
        S \to SE \,|\, a \,|\, (S)\,|\, ()\\
        E \to a \,|\, (S)\,|\,() \\
      \end{array}
    \]
    Finalmente, resta eliminar os símbolos inúteis. Note que a variável $L$ não
    é alcançável a partir de $P$. Logo, todas as regras que mencionam $L$ devem
    ser removidas da gramática. Com isso obtemos:
    \[
      \begin{array}{l}
        P \to (S) \,|\, () \\
        S \to SE \,|\, a \,|\, (S)\,|\, ()\\
        E \to a \,|\, (S)\,|\,() \\
      \end{array}
    \]
    Que é a gramática resultante da aplicação da construção do Lema \ref{lema1}.
    Para obter a GLC na FNC, basta transformar todas as regras de tamanho maior
    ou igual a 2. Fazendo esse passo, obtemos:
    \[
      \begin{array}{l}
        P \to AZ_1 \,|\, AF \\
        A \to (\\
        F \to )\\
        Z_1 \to SF \\
        S \to SE \,|\, a \,|\, AZ_1\,|\, AF\\
        E \to a \,|\, AZ_1\,|\,AF \\
      \end{array}
    \]
    que é a GLC equivalente a gramática original na FNC.
  \end{Example}

  \section{Forma Normal de Greibach (FNG)}


  \begin{Definition}[Forma normal de Greibach (FNG)]
      Seja uma GLC $G=(V,\Sigma,R,P)$. Dizemos que $G$ está na forma normal de
    Greibach (FNG) se todas as suas regras estão nas formas:
    \begin{itemize}
      \item $P \to \lambda$
      \item $X \to ay$, $X \in V, a \in\Sigma$ e $y \in V^*$
    \end{itemize}
  \end{Definition}

  \begin{Lemma}\label{lema2}
    Seja uma GLC $G = (V,\Sigma,R,P)$ em que $X \to uYv \in R$, $X,Y \in V$ e $X
    \neq Y$. Sejam $Y \to w_1\,|\,...\,|\,w_n$ todas as regras $Y$ em $G$.
    Seja $G' = (V,\Sigma,R',P)$ em que:
    \[
      R' = \{R - \{X \to uYv\}\} \cup \{X \to uw_1v\,|\,...\,|\,uw_nv\}
    \]
    Temos que $L(G') = L(G)$.
  \end{Lemma}
  
  \begin{Theorem}\label{thmfng}
    Seja uma GLC $G$. Existe uma GLC equivalente a $G$ na FNG.
  \end{Theorem}
  \begin{proof}
    Primeiramente, usamos o Lema \ref{lema1} e aplicamos o primeiro passo
    (converter $a\in \Sigma$ em variáveis) do Teorema \ref{thmfnc}. Com isso,
    teremos uma gramática em que toda regra possui a forma $X \to Yy$, em que
    $X,Y \in V$ e $y \in V^+$. Em seguida, para cada $X \in
    V$ faça enquanto possível:
    \begin{enumerate}
       \item Se existe uma regra $X \to Yy$, $|y| \geq 1$,
          aplica-se o Lema \ref{lema2} para substituir 
        $Y$ (isso não se aplica a variáveis introduzidas por eliminação de
        recursão à esquerda).
      \item Se existe uma regra $X \to Xy$, para $|y| \geq 1$, aplica-se 
        a remoção de recursão à esquerda.
      \item Se existir uma regra $X \to aw$ em que $w$ possui símbolos do
        alfabeto, substituí-lo por uma variável.
    \end{enumerate}
  \end{proof}

  \begin{Example}
    Como um exemplo da construção de uma gramática equivalente na FNG, vamos
    considerar a gramática utilizada anteriormente para a FNC. Aqui começaremos
    o processo a partir da aplicação do Lema \ref{lema1}.
    \[
      \begin{array}{l}
        P \to (S) \,|\, ()\\
        S \to SE \,|\, a \,|\, (S)\,|\, ()\\
        E \to a \,|\, (S)\,|\,()\\
      \end{array}
    \]
    Agora, vamos trocar todos os símbolos do alfabeto em questão por variáveis,
    em regras $X \to w$, $|w| \geq 2$. Isso resulta na seguinte gramática:
    \[
      \begin{array}{l}
        P \to ASF \,|\, AF \\
        S \to SE \,|\, a \,|\, ASF\,|\, AF\\
        E \to a \,|\, ASF\,|\,AF\\
        A \to ( \\
        F \to ) \\
      \end{array}
    \]
     Agora, vamos proceder
    à execução dos passos adicionais presentes no Teorema \ref{thmfng}.
    
    Inicialmente, vamos usar o Lema \ref{lema2} para eliminar a regra $A \to ($.
    Isso resulta na gramática:
    \[
      \begin{array}{l}
        P \to (SF \,|\, (F \\
        S \to SE \,|\, a \,|\, (SF\,|\, (F\\
        E \to a \,|\, (SF\,|\,(F\\
        F \to ) \\
      \end{array}
    \]
    Observe que a única regra que não está de acordo com a FNG é $S \to SE$.
    Agora aplicamos a eliminação de recursão à esquerda. Com isso, obtemos:
    \[
      \begin{array}{l}
        P \to (SF \,|\, (F \\
        S \to aZ_1\,|\, (SFZ_1 \,|\, (FZ_1\,|\, a \,|\, (SF\,|\, (F\\
        E \to a \,|\, (SF\,|\,(F\\
        F \to ) \\
        Z_1 \to EZ_1 \,|\, E
      \end{array}
    \]
    O próximo passo, é eliminar as regras $E$. Com isso, obtemos:
    \[
      \begin{array}{l}
        P \to (SF \,|\, (F \\
        S \to aZ_1\,|\, (SFZ_1 \,|\, (FZ_1\,|\, a \,|\, (SF\,|\, (F\\
        F \to ) \\
        Z_1 \to aZ_1 \,|\, (SFZ_1 \,|\, (FZ_1\,|\, a\,|\,(SF \,|\, (F\\
      \end{array}
    \]
    que é uma gramática equivalente na FNG.
  \end{Example}

  \section{Exercícios} 

  \begin{enumerate}
     \item Construa uma gramática equivalente na FNC.
       \[
         \begin{array}{l}
           P \to BPA \,|\,A \\
           A \to aA \,|\, \lambda\\
           B \to Bba \,|\,\lambda\\
         \end{array}
       \]
     \item Construa uma gramática equivalente na FNG
       \[
         \begin{array}{l}
           P \to A \,|\,BC\\
           A \to B \,|\, C\\
           C \to cC \,|\, c\\
         \end{array}
       \]
     \item Construa uma gramática equivalente na FNC.
       O alfabeto para essa gramática é $\{0,1,2\}$.
       \[
         \begin{array}{l}
           T \to 0\, |\, 1 \,|\, T\,T\,|\,2\,1\,T 
         \end{array}
       \]
     \item Mostre como construir uma gramática na FNC a partir de uma gramática
       na FNG.
  \end{enumerate}
\end{document}
