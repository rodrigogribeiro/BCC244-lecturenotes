\documentclass[a4paper]{article}

\usepackage[portuguese]{babel}
\usepackage[utf8]{inputenc}
\usepackage{graphicx,hyperref}
\usepackage{float}
\usepackage{listings}
\usepackage{proof,tikz}
\usepackage{amssymb,amsthm,stmaryrd}


\usepackage[edges]{forest}
\usetikzlibrary{automata, positioning, arrows}


\newtheorem{Lemma}{Lema}
\newtheorem{Theorem}{Teorema}
\theoremstyle{definition}
\newtheorem{Example}{Exemplo}
\newtheorem{Definition}{Definição}


\usepackage{fancyhdr}
  \pagestyle{fancy}
  \fancyhf{}
  \lhead{Teoria da Computação}
  \rhead{Aula 11}
  \lfoot{Prof. Rodrigo Ribeiro}
  \rfoot{\thepage}
  \renewcommand{\footrulewidth}{0.4pt}
  \pagestyle{fancy}

\tikzset{
        ->,  % makes the edges directed
        >=stealth', % makes the arrow heads bold
        node distance=3cm,
        every state/.style={thick, fill=gray!10},
        initial text=$\,$
        }
  

\begin{document}

  \title{Aula 11 - Autômatos de pilha não determinísticos}
  \author{Rodrigo Ribeiro}

  \maketitle


  \pagestyle{fancy}


  \section*{Objetivos}

  \begin{itemize}
     \item Apresentar os conceito de autômatos de pilha determinísticos não
       determinístico (APNs).
     \item Apresentar o critérios alternativos de aceitação por APNs.
     \item Demonstrar a equivalência entre os critérios de aceitação
  \end{itemize}


  \section{Autômatos de pilha não determinísticos}

  \begin{Definition}[Autômato de pilha não determinístico]
    Um autômato de pilha não determinístico (APN) é uma sêxtupla
    $M=(E,\Sigma,\Gamma,\delta,I,F)$ em que:
    \begin{itemize}
       \item $E$: conjunto de estados.
       \item $\Sigma$: alfabeto de entrada.
       \item $\Gamma$: alfabeto de pilha.
       \item $\delta : E \times (\Sigma \cup \{\lambda\}) \times (\Gamma \cup
    \{\lambda\}) \to D$, $D\subseteq E \times \Gamma^*$, $D$ finito: função de transição. Uma função parcial.
       \item $I \subseteq E$: conjunto de estados iniciais.
       \item $F\subseteq E$: conjunto de estados finais.
    \end{itemize}
  \end{Definition}
 
  \begin{Example}
    Considere a linguagem de palavras sobre $\Sigma=\{0,1\}$ que possuem o mesmo
    número de 0s e de 1s. A seguir apresentaremos um APN que aceita essa linguagem.
    \begin{figure}[H]
      \begin{tikzpicture}
        \node[state, initial, accepting](s0){$A$};
        \node[state, right of=s0](s1){$B$};
        \draw (s0) edge[bend left, above]node{$\begin{array}{l}0,\lambda/Z\\ 1,\lambda/U\end{array}$}(s1)
        (s1) edge[loop right]node{$\begin{array}{l}0,U/\lambda\\1,Z/\lambda\end{array}$}(s1)
        (s1) edge[bend left, below]node{$\lambda,\lambda/\lambda$}(s0);
      \end{tikzpicture}
      \centering
    \end{figure}
    Note que esse APN é apenas uma simplificação do APD visto na aula 10. Apesar
    de mais compacto, esse APN pode ser simplificado ainda mais. O seguinte APN
    de um único estado reconhece a mesma linguagem.
    \begin{figure}[H]
      \begin{tikzpicture}
        \node[state, initial, accepting](s0){$A$};
        \draw (s0) edge[loop above]node{$\begin{array}{l}0,\lambda/Z\\ 0,U/\lambda\end{array}$}(s0)
        (s0) edge[loop below]node{$\begin{array}{l}1,\lambda/U\\ 1,Z/\lambda\end{array}$}(s0);
      \end{tikzpicture}
      \centering
    \end{figure}
  \end{Example}

  \section{Critérios alternativos de reconhecimento}

  Na aula 10, vimos um critério de reconhecimento para um APD $M = (E,\Sigma,\Gamma,\delta,i,F)$, que repetimos
  abaixo:

  \[
    L(M) = \{w\in\Sigma^*\,|\,\exists e. e\in F \land [i,w,\lambda]\vdash^*[e,\lambda,\lambda]\}
  \]

  Quando pensamos em APNs, esse critério é conhecido por aceitação por estado
  final e pilha vazia, conforme a próxima definição.

  \begin{Definition}[Aceitação por estado final e pilha vazia]
    Seja $M = (E,\Sigma,\Gamma,\delta,I,F)$ um APN. A linguagem de aceita por $M$
    por estado final e pilha vazia é:
    \[
      L(M) = \{w\in\Sigma^*\,|\,\exists i.\exists e. i\in I \land e\in F \land [i,w,\lambda]\vdash^*[e,\lambda,\lambda]\}
    \]
  \end{Definition}

  \begin{Example}
    Apresentaremos um APN que aceita por estado final e pilha vazia a linguagem
    $\{0^m1^n\,|\,m \geq n\}$.
    \begin{figure}[H]
      \begin{tikzpicture}
        \node[state, initial, accepting](s0){$A$} ;
        \node[state,accepting, right of=s0](s1){$B$} ;
        \draw (s0)edge[loop above]node{$0,\lambda/X$}(s0)
        (s0)edge[above]node{
          $
          \begin{array}{l}
            \lambda, X/\lambda\\
            1, X/\lambda
          \end{array}
          $
        }(s1) 
        (s1)edge[loop above]node{
          $
          \begin{array}{l}
          \lambda, X/\lambda\\
            1, X /\lambda
          \end{array}
          $
        }(s1);
      \end{tikzpicture}
      \centering
    \end{figure}
  \end{Example}
  
  Um possível critério alternativo de aceitação é eliminar a restrição de toda
  computação terminar com pilha vazia. Esse critério é chamado de aceitação por
  estado final.

  \begin{Definition}[Aceitação por estado final]
    Seja $M = (E,\Sigma,\Gamma,\delta,I,F)$ um APN. A linguagem de aceita por $M$
    por estado final é:
    \[
      L_F(M) = \{w\in\Sigma^*\,|\,\exists i.\exists e. \exists y. i\in I \land
      e\in F \land y \in
      \Gamma^*.\land [i,w,\lambda]\vdash^*[e,\lambda,y]\}
    \]
  \end{Definition}

  \begin{Example}
    Apresentaremos um APN que aceita por estado final a linguagem
    $\{0^m1^n\,|\,m \geq n\}$.
    \begin{figure}[H]
      \begin{tikzpicture}
        \node[state, initial, accepting](s0){$A$} ;
        \node[state,accepting, right of=s0](s1){$B$} ;
        \draw (s0)edge[loop above]node{$0,\lambda/X$}(s0)
        (s0)edge[above]node{
          $
          \begin{array}{l}
            1, X/\lambda
          \end{array}
          $
        }(s1) 
        (s1)edge[loop above]node{
          $
          \begin{array}{l}
            1, X /\lambda
          \end{array}
          $
        }(s1);
      \end{tikzpicture}
      \centering
    \end{figure}
  \end{Example}
  
  O último critério alternativo aceita palavras cujo processamento pára em
  qualquer estado com pilha vazia. Este critério é conhecido como aceitação por
  pilha vazia.
  \begin{Definition}[Aceitação por pilha vazia]
    Seja $M = (E,\Sigma,\Gamma,\delta,I,F)$ um APN. A linguagem de aceita por $M$
    por pilha vazia é:
    \[
      L_V(M) = \{w\in\Sigma^*\,|\,\exists i. i\in I \land
      e \in E\land [i,w,\lambda]\vdash^*[e,\lambda,\lambda]\}
    \]
  \end{Definition}

  \begin{Example}
    Apresentaremos um APN que aceita por pilha vazia a linguagem
    $\{0^m1^n\,|\,m \leq n\}$.
    \begin{figure}[H]
      \begin{tikzpicture}
        \node[state, initial](s0){$A$} ;
        \node[state, right of=s0](s1){$B$} ;
        \draw (s0)edge[loop above]node{$0,\lambda/X$}(s0)
        (s0)edge[above]node{
          $
          \begin{array}{l}
            1, \lambda/\lambda\\
            1, X/\lambda
          \end{array}
          $
        }(s1) 
        (s1)edge[loop above]node{
          $
          \begin{array}{l}
          1, \lambda/\lambda\\
            1, X /\lambda
          \end{array}
          $
        }(s1);
      \end{tikzpicture}
      \centering
    \end{figure}
  \end{Example}

  \begin{Theorem}
    Seja $L$ uma linguagem. As seguintes afirmativas são equivalentes:
    \begin{enumerate}
      \item $L$ pode ser reconhecida por pilha vazia e estado final.
      \item $L$ pode ser reconhecida por estado final.
      \item $L\cup\{\lambda\}$ pode ser reconhecida por pilha vazia.
    \end{enumerate}
  \end{Theorem}
  \begin{proof}
    $1 \to 2$: Seja um APN $M = (E,\Sigma,\Gamma,\delta,I,F)$. Podemos
    obter um APN $M'$, de forma que $L_F(M') = L(M)$, em que
    $M'=(E\cup\{i',g\},\Sigma,\Gamma\cup\{F\},\delta',\{i'\},\{g\})$
    de forma que $\delta'$ inclui $\delta$ mais as seguintes transições:
    \begin{enumerate}
        \item para cada $i_k \in I$, $\delta'(i',\lambda,\lambda) =
               \{[i_k,F]\}$.
        \item para cada $f_j \in F$, $\delta'(f_j,\lambda,F)=\{[g,\lambda]\}$.
    \end{enumerate}
    em que $i',g \not\in E$ e $F \not\in \Gamma$.\\
    $2 \to 3$: Seja um APN $M = (E,\Sigma,\Gamma,\delta,I,F)$. Podemos
    obter um APN $M'$ de forma que $L_v(M') = L_F(M) \cup\{\lambda\}$, em que
    $M'=(E\cup\{i',g,h\},\Sigma,\Gamma \cup\{F\},\delta',\{i'\})$ de forma que
    $\delta'$ inclui $\delta$ mais as seguintes transições:
    \begin{enumerate}
      \item Para cada $i_k \in I$, $\delta'(i',\lambda,\lambda)=\{[i_k,F]\}$.
      \item Para cada $f_j \in F$, $\delta'(f_j,\lambda,\lambda)=\{[g,\lambda]\}$.
      \item Para cada $X \in \Gamma$, $\delta'(g,\lambda,X) = \{[g,\lambda]\}$.
      \item $\delta'(g,\lambda,F) = \{[h,\lambda]\}$
    \end{enumerate}
    $3 \to 1$: Seja um APN $M$ que aceita uma linguagem por pilha vazia. Para
    obtermos outro APN que aceita por pilha vazia e estado final, basta fazer
    todos os estados de $M$ finais.
  \end{proof}
  \section{Exercícios} 

  \begin{enumerate}
     \item Construa APNs que aceitem por pilha vazia e estado final para as seguintes linguagens.
       \begin{enumerate}
       \item $\{0^n1^n\,|\,n\geq 0\} \cup \{0^n1^{2n}\,|\,n\geq 0\}$
       \item $\{0^n1^n0^k\,|\,n,k \geq 0\}$
       \item $\{0^n1^m\,|\,m > n\}$
       \end{enumerate}
     \item Construa APNs que aceitem $\{0^n1^n\,|\,n\geq 0\}$:
       \begin{enumerate}
          \item Por estado final.
          \item Por pilha vazia.
          \end{enumerate}
     \item Explique como construir um APN de um único estado equivalente a um
       AFD.
  \end{enumerate}
\end{document}
