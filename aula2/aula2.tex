\documentclass[a4paper]{article}

\usepackage[portuguese]{babel}
\usepackage[utf8]{inputenc}
\usepackage{graphicx,hyperref}
\usepackage{float}
\usepackage{listings}
\usepackage{proof,tikz}
\usepackage{amssymb,amsthm,stmaryrd}


\usepackage[edges]{forest}
\usetikzlibrary{automata, positioning, arrows}


\newtheorem{Lemma}{Lema}
\newtheorem{Theorem}{Teorema}
\theoremstyle{definition}
\newtheorem{Example}{Exemplo}
\newtheorem{Definition}{Definição}


\usepackage{fancyhdr}
  \pagestyle{fancy}
  \fancyhf{}
  \lhead{Teoria da Computação}
  \rhead{Aula 2}
  \lfoot{Prof. Rodrigo Ribeiro}
  \rfoot{\thepage}
  \renewcommand{\footrulewidth}{0.4pt}
  \pagestyle{fancy}

\tikzset{
        ->,  % makes the edges directed
        >=stealth', % makes the arrow heads bold
        node distance=3cm,
        every state/.style={thick, fill=gray!10},
        initial text=$\,$
        }
  

\begin{document}

  \title{Aula 2 - Autômatos finitos determinísticos}
  \author{Rodrigo Ribeiro}

  \maketitle


  \pagestyle{fancy}


  \section*{Objetivos}

  \begin{itemize}
     \item Apresentar o conceito de autômato finito determínistico
           e exemplos dessa definição.
     \item Apresentar a definição de linguagem aceita por um autômato
           e equivalência entre autômatos finitos determínisticos.
  \end{itemize}

  \section{Um exemplo introdutório}

  \begin{Example}[Modelando manipulação de arquivos]
    Considere a tarefa de modelar o comportamento de uma biblioteca de leitura
    de arquivos em sua linguagem de programação predileta. Idealmente, essa
    deve conter funções para abrir um arquivo, ler conteúdo de um arquivo e
    fechar um arquivo.

    Uma maneira de modelarmos tal biblioteca seria usando um grafo, em que
    os vértices representam as possíveis configurações de um arquivo. Em
    um dado instante de tempo, um arquivo pode estar em uma de duas possíveis
    configurações: aberto ou fechado. A transição entre as possíveis
    configurações é feita por funções da biblioteca, a saber:
    \begin{itemize}
       \item \verb|open(f)|: Abre o arquivo \verb|f|.
       \item \verb|read(f,v)|: Lê uma parte do conteúdo do arquivo \verb|f|
         e o armazena em \verb|v|.
       \item \verb|close(f)|: fecha o arquivo \verb|f|.
    \end{itemize}
    Evidentemente, algumas configurações não são válidas. Por exemplo,
    chamar a operação \verb|read(f,v)| sobre um arquivo não aberto resulta
    em um erro em tempo de execução. Além disso, não fechar um arquivo após
    a execução pode ocasionar problemas de desempenho. Logo, além de nós para
    representar as configurações relativas ao arquivo estar aberto ou fechado
    devemos ter um nó para representar todas as configurações inválidas.
    Dessa forma, nosso modelo cobrirá todas as possibilidades de uso dessa
    biblioteca. Representaremos o estado fechado pela letra $C$, aberto por $O$
    e o estado de configuração inválida por $E$.

    Mas o que causará a mudança dessas configurações? Observe que a função
    \verb|open(f)| faz com que um arquivo fechado passe para o estado aberto,
    a função \verb|read(f,v)| não altera a configuração de um arquivo e
    a função \verb|close(f)| possui significado óbvio. Visando simplificar
    a escrita de nosso modelo, utilizaremos a letra $o$ para representar
    a função \verb|open(f)|, $r$ para \verb|read(f,v)| e $c$ para
    \verb|close(f)|. Abaixo apresentamos o diagrama que modela o
    comportamento descrito.
    \begin{figure}[ht]
      \begin{tikzpicture}[node distance=2.5cm]
        \node[state,initial,accepting](cl){$C$} ;
        \node[state, right of=cl] (op){$O$} ;
        \node[state, left of=cl] (err){$E$} ;
        
        \draw (cl) edge[bend left, above] node{$o$} (op)
              (op) edge[loop above] node{$r$} (op)
              (op) edge[bend left, below] node{$c$} (cl)
              (cl) edge[bend left, below] node{$r,c$} (err)
              (err)edge[loop above] node{$o,r,c$} (err) ;
      \end{tikzpicture}
      \centering      
    \end{figure}
    Note que o diagrama acima modela o comportamento esperado da biblioteca.
    Antes de lermos informações presentes em um arquivo, este deve estar
    aberto e todo uso da biblioteca deve ser finalizado com o fechamento
    de arquivo. Dessa forma, o comportamento correto da biblioteca pode
    ser especificado pelo seguinte conjunto de palavras sobre o alfabeto
    $\Sigma=\{o,c,r\}$: $\{o\}\{r\}^*\{c\}$.

    Existem tipos distintos de vértices nesse grafo. O primeiro é dito ser
    inicial e possui uma seta à sua esquerda. Intuitivamente, o processamento
    de uma palavra deve iniciar a partir desse vértice. Outro tipo de vértice é
    o final, representado com uma borda adicional (no exemplo, o mesmo vértice
    $C$ é inicial e final). Dizemos que uma palavra representa uma computação
    válida nesse modelo de biblioteca se iniciando o processamento no vértice
    inicial, consumindo um símbolo ao passar por cada aresta, a computação da
    palavra termina em um estado final. Formalizaremos essa idéia a seguir.

    O par $[e,w]$, em que $e$ é um dos vértices do grafo anterior e
    $w\in\{o,c,r\}^*$ é chamado de \emph{configuração instantânea} pois,
    representa um momento do processamento de uma palavra (ou estado da
    biblioteca de manipulação de arquivos). A palavra $orrc$
    repreesenta uma computação válida, visto que seu processamento termina
    em um nó final.
    \[
      [C,orrc] \vdash [O,rrc] \vdash [O,rc] \vdash [O,c] \vdash [C,\lambda] 
    \]
    Por sua vez, a palavra $orcrr$ termina em um estado não final ($E$),
    conforme a sequência de configurações abaixo.
    \[
      [C,orcrr] \vdash [O,rcrr] \vdash [O,crr] \vdash [C,rr] \vdash [E, r]
      \vdash [E,\lambda] 
    \]
    A próxima seção apresentará definições precisas deste tipo de modelo e
    de sua respectiva computação.
  \end{Example}
  
  \section{Autômatos finitos determinísticos}

  \begin{Definition}[Autômato finito determinístico]
    Um autômato finito determinístico (AFD) $M$ é uma quíntupla $M =
    (E,\Sigma,\delta,i,F)$, em que:
    \begin{itemize}
      \item $E$: conjunto finito de estados.
      \item $\Sigma$: alfabeto
      \item $\delta : E \times \Sigma \to E$: função de transição, uma
        função total.
      \item $i \in E$: estado inicial
      \item $F\subseteq E$: conjunto de estados finais.
    \end{itemize}
  \end{Definition}

  \begin{Example}
    Considere o AFD apresentado anteriormente que modela o comportamento
    de uma biblioteca de manipulação de arquivos. Sua representação de
    acordo com a definição formal de um AFD é: $M=(S,\{o,c,r\},\delta,C,\{C\})$,
    em que: $S = \{E,C,O\}$ e $\delta$ é tal que:
    \[
      \begin{array}{ll}
        \delta(E,r) = E & \delta(E,c) = E \\
        \delta(E,o) = E & \delta(C,r) = E \\
        \delta(C,c) = E & \delta(C,o) = O \\
        \delta(O,r) = O & \delta(O,c) = C \\
      \end{array}
    \]
    Observe que cada aresta rotulada representa uma entrada na
    função de transição. 
  \end{Example}

  \begin{Example}\label{afdimpar}
    Considere a seguinte linguagem $L\subseteq \{0,1\}^*$:
    \[
      L = \{w \in\{0,1\}^*\,|\, \exists k. |w| = 2k + 1\}
    \]
    isto é, palavras de tamanho ímpar. O seguinte diagrama de estados
    (grafo dirigido para representar um AFD) aceita palavras pertencentes
    a essa linguagem.
    \begin{figure}[ht]
      \begin{tikzpicture}
        \node[state,initial](p){$P$} ;
        \node[state, accepting, right of=p] (i){$I$} ;
        
        \draw (p) edge[bend left, above] node{$0,1$} (i)
              (i) edge[bend left, below] node{$0,1$} (p) ;
      \end{tikzpicture}
      \centering
    \end{figure}
  \end{Example}

  \begin{Example}\label{afd00}
    Considere a seguinte linguagem sobre $\Sigma = \{0,1\}$:
    \[
       L = \{0,1\}^*\{00\}\{0,1\}^*
     \]
     O seguinte AFD aceita as palavras pertencentes a esta
     linguagem.

     \begin{figure}[ht]
       \begin{tikzpicture}
         \node[state, initial]               (s0){$A$} ;
         \node[state,            right of=s0](s1){$B$} ;
         \node[state, accepting, right of=s1](s2){$C$} ;

         \draw (s0) edge[loop above]       node{$1$}   (s0)
               (s0) edge[bend left, above] node{$0$}   (s1)
               (s1) edge[above]            node{$0$}   (s2)
               (s1) edge[bend left, below] node{$1$}   (s0)
               (s2) edge[loop above]       node{$0,1$} (s2) ;
       \end{tikzpicture}
       \centering      
     \end{figure}
  \end{Example}  
  
  \subsection{Linguagem aceita por um AFD}

  \begin{Definition}[Função de transição estendida]
    Seja $M = (E,\Sigma, \delta, i, F)$ um AFD qualquer. A função de transição
    estendida para $M$, $\widehat{\delta} : E \times \Sigma^* \to E$, é definida
    da seguinte maneira:
    \[
      \begin{array}{lcl}
        \widehat{\delta}(e,\lambda) & = & e \\
        \widehat{\delta}(e,ay)      & = & \widehat{\delta}(\delta(e,a),y) \\
      \end{array}
    \]
    em que $a \in \Sigma$ e $y \in \Sigma^*$. Informalmente, dado um estado $e
    \in E$ e $w \in \Sigma^*$, $\widehat{\delta}(e,w)$ retorna o estado em que o
    processamento da palavra $w$ termina ao ser iniciado no estado $e$.
  \end{Definition}

  \begin{Example}
    Considere o AFD do Exemplo~\ref{afd00}. Temos que iniciando o processamento
    da palavra $010011$ no estado inicial, este termina no estado final.
    Conforme apresentado a seguir pelo cálculo de $\widehat{\delta}(A,010011)$.
    \[
      \begin{array}{lclcl}
       \widehat{\delta}(010011) & = & \widehat{\delta}(\delta(A,0),10011) & = &\widehat{\delta}(B,10011)\\
       \widehat{\delta}(\delta(B,1),0011) & = & \widehat{\delta}(A,0011) & =&\widehat{\delta}(\delta(A,0),011)\\
        \widehat{\delta}(B,011) & = & \widehat{\delta}(\delta(B,0),11) & = &\widehat{\delta}(C,11)\\ 
        \widehat{\delta}(\delta(C,1),1) & = & \widehat{\delta}(C,1) & =&\widehat{\delta}(\delta(C,1),\lambda)\\
        \widehat{\delta}(C,\lambda) & = & C
      \end{array}
    \]
    De maneira similar, podemos realizar o cálculo de $\widehat{\delta}(A,101)$.
    \[
      \begin{array}{lclcl}
        \widehat{\delta}(A,101) & = & \widehat{\delta}(A,01) &=
        &\widehat{\delta}(B,1) \\
        \widehat{\delta}(A,\lambda) & = & A
      \end{array}
    \]
    A partir desses exemplos, podemos perceber que se $i$ é o estado inicial de
    um AFD $M$ e $w$ é uma palavra sobre o alfabeto $\Sigma$ de $M$, então $w$ é
    aceita por $M$ se $\widehat{\delta}(i,w)$ é um estado final e recusada,
    caso contrário. A definição seguinte formaliza essa noção.
  \end{Example}

  \begin{Definition}{Linguagem aceita por um AFD}
    Seja $M = (E,\Sigma,\delta,i, F)$ um AFD qualquer. A linguagem aceita por
    $M$, $L(M)$, é definida como:
    \[
      L(M) = \{w \in \Sigma^*\,|\,\widehat{\delta}(i,w) \in F\}
    \]
  \end{Definition}


  \subsection{Relação entre configurações instantâneas de um AFD}

  \begin{Definition}[Configuração instantânea]
    Seja $M = (E,\Sigma, \delta,i, F)$ um AFD qualquer. Denomina-se por
    configuração instantânea de $M$ o par $[e,w]$ em que $e \in E$ e
    $w \in \Sigma^*$.
  \end{Definition}

  \begin{Definition}[Transição entre configurações instantâneas]
    Seja $M = (E,\Sigma, \delta,i, F)$ um AFD qualquer, $[e,w]$ e $[e',w']$ duas
    configurações instantâneas de $M$. A relação $\vdash \subseteq (E \times
    \Sigma^*)^2$ representa a transição entre duas configurações instantâneas de
    um AFD e é definida formalmente como: $[e,ay] \vdash [e', y]$ sempre que
    $\delta(e,a) = e'$. A notação $[e,w] \vdash^* [e',w']$ denotará o fecho
    transitivo e reflexivo de $\vdash$.
  \end{Definition}


  \subsection{Equivalência entre AFDs}


  \begin{Example}
    Considere o seguinte AFD que aceita a linguagem de palavras de tamanho ímpar:
    \begin{figure}[H]
      \begin{tikzpicture}[node distance=2cm]
        \node[state, initial]                (pp){$A$} ;
        \node[state, accepting, right of=pp] (pi){$B$} ;
        \node[state,accepting,  below of=pp] (ip){$C$} ;
        \node[state,  below of=pi] (ii){$D$} ;

        \draw (pp) edge[bend left, above]     node{$1$} (pi)
              (pi) edge[bend left, below]     node{$1$} (pp)
              (pp) edge[bend left, left]      node{$0$} (ip)
              (ip) edge[bend left, left]      node{$0$} (pp)
              (pi) edge[bend left, left]     node{$0$} (ii)
              (ii) edge[bend left, left]     node{$0$} (pi)
              (ip) edge[bend left, above]    node{$1$} (ii)
              (ii) edge[bend left, below]    node{$1$} (ip) ; 
      \end{tikzpicture}
      \centering
    \end{figure}

    Apesar de possuir mais estados, o AFD apresentado aceita a mesma linguagem
    do AFD apresentado no Exemplo~\ref{afdimpar}. A diferença entre os dois é
    o número de estados que cada um possui (note que o número de transições
    depende apenas do número de estados, visto que o alfabeto é o mesmo e a
    função de transição é total). Logo, isso nos leva as seguintes perguntas:
    Quando dois AFDs são equivalentes entre si?
    Existe um AFD com o menor número de estados possível para uma certa
    linguagem? A segunda pergunta será o assunto da próxima aula. A primeira, é
    respondida pela definição seguinte.
  \end{Example}

  \begin{Definition}{Equivalência de AFDs}
    Sejam $M_1$ e $M_2$ dois AFDs que reconhecem linguagens sobre um alfabeto
    $\Sigma$. Dizemos que $M_1$ é equivalente a $M_2$, $M_1 \equiv M_2$, se
    $L(M_1) = L(M_2)$.
  \end{Definition}
  
  \section{Exercícios}

  \begin{enumerate}
    \item Construa AFDs para as seguintes linguagens sobre $\Sigma = \{0,1\}$
      \begin{enumerate}
        \item Palavras cujo tamanho é múltiplo de 4.
        \item Palavras que terminam em 11.
        \item Palavras que não possuem nenhuma ocorrência de 000.
        \item Palavras que possuem um número par de 0's.
        \item Palavras que possuem um número par de 0's e ímpar de 1's.
      \end{enumerate}
    \item Apresente a definição formal dos AFDs apresentados nos exemplos dessas
      notas de aula.
    \item Apresente a definição da linguagem aceita por um AFD usando a relação
      de transição entre configurações intantâneas.
    \item Seja $M = (E,\Sigma,\delta,i,F)$ um AFD qualquer. Prove que,
      se $\widehat{\delta}(e,w) = e'$ então $[e,w]\vdash[e',\lambda]$, para todo
         $e,e' \in E$ e $w \in \Sigma^*$.
  \end{enumerate}
\end{document}
