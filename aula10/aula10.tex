\documentclass[a4paper]{article}

\usepackage[portuguese]{babel}
\usepackage[utf8]{inputenc}
\usepackage{graphicx,hyperref}
\usepackage{float}
\usepackage{listings}
\usepackage{proof,tikz}
\usepackage{amssymb,amsthm,stmaryrd}


\usepackage[edges]{forest}
\usetikzlibrary{automata, positioning, arrows}


\newtheorem{Lemma}{Lema}
\newtheorem{Theorem}{Teorema}
\theoremstyle{definition}
\newtheorem{Example}{Exemplo}
\newtheorem{Definition}{Definição}


\usepackage{fancyhdr}
  \pagestyle{fancy}
  \fancyhf{}
  \lhead{Teoria da Computação}
  \rhead{Aula 10}
  \lfoot{Prof. Rodrigo Ribeiro}
  \rfoot{\thepage}
  \renewcommand{\footrulewidth}{0.4pt}
  \pagestyle{fancy}

\tikzset{
        ->,  % makes the edges directed
        >=stealth', % makes the arrow heads bold
        node distance=3cm,
        every state/.style={thick, fill=gray!10},
        initial text=$\,$
        }
  

\begin{document}

  \title{Aula 10 - Autômatos de pilha}
  \author{Rodrigo Ribeiro}

  \maketitle


  \pagestyle{fancy}


  \section*{Objetivos}

  \begin{itemize}
     \item Apresentar os conceitos de autômatos de pilha determinísticos e
       transições compatíveis.
     \item Apresentar o conceito de linguagem aceita por um autômato de pilha.
     \item Exemplos de autômatos de pilha determinísticos.
  \end{itemize}

  \section{Introdução}

  Conforme apresentado em aulas passadas, linguagens regulares não são capazes
  de representar a estrutura de linguagens muito simples, como por exemplo,
  $\{0^n1^n\,|\,n\geq 0\}$. Isso se deve ao fato de que AF não possuem
  ``memória'' suficiente para armazenar o fato de que uma quantidade arbitrária
  de 0's foi processada e usar esse fato para verificar se a quantidade de 1's é
  a mesma. Como resolver esse problema?

  Bem, a solução óbvia é ``adicionar mais memória''. Os autômatos de pilha (AP )são,
  informalmente, autômatos finitos com uma pilha ilimitada. Uma transição em
  um AP possui o seguinte formato geral:
  \begin{figure}[H]
    \begin{tikzpicture}[node distance=3cm]
      \node[state](s0){$A$};
      \node[state, right of=s0](s1){$B$};
      \draw (s0)edge[above]node{$a,b/z$}(s1);
    \end{tikzpicture}
    \centering
  \end{figure}
  que pode ser formalmente descrito como $\delta(A,a,b) = [B,z]$. Informalmente,
  a transição sai do estado $A$ processando $a$ na palavra de entrada, com $b$
  no topo da pilha, resulta no estado $B$ e empilha $z$ após desempilhar $b$.
  A seguir apresentamos um exemplo concreto de AP e como a execução deste
  procede.

  \begin{Example}
    Considere o seguinte diagrama de estados para um AP que aceita
    $\{0^n1^n\,|\,n\geq 0\}$.
    \begin{figure}[H]
      \begin{tikzpicture}
        \node[state, initial, accepting](s0){$A$} ;
        \node[state, accepting, right of=s0](s1){$B$};
        \draw (s0) edge[loop above]node{$0,\,\lambda/X$}(s0)
              (s0) edge[above]node{$1,\,X/\lambda$}(s1)
              (s1) edge[loop above]node{$1,\,X/\lambda$}(s1);
      \end{tikzpicture}
      \centering
    \end{figure}
    Para exemplificar o funcionamento deste AP, vamos utilizar a noção de
    configuração instântanea, que consiste de uma tripla formada por um estado,
    a palavra atual e o estado da pilha. Representaremos a pilha como uma
    palavra sobre um alfabeto $\Gamma$ de pilha. Em nosso exemplo, o alfabeto de
    pilha é $\Gamma=\{X\}$ e a pilha vazia é representada por $\lambda$.

    De maneira intuitiva, dizemos que uma palavra $w \in \Sigma^*$ pertence a
    linguagem de um AP $M$ se esta iniciando seu processamento no estado inicial
    o termina com pilha vazia em estado final tendo sido completamente
    consumida. Apresentaremos alguns exemplos de transições nesse AP usando
    configurações instântaneas. Primeiramente, vamos mostrar o processamento de
    uma palavra aceita $0011$:
    \[
      \begin{array}{lcl}
        \lbrack A,0011,\lambda \rbrack & \vdash & \{\delta(A,0,\lambda) =
                                                  \lbrack A, X \rbrack\} \\
        \lbrack A,011, X \rbrack & \vdash & \{\delta(A,0,\lambda) =
                                            \lbrack A, X \rbrack\} \\
        \lbrack A,11, XX \rbrack & \vdash & \{\delta(A,1,X) =
                                            \lbrack B, \lambda \rbrack\} \\
        \lbrack B,1,X \rbrack & \vdash & \{\delta(B,1,X) = \lbrack B, \lambda
                                         \rbrack \} \\
        \lbrack B, \lambda, \lambda \rbrack
      \end{array}
    \]
    Logo, como $[A,0011,\lambda]\vdash^*[B,\lambda,\lambda]$, temos que $0011$ é
    aceita por esse AP.

    Agora, vamos considerar dois exemplos de palavras recusadas pelo AP
    apresentado. Primeiro, vamos considerar a palavra $001$. O processamento
    desta é apresentado a seguir.
    \[
      \begin{array}{lcl}
        \lbrack A,001,\lambda \rbrack & \vdash & \{\delta(A,0,\lambda) =
                                                  \lbrack A, X \rbrack\} \\
        \lbrack A,01, X \rbrack & \vdash & \{\delta(A,0,\lambda) =
                                            \lbrack A, X \rbrack\} \\
        \lbrack A,1, XX \rbrack & \vdash & \{\delta(A,1,X) =
                                            \lbrack B, \lambda \rbrack\} \\
        \lbrack B,\lambda,X \rbrack 
      \end{array}
    \]
    Neste caso, a palavra não é aceita por ter seu processamento finalizado com
    a pilha não vazia. Outro exemplo de não aceitação acontece quando a palavra
    de entrada não é completamente processada pelo AP. A palavra $011$ é um
    exemplo deste caso. Sua execução é apresentada abaixo.
    \[
      \begin{array}{lcl}
        \lbrack A,011,\lambda \rbrack & \vdash & \{\delta(A,0,\lambda) =
                                                  \lbrack A, X \rbrack\} \\
        \lbrack A,11, X \rbrack & \vdash & \{\delta(A,1,X) =
                                            \lbrack B, \lambda \rbrack\} \\
        \lbrack B,1,\lambda \rbrack 
      \end{array}
    \]
  \end{Example}

  \section{Autômatos de pilha determinísticos}

  \begin{Definition}[Transições compatíveis]
    Seja $\delta : E \times (\Sigma \cup \{\lambda\}) \times (\Gamma \cup
    \{\lambda\}) \to E \times \Gamma^*$. Duas transições $\delta(e,a,b)$ e
    $\delta(e',a',b')$ são compatíveis se, e somente se:
    \[
      (a = a' \lor a = \lambda \lor a' = \lambda) \land (b = b' \lor b = \lambda \lor b' = \lambda)
    \]
  \end{Definition}


  \begin{Example}
    Vamos considerar novamente o APD para $\{0^n1^n\,|\,n \geq 0\}$:
    \begin{figure}[H]
      \begin{tikzpicture}
        \node[state, initial, accepting](s0){$A$} ;
        \node[state, accepting, right of=s0](s1){$B$};
        \draw (s0) edge[loop above]node{$0,\,\lambda/X$}(s0)
              (s0) edge[above]node{$1,\,X/\lambda$}(s1)
              (s1) edge[loop above]node{$1,\,X/\lambda$}(s1);
      \end{tikzpicture}
      \centering
    \end{figure}
    Somente o estado $A$ possui duas transições e, portanto, tem a possibilidade
    de possuir transições compatíveis. Porém, as transições $\delta(A,0,\lambda)
    = [A,X]$ e $\delta(A,1,X) = [B,\lambda]$ não são compatíveis, visto que
    operam sobre diferentes símbolos do alfabeto de entrada.

    Como um exemplo de AP possuindo transições compatíveis considere:
    \begin{figure}[H]
      \begin{tikzpicture}
        \node[state, initial, accepting](s0){$A$} ; 
        \node[state, accepting, right of=s0](s1){$B$} ; 
        \draw (s0) edge[loop above]node{$\begin{array}{l}0,\lambda/Z\\1,\lambda/U\end{array}$}(s0)
              (s0) edge[above] node{$\begin{array}{l}0,\lambda/\lambda\\1,\lambda/\lambda\\\lambda,\lambda/\lambda\end{array}$}(s1)
              (s1) edge[loop above]node{$\begin{array}{l}0,Z/\lambda\\1,U/\lambda\end{array}$}(s1) ;
      \end{tikzpicture}
      \centering
    \end{figure}
    Observe que as transições $\delta(A,0,\lambda) = [B,\lambda]$ e
    $\delta(A,\lambda,\lambda) = [B,\lambda]$ são compatíveis, como pode ser
    facilmente verificado.
  \end{Example}

  \begin{Definition}[Autômato de pilha determinístico]
    Um autômato de pilha determinístico (APD) é uma sêxtupla
    $M=(E,\Sigma,\Gamma,\delta,i,F)$ em que:
    \begin{itemize}
       \item $E$: conjunto de estados.
       \item $\Sigma$: alfabeto de entrada.
       \item $\Gamma$: alfabeto de pilha.
       \item $\delta : E \times (\Sigma \cup \{\lambda\}) \times (\Gamma \cup
    \{\lambda\}) \to E \times \Gamma^*$: função de transição. Uma função parcial
    sem transições compatíveis.
       \item $i \in E$: estado inicial.
       \item $F\subseteq E$: conjunto de estados finais.
    \end{itemize}
  \end{Definition}

   \begin{Example}
    Vamos considerar novamente o APD para $\{0^n1^n\,|\,n \geq 0\}$:
    \begin{figure}[H]
      \begin{tikzpicture}
        \node[state, initial, accepting](s0){$A$} ;
        \node[state, accepting, right of=s0](s1){$B$};
        \draw (s0) edge[loop above]node{$0,\,\lambda/X$}(s0)
              (s0) edge[above]node{$1,\,X/\lambda$}(s1)
              (s1) edge[loop above]node{$1,\,X/\lambda$}(s1);
      \end{tikzpicture}
      \centering
    \end{figure}
    Como discutido em um exemplo anterior, este AP não possui transições
    compatíveis e, portanto, é determinístico.
  \end{Example}
 
  
  \begin{Definition}[Transição entre configurações instântaneas]
    Seja $M=(E,\Sigma,\Gamma,\delta,i,F)$ um APD. A relação $\vdash(E \times
    \Sigma^* \times \Gamma^*)^2$ para $M$ é tal que para todo $e\in E$,
    $a\in \Sigma\cup\{\lambda\}$, $b\in \Gamma\cup\{\lambda\}$ e $x \in
    \Gamma^*$:
    \[
      \lbrack e, ay, bz \rbrack \vdash \lbrack e', y, xz \rbrack
    \]
    sempre que $\delta(e,a,b) = [e',x]$. Como usual, $\vdash^*$ irá denotar
    o fecho transitivo e reflexivo de $\vdash$.
  \end{Definition}

  \begin{Definition}[Linguagem aceita por um AP]
    Seja $M=(E,\Sigma,\Gamma,\delta,i,F)$ um APD. A linguagem aceita por $M$,
    $L(M)$, é definida como:
    \[
      L(M) = \{w\in\Sigma^*\,|\,\exists e. e \in F \land [i,w,\lambda]\vdash^*[e,\lambda,\lambda]\}
    \]
  \end{Definition}

  \begin{Example}
    Considere a linguagem de palavras sobre $\Sigma=\{0,1\}$ que possuem o mesmo
    número de 0s e de 1s. O APD para essa linguagem é apresentado a seguir.
    \begin{figure}[H]
      \begin{tikzpicture}
        \node[state, initial, accepting](s0){$=$};
        \node[state, right of=s0](s1){$\neq$};
        \draw (s0) edge[bend left, above]node{$\begin{array}{l}0,\lambda/ZF\\ 1,\lambda/UF\end{array}$}(s1)
        (s1) edge[loop right]node{$\begin{array}{l}0,Z/ZZ\\0,U/\lambda\\1,U/UU\\1,Z/\lambda\end{array}$}(s1)
        (s1) edge[bend left, below]node{$\lambda,F/\lambda$}(s0);
      \end{tikzpicture}
      \centering
    \end{figure}
    Neste exemplo, o APD usa um símbolo (F) para marcar o final da pilha. Como a
    cada 0 encontrado empilha-se um novo Z ou desempilha-se um U (o mesmo
    raciocínio aplica-se para 1), temos que a pilha está vazia apenas quando um
    prefixo contendo a mesma quantidade de 0s e de 1s foi processado. Para
    detectarmos essa fato, usamos o símbolo F como o fundo de pilha.
  \end{Example}

  \section{Exercícios}

  \begin{enumerate}
     \item Construa APDs para as seguintes linguagens.
       \begin{enumerate}
       \item $\{0^n1^{2n}\,|\,n\geq 0\}$
       \item $\{0^{3n}1^{2n}\,|\,n\geq 0\}$
       \item $\{wxw^R\,|\,w\in\{0,1\}^*\}$
       \item $\{0^m1^n\,|\,m < n\}$
       \end{enumerate}
  \end{enumerate}
\end{document}
