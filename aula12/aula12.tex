\documentclass[a4paper]{article}

\usepackage[portuguese]{babel}
\usepackage[utf8]{inputenc}
\usepackage{graphicx,hyperref}
\usepackage{float}
\usepackage{listings}
\usepackage{proof,tikz}
\usepackage{amssymb,amsthm,stmaryrd}


\usepackage[edges]{forest}
\usetikzlibrary{automata, positioning, arrows}


\newtheorem{Lemma}{Lema}
\newtheorem{Theorem}{Teorema}
\theoremstyle{definition}
\newtheorem{Example}{Exemplo}
\newtheorem{Definition}{Definição}


\usepackage{fancyhdr}
  \pagestyle{fancy}
  \fancyhf{}
  \lhead{Teoria da Computação}
  \rhead{Aula 12}
  \lfoot{Prof. Rodrigo Ribeiro}
  \rfoot{\thepage}
  \renewcommand{\footrulewidth}{0.4pt}
  \pagestyle{fancy}

\tikzset{
        -,  % makes the edges directed
        >=stealth', % makes the arrow heads bold
        node distance=3cm,
        every state/.style={thick, fill=gray!10},
        initial text=$\,$
        }
  

\begin{document}

  \title{Aula 12 - Gramáticas livres de contexto}
  \author{Rodrigo Ribeiro}

  \maketitle


  \pagestyle{fancy}


  \section*{Objetivos}

  \begin{itemize}
     \item Apresentar os conceito de gramática livre de contexto.
     \item Apresentar os conceitos de derivação, derivação mais à esquerda e
       mais à direita.
     \item Apresentar o conceito de árvore de derivação.
     \item Apresentar o conceito de ambiguidade.
     \item Definir linguagens livres de contexto.
  \end{itemize}


  \section{Gramáticas livres de contexto}

  \begin{Definition}[Gramática livre de contexto]
    Uma gramática é dita ser livre de contexto $G=(V,\Sigma,R,P)$ (GLC) se
    toda regra possui a forma $X \to w \in R$, em que $X \in V$ e $w \in (V \cup
    \Sigma)^*$.
  \end{Definition}
 
  \begin{Example}
    Como um exemplo de GLC, considere a seguinte gramática que gera
    $\{0^n1^n\,|\,n\geq 0\}$.
    \[
      P \to 0P1\,|\,\lambda
    \]
    Por exemplo, podemos mostrar que $P\Rightarrow^*0011$ usando a seguinte
    derivação.
    \[
      \begin{array}{lcl}
        P & \Rightarrow & \{P \to 0P1\}\\
        0P1 & \Rightarrow & \{P \to 0P1\}\\
        00P11 & \Rightarrow & \{P \to \lambda\}\\
        0011
       \end{array}
    \]
  \end{Example}

  \begin{Example}
    Outro exemplo de GLC é a para a linguagem de palíndromos sobre $\{0,1\}$.
    \[
      P \to 0P0\,|\,1P1\,|\,0\,|\,1\,|\,\lambda
    \]
    Note que as regras dessa gramática mostram a estrutura recursiva de
    palíndromos. Como caso base, temos os palíndromos $\lambda, 0$ e $1$.
    Palíndromos maiores são criados concatenando um 0 (1) à esquerda e a
    direita de um palíndromo.
  \end{Example}

  \begin{Example}
    Considere a linguagem sobre $\Sigma=\{0,1\}$ em que toda palavra
    possui o mesmo número de 0s e de 1s.
    \[
      P \to 0P1P\,|\,1P0P\,|\,\lambda
    \]
    Novamente, temos que as regras dessa gramática exibem a mesma estrutura
    de uma definição recursiva para essa linguagem. São 3 casos, a saber:
    \begin{enumerate}
      \item $\lambda$, que é a menor palavra com o mesmo número de 0s e de 1s.
      \item A palavra começa com um 0: Se a palavra começa com um 0, este possui
        um 1 correspondente em algum ponto da palavra. O mesmo vale para o caso
        da palavra iniciar com 1.
    \end{enumerate}
  \end{Example}

  \begin{Example}
    A gramática a seguir descreve a estrutura de expressões aritméticas
    encontradas em linguagens de programação.
    \[
      \begin{array}{l}
        E \to E + T\,|\,T\\
        T \to T * F\,|\, F \\
        F \to (E) \,|\, n \\
      \end{array}
    \]
    em que $n$ denota um número qualquer. Note que o alfabeto dessa gramática é
    $\Sigma=\{(,),+,*,n\}$.
  \end{Example}

  \begin{Definition}[Linguagem livre de contexto]
    Dizemos que uma linguagem $L\subseteq \Sigma^*$ é uma linguagem livre de
    contexto (LLC) se existe uma gramática livre de contexto que a gera.
  \end{Definition}
  
  \section{Derivações e ambiguidade}

  \begin{Definition}[Forma sentencial]
    Seja $G = (V,\Sigma,R,P)$ uma gramática. Uma forma sentencial para $G$ é uma
    palavra $w \in (V\cup\Sigma)^*$.
  \end{Definition}

  \begin{Definition}[Árvore de derivação]
    Seja $G = (V,\Sigma,R,P)$ uma GLC. Uma árvore de derivação (AD) de uma forma
    sentencial de $G$ é uma árvore ordenada construída recursivamente como se
    segue:
    \begin{enumerate}
      \item uma árvore cujo único vértice possui o rótulo $P$ é uma AD para $P$;
      \item Se $X \in V$ é o rótulo de uma folha $f$ de uma AD A, então:
        \begin{enumerate}
           \item Se $X \to \lambda \in R$, então a árvore de derivação obtida
             adicionando mais um vértice $v$ com rótulo $\lambda$ e uma aresta
             $\{f,v\}$ é uma AD.
           \item Se $X \to x_1...x_n$, $x_1...x_n\in V \cup \Sigma$, então a
             árvore obtida adicionando a A $n$ vértices $v_1,...,v_n$ com os
             rótulos $x_1,...,x_n$ e as arestas $\{f,v_1\},...,\{f,v_n\}$
             é uma AD.
        \end{enumerate}
    \end{enumerate}
  \end{Definition}

  \begin{Example}
    Para exemplificar o conceito de árvore de derivação, vamos considerar a
    seguinte gramática para $\{0^n1^n\,|\,n\geq 0\}$.
    \[
      P \to 0P1\,|\,\lambda
    \]
    A árvore de derivação para $0011$ será:
    \begin{figure}[H]
      \begin{tikzpicture}[
        tlabel/.style={pos=0.4,right=-1pt,font=\footnotesize\color{red!70!black}},
      ]
      \node{P}
      child {node {0}}
      child {node {P} 
        child {node {0}}
        child {node {P}
               child{node{$\lambda$}}}
        child {node {1}}}
      child{node{1}};
    \end{tikzpicture}      
      \centering
    \end{figure}
  \end{Example}


  \begin{Definition}[Gramática ambígua]
    Dizemos que uma GLC $G$ é ambígua se esta possui mais de uma AD para alguma
    sentença que ela gera.
  \end{Definition}


  \begin{Example}
    Vamos considerar uma outra GLC para expressões aritméticas.
    \[
      \begin{array}{l}
        E \to E + E\,|\,E * E\,|\,(E)\,|\,n\\
      \end{array}
    \]
    A seguir, apresentamos duas árvores de derivação distintas para
    $n + n + n$.
    \begin{figure}[H]
      \begin{minipage}{.45\textwidth}
        \begin{tikzpicture}[
          -,tlabel/.style={pos=0.4,right=-1pt,font=\footnotesize\color{red!70!black}},
          ]
          \node{E}
          child{
            node{E}
            child{node{n}}
           }
           child{node{+}}
           child{
             node{E}
             child{
               node{E}
               child{node{n}}
             }
             child{node{+}}
             child{
               node{E}
               child{node{n}}
             }
           };
        \end{tikzpicture}
      \end{minipage} \hfill
      \begin{minipage}{.45\textwidth}
        \begin{tikzpicture}[
          tlabel/.style={pos=0.4,right=-1pt,font=\footnotesize\color{red!70!black}},
          ]
          \node{E}
          child{
            node{E}
            child{
              node{E}
              child{node{n}}
            }
            child{node{+}}
            child{
              node{E}
              child{node{n}}
            }
          }
          child{node{+}}
          child{
            node{E}
            child{node{n}}
          };
        \end{tikzpicture}        
      \end{minipage}
    \end{figure}
    Como a gramática acima gera duas distintas ADs para a mesma sentença, temos
    que essa é ambígua.
  \end{Example}

  \begin{Definition}[Derivação mais a esquerda (DME) e Derivação mais a direita
    (DMD)]
    Uma derivação é dita mais a esquerda (mais a direita) se a cada passo é
    expandida a variáveis mais a esquerda (mais a direita). Usa-se o símbolo
    $\Rightarrow_E$ para representar uma DME e $\Rightarrow_D$ para uma DMD.

    Podemos dizer que uma gramática é ambígua se existem duas ou mais DMEs ou
    DMDs para uma mesma sentença.
  \end{Definition}

  \begin{Example}
    Vamos considerar novamente a seguinte GLC para expressões aritméticas
    \[
      \begin{array}{l}
        E \to E + E\,|\,E * E\,|\,(E)\,|\,n\\
      \end{array}
    \]
    e a seguinte palavra: $n + n + n$. Apresentaremos duas derivações mais a
    direita para essa palavra.

    A primeira derivação é:
    \[
      \begin{array}{lcl}
        E & \Rightarrow_D & \{E \to E + E\}\\
        E + E & \Rightarrow_D & \{E \to E + E\}\\
        E + E + E & \Rightarrow_D & \{E \to n\}\\
        E + E + n & \Rightarrow_D & \{E \to n\}\\
        E + n + n & \Rightarrow_D & \{E \to n\}\\
        n + n + n
      \end{array}
    \]

    A segunda é:

    \[
      \begin{array}{lcl}
        E & \Rightarrow_D & \{E \to E + E\} \\
        E + E & \Rightarrow_D & \{E \to n\} \\
        E + n & \Rightarrow_D & \{E \to E + E\}\\
        E + E + n & \Rightarrow_D & \{E \to n\} \\
        E + n + n & \Rightarrow_D & \{E \to n\} \\
        n + n + n
      \end{array}
    \]
    O que também mostra que a gramática acima é ambígua.
  \end{Example}

  \begin{Definition}[Linguagem inerentemente ambígua]
    Dizemos que uma linguagem é inerentemente ambígua se existem apenas
    gramáticas livres de contexto ambíguas para ela.
  \end{Definition}

  \begin{Example}
    A seguinte linguagem é inerentemente ambígua:
    \[
      \{a^nb^mc^k\,|\,n = m \lor n = k\}
    \]
    Palavras do formato $a^nb^nc^n$, $n \geq 0$, possuem mais de uma AD.
  \end{Example}
  
  \section{Exercícios} 

  \begin{enumerate}
     \item Construa GLCs para as seguintes linguagens.
       \begin{enumerate}
       \item $\{0^n1^n\,|\,n\geq 0\} \cup \{0^n1^{2n}\,|\,n\geq 0\}$
       \item $\{0^n1^n0^k\,|\,n,k \geq 0\}$
       \item $\{0^n1^m\,|\,m > n\}$
       \item $\{a^nb^mc^k\,|\,n = m \lor n = k\}$
       \end{enumerate}
     \item Considerando a gramática que você construiu para $\{a^nb^mc^k\,|\,n =
       m \lor n = k\}$, apresente duas AD e derivações mais a esquerda e a
       direita para a palavra $a^2b^2c^2$ mostrando que sua gramática é ambígua.
  \end{enumerate}
\end{document}
